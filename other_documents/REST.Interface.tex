\documentclass[10pt,a4paper]{scrartcl}
\pagestyle{empty}
\usepackage{a4wide}
\usepackage[ngerman]{babel} % Neudeutsche Silbentrennung (mehrsprachiges Dokument)
\usepackage{parskip} % Skip indentation of first row
\usepackage{graphicx} % Graphics support
\usepackage{longtable} % Tables across several pages
\usepackage{booktabs}
\usepackage{mdwlist} % lists with less spacing, use itemize*
\usepackage{hyperref} % Hyperlinks
\usepackage{float} % Force float position
\usepackage[automark]{scrpage2} %kopf/fusszeile
\usepackage{listings}
\usepackage[utf8x]{inputenc} % Unicode-Encoding
\usepackage[framemethod=TikZ]{mdframed}
\usepackage{mdframed}
\mdfdefinestyle{def}{%
	linecolor=gray!50!white,
	outerlinewidth=0.5pt,
	roundcorner=3pt,
	skipabove=\topskip,
	skipbelow=\topskip
	innertopmargin=\baselineskip,
	innerbottommargin=\baselineskip,
	innerrightmargin=10pt,
	innerleftmargin=10pt,
	nobreak=true
}

\lstset{
	basicstyle=\linespread{.94}\ttfamily,
	tabsize=2,
}

\linespread{1.3}

\author{Danilo Bargen, Christian Fässler, Jonas Furrer} 
\title{strongTNC REST API\\DRAFT \\ \small{Im Rahmen der strongTNC BA} }

\pagestyle{scrheadings}
\ihead{REST API} %linke Kopfzeile
\ohead{strongTNC BA} %rechte Kopfzeile

\let\textquotedbl="

\begin{document}

\begin{titlepage}
	\maketitle
	\vspace{120mm}
	\thispagestyle{empty} % Don't start page numbers on this page
\end{titlepage}

\newpage
	\tableofcontents
\newpage

\section{Einleitung}

Die Definition der Ressourcen orientiert sich an den Regeln des Buches
\textit{REST API Design Rulebook}\cite{masse2011rest} aus dem O'Reilly Verlag.

\subsubsection*{URI Definition}

Bei der Bezeichnung der URIs\footnote{Uniform Resource Identifier} wurde folgende Terminologie gemäss RFC 3986\cite{rfc3986} verwendet:

\texttt{URI = scheme \textquotedbl ://\textquotedbl{} authority \textquotedbl /\textquotedbl{}
path [ \textquotedbl ?\textquotedbl{} query ] [ \textquotedbl \#\textquotedbl{} fragment ]}

\subsubsection*{Ressource-Archetypen}

Nachfolgend die Ressource-Archetypen gemäss Masse 2011\cite{masse2011rest}. Die Erklärungstexte wurden direkt dem besagten
Buch entnommen.

\begin{description}
	\item[Document] A document resource is a singular concept that is akin to an object instance or database
		record. A document’s state representation typically includes both fields with values and
		links to other related resources.
	\item[Collection] A collection resource is a server-managed directory of resources. Clients may								 propose
		new resources to be added to a collection. However, it is up to the collection to choose
		to create a new resource, or not.
	\item[Store] A store is a client-managed resource repository. A store resource lets an API client put
		resources in, get them back out, and decide when to delete them. On their own, stores
		do not create new resources; therefore a store never generates new URIs. Instead, each
		stored resource has a URI that was chosen by a client when it was initially put into the
		store.
	\item[Controller] A controller resource models a procedural concept. Controller resources are like
		executable functions, with parameters and return values; inputs and outputs.
		Like a traditional web application’s use of HTML forms, a REST API relies on controller
		resources to perform application-specific actions that cannot be logically mapped to
		one of the standard methods (create, retrieve, update, and delete, also known as
		CRUD).
\end{description}

\pagebreak
\section{Repräsentation der Archetypen}

Die JSON-Repräsentation ist abhängig vom Ressource-Archetypen.

\begin{description}
	\item[Document] Ein Document gibt ein JSON Objekt zurück, welches alle relevanten Felder enthält.
		Informationen, welche durch die Ressource-URI bereits gegeben sind (z.B. Type), müssen nicht, können jedoch, erneut in der Liste erscheinen.\\
		Falls im Document Unterobjekte auftauchen, wird für das Feld die URI des betreffenden Unterobjekts
		eingesetzt, damit kann dieses Objekt direkt abgefragt werden. Zusätzlich wird ein Query-Parameter
		bereit gestellt, welcher es erlaubt, Unterobjekte bis zu einer gewissen Tiefe aufzulösen.\\
		Beispiel: \hfill
\begin{lstlisting}
{
	"field1": "<str,value-1>",
	"field2": <int,value-2>,
	"field3": <bool,value-3>,
	"subObject": "<uri,sub-object>"
	"subCollection": [
		{
			"uri": "<uri,sub-object>"
		},
		{
			"uri": "<uri,sub-object>"
		}
	]
}
\end{lstlisting}
Beispiel mit Query \texttt{depth=1}:
\begin{lstlisting}
{
	"field1": "<str,value-1>",
	"field2": <int,value-2>,
	"field3": <bool,value-3>,
	"subObject": {
		"field1": <int,value-1>,
		...
		"uri": "<uri,sub-object>"
	}
	"subCollection": [
		{
			"field": <int, field>,
			...
			"uri": "<uri,sub-object>"
		},
		{
			"field": <int, field>,
			...
			"uri": "<uri, sub-object>"
		}
	]
}
\end{lstlisting}

	\item[Collection] Eine Collection gibt eine Liste mit allen enthaltenen (ggf. gefilterten)
		JSON-Objekten zurück. Die Objekte werden zusätzlich um ein Feld \texttt{uri} ergänzt,
		welches die Ressource-URI des jeweiligen Objektes enthält. Wie bei einem Document gibt es bei
		Collections einen optionalen Query Parameter welcher es erlaubt Unterobjekte aufzulösen, \\
		Beispiel: \hfill
\begin{lstlisting}
{
	{
		"field1": "<str,value-1>",
		"subObject": "<uri,sub-object>",
		"uri": "<uri,objek>"
	},
	{
		"field1": "'<str,value-1>",
		"subObject": "<uri,sub-object>",
		"uri": "<uri,resource>"
	},
}
\end{lstlisting}
Beispiel mit Query \texttt{depth=1}:
\begin{lstlisting}
{
	{
		"field1": "<str,value-1>",
		"subObject": {
			"field": "<str,value>",
		},
		"uri": "<uri,resource>"
	},
	{
		"field1": "'<str,value-1>",
		"subObject": {
			"field": "<str,value>",
		},
		"uri": "<uri,resource>"
	},
}
\end{lstlisting}

	\item[Store] Ein Store verhält sich wie eine Collection, wenn sie direkt angesprochen wird
		(\texttt{/store}) und wie ein Document, wenn ein spezifisches Element des Store angesprochen
		wird (\texttt{/store/\{element-id\}}).
	\item[Controller] Der Output des Controllers ist abhängig vom Verwendungszweck.
\end{description}

\pagebreak
\section{HTTP Statuscodes}
Das Ergebnis einer Anfrage wird grundsätzlich über einen HTTP Statuscode mitgeteilt, gewissen Fällen
kann auch noch Payload im Body geliefert werden. Folgendes ist eine Auswahl von Statuscodes die zu
erwarten sind:

\begin{description*}
	\item[200 OK] Request erfolgreich
	\item[201 Created] Entity wurde erfolgreich erstellt
	\item[204 No Content] Entity wurde erfolgreich geändert
	\item[400 Bad Request] Generischer Client Fehler
	\item[404 Not Found] Zu ändernde Entity nicht gefunden
	\item[405 Method Not Allowed] Die verwendete HTTP Methode ist auf dieser Ressource nicht erlaubt
	\item[409 Conflict] Zu erstellende Entity existiert bereits
	\item[412 Precondition Failed] Es sind zusätzliche Schritte nötig um die Anfrage auszuführen.
	\item[500 Internal Server Error] Generischer Server Fehler
\end{description*}

Falls zusätzliche Codes verwendet werden oder die Codes eine spezielle oder erwähnenswerte Bedeutung haben, ist dies bei der betroffenen Resource vermerkt.


\pagebreak
\section{Allgemeine Hinweise}

\subsection{Schreibweise}
Für die Definition des Response- und Request-Formats wird eine pseudo JSON schreibweise verwendet. Die
auszufüllenden Werte werden als Tupel in spitzen Klammern dargestellt. Der erste Teil des Tupels zeigt 
den Typ, der zweite Teil steht zur Beschreibung für den Wert. Folgende Typen werden verwendet:
\begin{description*}
	\item[int] Integer Zahlenwert
	\item[num] Dezimalzahl, 
	\item[str] String
	\item[bool] Wahrheitswert
	\item[xml] Ein XML Dokument
	\item[hex] Ein HEX-Encoded String
	\item[uri] Die voll qualifizerte URI einer Resource
	\item[doc] Das Dokument einer Resource, wie dieses Aussieht kann im jeweiligen Abschnitt zur genannten Resource nachgeschlagen werden
\end{description*}

\subsection{Standard Verhalten}
Folgende Punkte beschreiben das standard Verhalten der Resourcen:
\begin{description*}
	\item[Depth Query-Parameter] Jede Resource die als Antwort eine URI auf eine
		andere Resource zurück gibt unterstützt den \texttt{depth} Query-Parameter.
		Dieser löst die URIs bis zur gewünschten tiefe auf.  Möglicherweise wird die
		maximale Tiefe eingeschränkt. Wenn keine tiefe angegeben wird, gilt die
		Auflösungstiefe 0.
	\item[Filter Query-Parameter] Filter Query-Parameter sind, falls vorhanden,
		immer optional, sie dienen der Präzisierung des Rückgabesets. Wenn sie
		weggelassen werden, wird das vollständige Set zurückgegeben.
	\item[Generische Filter] Collections unterstützen wenn nicht anders erwähnt
		generische Filter auf allen Feldern mit Ausnahme der Foreign Keys. Das
		heisst, es kann grundsätzlich nach jedem Attribut des Rückgabedokuments
		gefiltert werden. Der Parameter ist wie folgt aufgebaut:
		\texttt{attributeName=query}. Filter können auch auf Document-Resourcen
		genutzt werden, dann wird bei nicht passenden Filterwerten ein HTTP 404
		Status Code zurückgegeben. 
	\item[Batch Create] Collections unterstützen, sofern sie nicht readonly sind
		oder es anders erwähnt ist, das sogenannte 'Batch Create'. D.h zum Erstellen
		von neuen Dokumenten kann einer Collection auch eine Liste gesendet werden.
		Wird 'Batch Create' verwendet, besteht die Antwort des Servers nur aus einem
		Statuscode, es wird nicht eine Liste der Erstellten Dokumente zurück
		geschickt. Fall die Erstellung fehlschlägt wird kein Dokument angelegt. Es
		wird versucht mitzuteilen warum die Erstellung fehlschlug.
	\item[Paging] //TODO: Beschreiben//  
\end{description*}
 
\pagebreak
\subsection{Bisher verwendete SQL Vorgänge}
\paragraph{Session}
\begin{description*}
	\item[SQL] \texttt{SELECT/INSERT/JOIN(devices,results)}
	\item[Felder] *id, *device, *time, connection, product, identity, +rec
\end{description*}

\paragraph{Workitem}
\begin{description*}
	\item[SQL] \texttt{SELECT/INSERT/DELETE/JOIN(enforcements)}
	\item[Felder] *id, *session, *result, enforcement, type, arg\_str, arg\_int, rec\_fail, rec\_noresult, rec\_final
\end{description*} 
 
\paragraph{Products}
\begin{description*}
	\item[SQL] \texttt{SELECT/INSERT}
	\item[Felder] id, *name
\end{description*}

\paragraph{Packages}
\begin{description*}
	\item[SQL] \texttt{SELECT/INSERT}
	\item[Felder] id, *name
\end{description*}

\paragraph{Versions}
\begin{description*}
	\item[SQL] \texttt{SELECT/INSERT/DELETE}
	\item[Felder] *id, *product, *package, release, security, blacklist, time
\end{description*}

\paragraph{Identities}
\begin{description*}
	\item[SQL] \texttt{SELECT/INSERT}
	\item[Felder] id, *type, *value
\end{description*}

\paragraph{Devices}
\begin{description*}
	\item[SQL] \texttt{SELECT/INSERT/JOIN(sessions)}
	\item[Felder] *id, *value, *product, trusted, created, +desription
\end{description*}

\paragraph{Enforcements}
\begin{description*}
	\item[SQL] \texttt{SELECT/JOIN(policies,workitems)}
	\item[Felder] *id, *group\_id, policy, rec\_fail, rec\_noresult, max\_age
\end{description*}

\paragraph{Policies}
\begin{description*}
	\item[SQL] \texttt{SELECT/JOIN(enforcements)}
	\item[Felder] *id, type, argument, rec\_fail, rec\_noresult, file, dir, +name
\end{description*}

\paragraph{Results}
\begin{description*}
	\item[SQL] \texttt{SELECT/INSERT/JOIN(sessions)}
	\item[Felder] *policy, session, rec, result, +id
\end{description*}

\paragraph{Groups}
\begin{description*}
	\item[SQL] \texttt{SELECT}
	\item[Felder] *id, parent, +name
\end{description*}

\paragraph{Groups\_members}
\begin{description*}
	\item[SQL] \texttt{SELECT/INSERT}
	\item[Fe•lder] *device\_id, group\_id, +id
\end{description*}

\paragraph{Groups\_products\_defaults}
\begin{description*}
	\item[SQL] \texttt{SELECT}
	\item[Felder] *product\_id, group\_id, +id
\end{description*} 

\pagebreak
\section{Daten Definition}

\subsection{Policy Arguments}

Folgende Workitem, bzw. Policy Typen sind momentan vorhanden:
\begin{description*}
	\item[\texttt{00: RESVD}] Deny
	\item[\texttt{01: PCKGS}] Installed Packages
	\item[\texttt{02: UNSRC}] Unknown Source
	\item[\texttt{03: FWDEN}] Forwarding Enabled
	\item[\texttt{04: PWDEN}] Default Password Enabled
	\item[\texttt{05: FREFM}] File Reference Measurement
	\item[\texttt{06: FMEAS}] File Measurement 
	\item[\texttt{07: FMETA}] File Metadata
	\item[\texttt{08: DREFM}] Directory Reference Measurement
	\item[\texttt{09: DMEAS}] Directory Measurement
	\item[\texttt{10: DMETA}] Directory Metadata
	\item[\texttt{11: TCPOP}] Open TCP Listening Ports
	\item[\texttt{12: TCPBL}] Blocked TCP Listening Ports
	\item[\texttt{13: UDPOP}] Open UDP Listening Ports
	\item[\texttt{14: UDPBL}] Blocked UDP Listening Ports
	\item[\texttt{15: SWIDT}] SWID Tag Inventory
	\item[\texttt{16: TPMRA}] TPM Remote Attestation
\end{description*}

Einige Typen benötigen unterschiedliche Argumente, Gemeinsamkeiten lassen sich wie folgt gruppieren:
\begin{description*}
	\item[Keine Argumente] 00, 01, 02, 03, 04
	\item[Datei Pfad] 05, 06, 07
	\item[Verzeichnis Pfad] 08, 09, 10
	\item[Port Liste] 11, 12, 13, 14
	\item[SWID Request Flags] 15
	\item[TPM Attestation Flags] 16
\end{description*}
Entsprechend sollen die Argumente der Workitems übermittelt werden. Folgende Objekte werden je nach Type übermittelt, als Verweis wir \texttt{<doc,policy-argument>} verwendet.
\begin{description*}
	\item[Keine Argumente] \hfill
\begin{lstlisting}
{}
\end{lstlisting}
	\item[Datei Pfad] \hfill
\begin{lstlisting}
{
	"file": "<str,file-path>"
}
\end{lstlisting}   
	\item[Verzeichnis Pfad] \hfill
\begin{lstlisting}
{
	"directory": "<str,directory-path>"
}
\end{lstlisting} 
	\item[Port Liste] \hfill
\begin{lstlisting}
{
	"portList": [
		"<str,port or port-range>"
	]
}
\end{lstlisting} 
	\item[SWID Request Flags] \hfill
\begin{lstlisting}
{
	"swidFlags": [
		"<str,swid-flag>"
	]
}
\end{lstlisting} 
	\item[TPM Attestation Flags] \hfill
\begin{lstlisting}
{
	"tpmFlags": [
		"<str,tpm-flag>"
	]
}
\end{lstlisting} 
\end{description*}

\subsection{Recommendation Types}
Folgende Recommendation Types bzw. Actions sind vorhanden:
\begin{description*}
	\item[\texttt{0: ALLOW}] Die Empfehlung/Aktion ist, den Client zuzulassen
	\item[\texttt{1: BLOCK}] Die Empfehlung/Aktion ist, den Client zu blockieren
	\item[\texttt{2: ISOLATE}] Die Empfehlung/Aktion ist, den Client in einem isoliertem Segment zu platzieren
	\item[\texttt{3: NONE}]
\end{description*}

Diese Werte werden von der API als Id verwendet. Es wird kein Objekt für die Repräsentation erstellt.

\subsection{Hash-Set}
Ein Hash-Set wird innerhalb eines File-Documents ausgeliefert, der Verweis lautet \texttt{<doc,hash-set>} und ist wie folgt aufgebaut:

\begin{lstlisting}
[
	{
		"algorithm": "<str,algorithm-name>",
		"hash": "<hex,hash>	"
	}
]
\end{lstlisting}

Falls ein Algorithmus noch nicht existiert wird er erfasst, momentan sind folgende Algorithmen erfasst:
\begin{itemize*}
	\item SHA384
	\item SHA256
	\item SHA1
	\item SHA1-IMA
\end{itemize*}


\pagebreak
\section{REST Ressourcen}

\subsection{Session Steuerung und Ablauf}

\subsubsection{Controller}

\begin{mdframed}[style=def]
\begin{description*}
	\item[URI Path] \texttt{/sessions/start/}
	\item[Archetype] Controller
	\item[Methods] POST
	\item[Request Parameter] \hfill
	\begin{description*}
		\item[\texttt{connectionId}] strongSwan Connection Id
		\item[\texttt{clientIdentity}] strongSwan Client-Identity
		\item[\texttt{hardwareId}] Die Id, welche das Gerät identifiziert, so zum Beispiel, AIK, Android-Id, DBUS machine-id, o.ä. Dies entspricht dem \texttt{value} Feld in der \texttt{device} Tabelle in der Datenbank
		\item[\texttt{productName}] Der Productname ist der Name des OS wie er in der \texttt{product} Tabelle der Datenbank steht
	\end{description*}
	\item[JSON Format Response] \hfill
\begin{lstlisting}
{
	"sessionId": <int,id>,
	"workitems": [
		 <doc,workitem>,
		 ...
	],
	"uri": "<uri,session>"
}
\end{lstlisting}
	\item[Beschreibung] Dieser Controller erstellt und startet eine Session, das Device welches der Session zugeordnet werden soll wird anhand der \texttt{hardwareId} und dem \texttt{productName} bestimmt. Falls eines der Objekte noch nicht existiert wird dieses durch den Controller erstellt. Die Id, die im Response Dokument zurück geliefert wird, dient zur zukünftigen Identifikation der soeben gestarteten Session. Ausserdem erhält man eine Liste von Workitems die für diese Session abgearbeitet werden müssen.
\end{description*}
\end{mdframed}


\begin{mdframed}[style=def]
\begin{description*}
	\item[URI Path] \texttt{/sessions/\{id\}/end/}
	\item[Archetype] Controller
	\item[Methods] POST
	\item[Request Parameter] \hfill
	\begin{description*}
		\item[\texttt{recommendation}] Endgültiges Resultat/Empfehlung für diese Session.
	\end{description*}
	\item[Beschreibung] Dieser Controller schliesst die Session ab und startet alle zusätzlich nötigen Vorgänge, so werden zum Beispiel die Workitems dieser Session abgeräumt und die jeweiligen Resultate gespeichert und können via \texttt{/sessions/\{id\}/results} abgerufen werden.
\end{description*}
\end{mdframed}

\pagebreak
\subsubsection{Documents und Collections}

\begin{mdframed}[style=def]
\begin{description*}
	\item[URI Path] \texttt{/sessions/\{id\}/}
	\item[Archetype] Readonly Document
	\item[Methods] GET
	\item[JSON Format Response] \hfill
\begin{lstlisting}
{
	"id": <int,id>,
	"uri": "<uri,resource>",
	"time": <int,time>,
	"identity": "<uri,identity>",
	"connectionId": <int,connection-id>,
	"device": "<uri,device>",
	"recommendation": <int,rec>
}
\end{lstlisting}
	\item[Beschreibung] Informationen zu einer bestimmten Session. Sessions sollen nicht direkt geändert werden, sondern nur über die entsprechenden Controller, der Grund dafür ist, dass im Hintergrund noch zusätzliche Operationen vorgenommen werden müssen.
\end{description*}
\end{mdframed}

\begin{mdframed}[style=def]
\begin{description*}
	\item[URI Path] \texttt{/sessions/}
	\item[Archetype] Readonly Collection
	\item[Filter Query] \hfill
	\begin{description*}
		\item[timeFrom] \texttt{<int,timestamp>}
		\item[timeTo] \texttt{<int,timestamp>}
	\end{description*}
	\item[Methods] GET
	\item[JSON Format Response] \hfill
\begin{lstlisting}
[<doc,session>, ...]
\end{lstlisting}
	\item[Beschreibung] Liste aller Sessions. Neue Sessions können nur über den entsprechenden Controller erstellt werden.
\end{description*}
\end{mdframed}

\pagebreak
\paragraph{Workitems}\hfill \\

\begin{mdframed}[style=def]
\begin{description*}
	\item[URI Path] \texttt{/session/\{id\}/workitems/\{id\}/}
	\item[Archetype] Readonly Document
	\item[Methods] GET
	\item[JSON Format Response] \hfill
\begin{lstlisting}
{
	"id": <int,id>,
	"uri": "<uri,resource>",
	"session": "<uri,session>",
	"type": <int,type>,
	"argument": <policy-argument>
}
\end{lstlisting}
	\item[Beschreibung] Ein zu einer bestimmten Session zugehöriges Workitem.
		Workitems können nicht direkt erstellt werden, sondern werden beim Erstellen
		einer Session anhand der Enforcements erstellt. Workitems gibt es nur, so
		lange eine Session nicht beendet wurde. Nach dem Ende der Session wird ein
		HTTP404 Status Code zurückgeliefert.
\end{description*}
\end{mdframed}

\begin{mdframed}[style=def]
\begin{description*}
	\item[URI Path] \texttt{/session/\{id\}/workitems/}
	\item[Archetype] Readonly Collection
	\item[Filter Query] \hfill
	\begin{description*}
		\item[type] \texttt{<int,policy-type>}
	\end{description*}
	\item[Methods] GET
	\item[JSON Format Response] \hfill
\begin{lstlisting}
[<doc,workitem>, ...]
\end{lstlisting}
	\item[Beschreibung] Eine Liste aller Workitems die zu einer bestimmten Session
		gehören. Workitems können nicht direkt erstellt werden, sondern werden beim
		Erstellen einer Session anhand der Enforcements erstellt. Workitems gibt es
		nur, so lange eine Session nicht beendet wurde. Nach dem Ende der Session
		wird eine leere Liste geliefert.
\end{description*}
\end{mdframed}


\begin{mdframed}[style=def]
\begin{description*}
	\item[URI Path] \texttt{/session/\{id\}/workitems/\{id\}/result/}
	\item[Archetype] Document
	\item[Request Parameter] \hfill
	\begin{description*}
		\item[\texttt{recommendation}] Resultat/Empfehlung für dieses Workitem.
		\item[\texttt{comment}] Kommentar zum Resultat.
	\end{description*}
	\item[Methods] GET, POST
	\item[Response Statuscodes] \hfill
		\begin{description*}
			\item[201 Created] Resultat wurde erfolgreich gespeichert.
			\item[409 Conflict] Resultat existiert bereits.
		\end{description*}
	\item[JSON Format Response] \hfill
\begin{lstlisting}
{
	"recommendation": <int,type>,
	"comment": <str,comment>
}
\end{lstlisting}
	\item[Beschreibung] Auf dieser Ressource soll das Resultat des jeweiligen
		Workitems eingetragen werden. Diese Resultate werden beim Beenden der
		Session von den Workitems in die Session-Resulate übertragen.
\end{description*}
\end{mdframed}

\pagebreak
\paragraph{Results} \hfill \\
\nopagebreak

\begin{mdframed}[style=def]
\begin{description*}
	\item[URI Path] \texttt{/session/\{id\}/results/\{id\}/}
	\item[Archetype] Readonly Collection
	\item[Filter Query] \hfill
	\item[Methods] GET
	\item[JSON Format Response] \hfill
\begin{lstlisting}
{
	"id": <int,id>,
	"uri": "<uri,resource>",
	"enforcement": "<uri,enforcement>",
	"recommendation": <int,type>,	 
	"comment": "<str,comment>"
}
\end{lstlisting}
	\item[Beschreibung] Einzelnes Resultat. Im Gegensatz zur Resultat-Resource
		unter der Workitem Resource enthält dieses Document auch einen Link zum
		zugehörigen Enforcement.
\end{description*}
\end{mdframed}

\begin{mdframed}[style=def]
\begin{description*}
	\item[URI Path] \texttt{/session/\{id\}/results/}
	\item[Archetype] Readonly Collection
	\item[Filter Query] \hfill
	\item[Methods] GET
	\item[JSON Format Response] \hfill
\begin{lstlisting}
[<doc,result>, ...]
\end{lstlisting}
	\item[Beschreibung] Nach dem Beenden einer Session können die eingetragenen Resultate in dieser Collection abgefragt werden. Diese Collection ist readonly, damit die Session-Ergebnisse nicht nachträglich geändert werden können.
\end{description*}
\end{mdframed}




\pagebreak
\subsection{SWID Erweiterung}

\subsubsection{SWID Tags: Messung}

\begin{mdframed}[style=def]
\begin{description*}
	\item[URI Path] \texttt{/session/\{id\}/register-swid-measurement/}
	\item[Archetype] Controller
	\item[Methods] POST
	\item[Content-Type] \texttt{application/json; charset=utf-8}
	\item[Request Parameter] \hfill
	\begin{description*}
		\item[\texttt{softwareId}] Software-IDs als JSON-Liste.
	\end{description*}
	\item[Response Statuscodes] \hfill
		\begin{description*}
			\item[200 OK] SWID Tags der übermittelten Software-IDs sind eingetragen \\
				und wurden für die übermittelte Session eingetragen.
			\item[404 Not Found] Session mit der spezifizierten ID wurde nicht gefunden. 
			\item[412 Precondition Failed] Es existieren nicht alle SWID Tags für die
				übertragenen Software-IDs. Als Payload werden die fehlenden Software-IDs
				übertragen.
		\end{description*}
	\item[JSON Format Response] \hfill
\begin{lstlisting}
["<str,software-id>", ...]
\end{lstlisting}
	\item[Beschreibung] Die strongSwan Komponente sendet eine Liste von
		Software-IDs an die strongTNC Schnittstelle. Die Software-IDs wurden zuvor
		auf dem Client gemessen und widerspiegeln die momentan installierten
		Software Pakete.\\
		Wenn bereits SWID Tags für alle übertragenen Software-IDs bestehen, können
		diese direkt eingetragen werden.
\end{description*}
\end{mdframed}


\subsubsection{SWID Tags: Erstellung}

\begin{mdframed}[style=def]
\begin{description*}
	\item[URI Path] \texttt{/swid/add-tags/}
	\item[Archetype] Controller
	\item[Methods] POST
	\item[Content-Type] \texttt{application/json; charset=utf-8}
	\item[Request Parameter] \hfill
		\begin{description*}
			\item[\texttt{xmlData}] SWID Tag als JSON-Liste.
		\end{description*}
	\item[Response Statuscodes] \hfill
		\begin{description*}
			\item[200 OK] SWID Tags wurden erfolgreich verarbeitet.
			\item[400 Bad Request] Fehlerhafter Request. Details zum Fehler werden im
				Response-Body zurückgesendet.
		\end{description*}
	\item[Beschreibung] Die übermittelten Tags werden gelesen und in die Datenbank
		gespeichert. Bereits vorhandene Tags werden übersprungen, beziehungsweise
		ergänzt, wenn der neu übermittelte Tag mehr optionale Felder enthält als der
		bereits gespeicherte. (TODO specify genauer)
\end{description*}
\end{mdframed}

\pagebreak
\subsection{CRUD Resourcen}
\subsubsection{Products}

\begin{mdframed}[style=def]
\begin{description*}
	\item[URI Path] \texttt{/products/\{id\}/}
	\item[Archetype] Document
	\item[Methods] GET, PUT, PATCH
	\item[JSON Format Response] \hfill
\begin{lstlisting}
{
	"id": <int,id>,
	"uri": "<uri,resource>",
	"name": "<str,productname>"
}
\end{lstlisting}
\end{description*}
\end{mdframed}

\begin{mdframed}[style=def]
\begin{description*}
	\item[URI Path] \texttt{/products/}
	\item[Archetype] Collection
	\item[Methods] GET, POST
	\item[JSON Format Response] \hfill
\begin{lstlisting}
[
	{
		"id": <int,id>,
		"uri": "<uri,resource>",
		"name": "<str,productname>"	
	}
]
\end{lstlisting}
\end{description*}
\end{mdframed}

\begin{mdframed}[style=def]
\begin{description*}
	\item[URI Path] \texttt{/products/\{id\}/default-groups/}
	\item[Archetype] Collection
	\item[Methods] GET, POST
	\item[Request Parameter] \hfill
	\begin{description*}
		\item[\texttt{groupId}] group-id
	\end{description*}
	\item[JSON Format Response] \hfill
\begin{lstlisting}
[<doc,group>, ...]
\end{lstlisting}
\end{description*}
\end{mdframed}


\pagebreak
\subsubsection{Packages}

\begin{mdframed}[style=def]
\begin{description*}
	\item[URI Path] \texttt{/packages/\{id\}/}
	\item[Archetype] Document
	\item[Methods] GET, PUT, PATCH
	\item[JSON Format Response] \hfill
\begin{lstlisting}
{
	"id": <int,id>,
	"uri": "<uri,resource>",
	"name": "<str,packagename>"
}
\end{lstlisting}
\end{description*}
\end{mdframed}

\begin{mdframed}[style=def]
\begin{description*}
	\item[URI Path] \texttt{/packages/}
	\item[Archetype] Collection
	\item[Methods] GET, POST
	\item[JSON Format Response] \hfill
\begin{lstlisting}
[
	{
		"id": <int,id>,
		"uri": "<uri,resource>",
		"name": "<str,packagename>",
	}
]
\end{lstlisting}
\end{description*}
\end{mdframed}

\begin{mdframed}[style=def]
\begin{description*}
	\item[URI Path] \texttt{/packages/\{id\}/versions/}
	\item[Archetype] Collection
	\item[Methods] GET, POST
	\item[JSON Format Response] \hfill
\begin{lstlisting}
[<doc,version>, ...]
\end{lstlisting}
\end{description*}
\end{mdframed}


\pagebreak
\subsubsection{Versions}

\begin{mdframed}[style=def]
\begin{description*}
	\item[URI Path] \texttt{/versions/\{id\}/}
	\item[Archetype] Document
	\item[Methods] GET, PUT, PATCH, DELETE
	\item[JSON Format Response] \hfill
\begin{lstlisting}
{
	"id": <int,id>,
	"uri": "<uri,resource>",
	"package": "<uri,package>",
	"product": "<uri,product>",
	"release": "<str,release>",
	"securtiy": <int,security>,
	"blacklist": <bool,blacklist>,
	"time": <int,time>
}
\end{lstlisting}
\end{description*}
\end{mdframed}

\begin{mdframed}[style=def]
\begin{description*}
	\item[URI Path] \texttt{/versions/}
	\item[Archetype] Collection
	\item[Filter Query] \hfill
	\begin{description*}
		\item[productName] \texttt{<str,product-name>}
		\item[packageName] \texttt{<str,package-name>}
	\end{description*}
	\item[Methods] GET, POST
	\item[JSON Format Response] \hfill
\begin{lstlisting}
[
	{
		"id": <int,id>,
		"uri": "<uri,resource>",
		"package": "<uri,package>",
		"product": "<uri,product>",
		"release": "<str,release>",
		"securtiy": <int,security>,
		"blacklist": <bool,blacklist>,
		"time": <int,time>
	}
]
\end{lstlisting}
\end{description*}
\end{mdframed}


\pagebreak
\subsubsection{Identities}
\begin{mdframed}[style=def]
\begin{description*}
	\item[URI Path] \texttt{/identities/\{id\}/}
	\item[Archetype] Document
	\item[Methods] GET, PUT, PATCH
	\item[JSON Format Response] \hfill
\begin{lstlisting}
{
	"id": <int,id>,
	"uri": "<uri,resource>",
	"type": <int,type>,
	"value": "<str,value>"
}
\end{lstlisting}
\end{description*}
\end{mdframed}

\begin{mdframed}[style=def]
\begin{description*}
	\item[URI Path] \texttt{/identities/}
	\item[Archetype] Collection
	\item[Methods] GET, POST
	\item[JSON Format Response] \hfill
\begin{lstlisting}
[
	{
		"id": <int,id>,
		"uri": "<uri,resource>",
		"type": <int,type>,
		"value": "<str,value>"
	}
]
\end{lstlisting}
\end{description*}
\end{mdframed}


\pagebreak
\subsubsection{Devices}

\begin{mdframed}[style=def]
\begin{description*}
	\item[URI Path] \texttt{/devices/\{id\}/}
	\item[Archetype] Document
	\item[Methods] GET, PUT, PATCH
	\item[JSON Format Response] \hfill
\begin{lstlisting}
{
	"id": <int,id>,
	"uri": "<uri,resource>",
	"description": "<str,description>",
	"value": "<str,value>",
	"product": "<uri,product>",
	"created": <int,created>
}
\end{lstlisting}
\end{description*}
\end{mdframed}

\begin{mdframed}[style=def]
\begin{description*}
	\item[URI Path] \texttt{/devices/}
	\item[Archetype] Collection
	\item[Methods] GET, POST
	\item[JSON Format Response] \hfill
\begin{lstlisting}
[
	{
		"id": <int,id>,\\
		"uri": "<uri,resource>",
		"description": "<str,description>",
		"value": "<str,value>",
		"product": "<uri,product>",
		"created": <int,created>,
	}
]
\end{lstlisting}
\end{description*}
\end{mdframed}

\begin{mdframed}[style=def]
\begin{description*}
	\item[URI Path] \texttt{/device/\{id\}/sessions/}
	\item[Archetype] Readonly Collection
	\item[Filter Query] \hfill
	\begin{description*}
		\item[timeFrom] \texttt{<int,timestamp>}
		\item[timeTo] \texttt{<int,timestamp>}
	\end{description*}	
	\item[Methods] GET
	\item[JSON Format Response] \hfill
\begin{lstlisting}
[<doc,session>, ...]
\end{lstlisting}
\end{description*}
\end{mdframed}

\begin{mdframed}[style=def]
\begin{description*}
	\item[URI Path] \texttt{/device/\{id\}/sessions/\{id\}/results/}
	\item[Archetype] Readonly Collection 
	\item[Methods] GET
	\item[JSON Format Response] \hfill
\begin{lstlisting}
[<doc,result>, ...]
\end{lstlisting}
\end{description*}
\end{mdframed}

\begin{mdframed}[style=def]
\begin{description*}
	\item[URI Path] \texttt{/device/\{id\}/groups/}
	\item[Archetype] Collection 
	\item[Methods] GET, POST
	\item[JSON Format Response] \hfill
\begin{lstlisting}
[<doc,group>, ...]
\end{lstlisting}
\end{description*}
\end{mdframed}

\begin{mdframed}[style=def]
\begin{description*}
	\item[URI Path] \texttt{/device/\{id\}/swid-tags/}
	\item[Archetype] Readonly Collection 
	\item[Methods] POST
	\item[JSON Format Response] \hfill
\begin{lstlisting}
[<doc,swid-tag>, ...]
\end{lstlisting}
\end{description*}
\end{mdframed}


\pagebreak
\subsubsection{Enforcements}
\begin{mdframed}[style=def]
\begin{description*}
	\item[URI Path] \texttt{/enforcements/\{id\}/}
	\item[Archetype] Readonly Document
	\item[Methods] GET
	\item[JSON Format Response] \hfill
\begin{lstlisting}
{
	"id": <int,id>,
	"uri": "<uri,resource>",
	"policy": "<uri,policy>",
	"group": "<uri,group>",
	"failRecommendation": <int,rec_fail>,
	"noresultRecommendation": <int,rec_noresult>", 
	"maxAge": <int,max_age>
}
\end{lstlisting}
\end{description*}
\end{mdframed}

\begin{mdframed}[style=def]
\begin{description*}
	\item[URI Path] \texttt{/enforcements/}
	\item[Archetype] Readonly Collection
	\item[Methods] GET
	\item[Filter Query] \hfill
	\begin{description*}
		\item[groupName] \texttt{<str,group-name>}
		\item[policyName] \texttt{<str,policy-name>}
	\end{description*}	
	\item[JSON Format] \hfill
\begin{lstlisting}
[
	{
		"id": <int,id>,
		"uri": "<uri,resource>",
		"policy": "<uri,policy>",
		"group": "<uri,group>",
		"failRecommendation": <int,rec_fail>,
		"noresultRecommendation": <int,rec_noresult>",
		"maxAge": <int,max_age>
	}
]
\end{lstlisting}
\end{description*}
\end{mdframed}

\begin{mdframed}[style=def]
\begin{description*}
	\item[URI Path] \texttt{/enforcements/\{id\}/groups/}
	\item[Archetype] Readonly Collection
	\item[Methods] GET
	\item[JSON Format] \hfill
\begin{lstlisting}
[<doc,group>, ...]
\end{lstlisting}
\end{description*}
\end{mdframed}

\pagebreak
\subsubsection{Policies}

\begin{mdframed}[style=def]
\begin{description*}
	\item[URI Path] \texttt{/policies/\{id\}/}
	\item[Archetype] Readonly Document
	\item[Methods] GET
	\item[JSON Format Response] \hfill
\begin{lstlisting}
{
	"id": <int,id>,
	"uri": "<uri,resource>",
	"type": <int,type>,
	"name": "<str,name>",
	"argument": <policy-argument>,
	"failRecommendation": <int,rec_fail>,
	"noresultRecommendation": <int,rec_noresult>",
}
\end{lstlisting}
\end{description*}
\end{mdframed}

\begin{mdframed}[style=def]
\begin{description*}
	\item[URI Path] \texttt{/policies/}
	\item[Archetype] Readonly Collection
	\item[Methods] GET
	\item[JSON Format] \hfill
\begin{lstlisting}
[
	{
		"id": <int,id>,
		"uri": "<uri,resource>",
		"type": <int,type>,
		"name": "<str,name>",
		"argument": <policy-argument>,
		"failRecommendation": <int,rec_fail>,
		"noresultRecommendation": <int,rec_noresult>"
	}
]
\end{lstlisting}
\end{description*}
\end{mdframed}

\begin{mdframed}[style=def]
\begin{description*}
	\item[URI Path] \texttt{/policies/\{id\}/enforcements/}
	\item[Archetype] Readonly Collection
	\item[Methods] GET
	\item[JSON Format] \hfill
\begin{lstlisting}
[<doc,enforcement>, ...]
\end{lstlisting}
\end{description*}
\end{mdframed}


\pagebreak
\subsubsection{Groups}

\begin{mdframed}[style=def]
\begin{description*}
	\item[URI Path] \texttt{/groups/\{id\}/}
	\item[Archetype] Readonly Document
	\item[Methods] GET
	\item[JSON Format Response] \hfill
\begin{lstlisting}
{
	"id": <int,id>,
	"uri": "<uri,resource>",
	"name": "<str,name>",
	"parent": "<uri,group>"
}
\end{lstlisting}
\end{description*}
\end{mdframed}

\begin{mdframed}[style=def]
\begin{description*}
	\item[URI Path] \texttt{/groups/}
	\item[Archetype] Readonly Collection
	\item[Methods] GET
	\item[JSON Format] \hfill
\begin{lstlisting}
[
	{
		"id": <int,id>,
		"uri": "<uri,resource>",
		"name": "<str,name>",
		"parent": "<uri,group>"
	}
]
\end{lstlisting}
\end{description*}
\end{mdframed}

\begin{mdframed}[style=def]
\begin{description*}
	\item[URI Path] \texttt{/groups/\{id\}/devices/}
	\item[Archetype] Collection
	\item[Methods] GET, POST
	\item[JSON Format Response] \hfill
\begin{lstlisting}
[<doc,device>, ...]
\end{lstlisting}
\end{description*}
\end{mdframed}

\begin{mdframed}[style=def]
\begin{description*}
	\item[URI Path] \texttt{/groups/\{id\}/enforcements/}
	\item[Archetype] Readonly Collection
	\item[Methods] GET
	\item[JSON Format Response] \hfill
\begin{lstlisting}
[<doc,enfocement>, ...]
\end{lstlisting}
\end{description*}
\end{mdframed}


\pagebreak
\subsubsection{Groups\_members}

Keine Ressourcen für Zwischentabellen (siehe 'devices' und 'enforcements').


\subsubsection{Groups\_products\_defaults}
Keine Ressourcen für Zwischentabellen (siehe 'products').

\pagebreak
\subsubsection{Files}

\begin{mdframed}[style=def]
\begin{description*}
	\item[URI Path] \texttt{/files/\{id\}/}
	\item[Archetype] Document
	\item[Methods] GET, PUT, PATCH
	\item[JSON Format Response] \hfill
\begin{lstlisting}
{
	"id": <int,id>,
	"uri": "<uri,resource>",
	"name": "<str,name>",
	"hashes: <doc,hash-set>,
	"dir": "<uri,directory>"
}
\end{lstlisting}
\end{description*}
\end{mdframed}

\begin{mdframed}[style=def]
\begin{description*}
	\item[URI Path] \texttt{/files/}
	\item[Archetype] Collection
	\item[Methods] GET, POST
	\item[JSON Format Response] \hfill
\begin{lstlisting}
[
	{
		"id": <int,id>,
		"uri": "<uri,resource>",
		"name": "<str,name>"
		"hashes: <doc,hash-set>,
		"dir": "<uri,directory>"
	}
]
\end{lstlisting}
\end{description*}
\end{mdframed}


\pagebreak
\subsubsection{Directories}

\begin{mdframed}[style=def]
\begin{description*}
	\item[URI Path] \texttt{/directories/\{id\}/}
	\item[Archetype] Document
	\item[Methods] GET, PUT, PATCH
	\item[JSON Format Response] \hfill
\begin{lstlisting}
{
	"id": <int,id>,
	"uri": "<uri,resource>",
	"path": "<str,path>"
}
\end{lstlisting}
\end{description*}
\end{mdframed}

\begin{mdframed}[style=def]
\begin{description*}
	\item[URI Path] \texttt{/directories/}
	\item[Archetype] Collection
	\item[Methods] GET, POST
	\item[JSON Format Response] \hfill
\begin{lstlisting}
[
	{
		"id": <int,id>,
		"uri": "<uri,resource>",
		"path": "<str,path>"
	}
]
\end{lstlisting}
\end{description*}
\end{mdframed}

\begin{mdframed}[style=def]
\begin{description*}
	\item[URI Path] \texttt{/directories/\{id\}/files/}
	\item[Archetype] Collection
	\item[Methods] GET, POST
	\item[JSON Format Response] \hfill
\begin{lstlisting}
[<doc,file>, ...]
\end{lstlisting}
\end{description*}
\end{mdframed}

\pagebreak
\subsubsection{SWID Tags}

\begin{mdframed}[style=def]
\begin{description*}
	\item[URI Path] \texttt{/swid-tags/}
	\item[Archetype] Readonly Collection
	\item[Methods] GET
	\item[JSON Format Response] \hfill
\begin{lstlisting}
[<doc,swid-tag>, ...]
\end{lstlisting}
\end{description*}
\end{mdframed}

\begin{mdframed}[style=def]
\begin{description*}
	\item[URI Path] \texttt{/swid-tags/\{id\}/}
	\item[Archetype] Readonly Document
	\item[Methods] GET
	\item[Filter Query] \hfill
	\begin{description*}
		\item[packageName] \texttt{<str,package-name>}
		\item[version] \texttt{<str,version>}
		\item[uniqueId] \texttt{<str,unique-id>}
	\end{description*}
	\item[JSON Format Response] \hfill
\begin{lstlisting}
{
	"id": <int,id>,
	"uri": "<uri,resource>",
	"packageName": "<str,package-name>",
	"version": "<str,version>",
	"uniqueId": "<str,unique-id>",
	"entities": [
		{
		"entity": "<uri,entity>",
		"role":  <int,role>
		}
	],
	"tagXml": "<xml,swid-tag>"
}
\end{lstlisting}
\end{description*}
\end{mdframed}

\begin{mdframed}[style=def]
\begin{description*}
	\item[URI Path] \texttt{/swid-tags/\{id\}/files/}
	\item[Archetype] Readonly Collection
	\item[Methods] GET
	\item[JSON Format Response] \hfill
\begin{lstlisting}
[<doc,file>, ...]
\end{lstlisting}
\end{description*}
\end{mdframed}


\pagebreak
\subsubsection{Entities}

\begin{mdframed}[style=def]
\begin{description*}
	\item[URI Path] \texttt{/swid-entities/\{id\}/}
	\item[Archetype] Readonly Document
	\item[Methods] GET
	\item[JSON Format Response] \hfill
\begin{lstlisting}
{
	"id": <int,id>,
	"uri": "<uri,resource>",
	"name": "<str,name>",
	"regid": "<str,regid>"
}
\end{lstlisting}
\end{description*}
\end{mdframed}

\begin{mdframed}[style=def]
\begin{description*}
	\item[URI Path] \texttt{/swid-entities/}
	\item[Archetype] Readonly Collection
	\item[Methods] GET
	\item[JSON Format Response] \hfill
\begin{lstlisting}
[
	{
		"id": <int,id>,
		"uri": "<uri,resource>",
		"name": "<str,name>",
		"regid": "<str,regid>",
	}
]
\end{lstlisting}
\end{description*}
\end{mdframed}

\begin{mdframed}[style=def]
\begin{description*}
	\item[URI Path] \texttt{/swid-entities/\{id\}/swid-tags/}
	\item[Archetype] Readonly Collection
	\item[Methods] GET
	\item[JSON Format Response] \hfill
\begin{lstlisting}
[<doc,swid-tag>, ...]
\end{lstlisting}
\end{description*}
\end{mdframed}

\pagebreak
\begin{thebibliography}{1}
	\bibitem{masse2011rest}
	Masse, Mark. 
	\emph{REST API design rulebook.} 
	O'Reilly Media, Inc., 2011.
	
	\bibitem{rfc3986}
	Berners-Lee, Tim, Roy Fielding, and Larry Masinter. 
	\emph{RFC 3986: Uniform resource identifier (uri): Generic syntax.}
	The Internet Society (2005).
		
\end{thebibliography}
\end{document}
