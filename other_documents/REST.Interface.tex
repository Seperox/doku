\documentclass[10pt,a4paper]{scrartcl}
\pagestyle{empty}
\usepackage{a4} % alternativ \usepackage{a4wide}
\usepackage[ngerman]{babel} % Neudeutsche Silbentrennung (mehrsprachiges Dokument)
\usepackage{parskip} % Skip indentation of first row
\usepackage{graphicx} % Graphics support
\usepackage{longtable} % Tables across several pages
\usepackage{booktabs}
\usepackage{mdwlist} % lists with less spacing, use itemize*
\usepackage{hyperref} % Hyperlinks
\usepackage{float} % Force float position
\usepackage[automark]{scrpage2} %kopf/fusszeile
\usepackage{listings}
\usepackage[utf8x]{inputenc} % Unicode-Encoding
\usepackage[framemethod=TikZ]{mdframed}
\usepackage{mdframed}
\mdfdefinestyle{def}{%
    linecolor=gray!50!white,
    outerlinewidth=0.5pt,
    roundcorner=3pt,
    skipabove=\topskip,
    skipbelow=\topskip
    innertopmargin=\baselineskip,
    innerbottommargin=\baselineskip,
    innerrightmargin=10pt,
    innerleftmargin=10pt
}

\lstset{
    basicstyle=\linespread{.94}\ttfamily,
    tabsize=2,
}

\linespread{1.3}

\author{Danilo Bargen, Christian Fässler, Jonas Furrer} 
\title{REST API\\strongTNC BA}

\pagestyle{scrheadings}
\ihead{REST API} %linke Kopfzeile
\ohead{strongTNC BA} %rechte Kopfzeile

\let\textquotedbl="

\begin{document}

\begin{titlepage}
	\maketitle
	\vspace{120mm}
	\thispagestyle{empty} % Don't start page numbers on this page
\end{titlepage}

\newpage
	\tableofcontents
\newpage

\section{Einleitung}

Die Definition der Ressourcen orientiert sich an den Regeln des Buches
\textit{REST API Design Rulebook} \cite{masse2011rest} aus dem O'Reilly Verlag.

\subsubsection*{URI Definition}

Bei der Bezeichnung der URIs\footnote{Uniform Resource Identifier} wurde folgende Terminologie gemäss RFC 3986 verwendet:

\texttt{URI = scheme \textquotedbl ://\textquotedbl{} authority \textquotedbl /\textquotedbl{}
path [ \textquotedbl ?\textquotedbl{} query ] [ \textquotedbl \#\textquotedbl{} fragment ]}

\subsubsection*{Ressource-Archetypen}

Nachfolgend die Ressource-Archetypen gemäss \cite{masse2011rest}. Die Erklärungstexte wurden direkt dem besagten
Buch entnommen.

\begin{description}
	\item[Document] A document resource is a singular concept that is akin to an object instance or database
		record. A document’s state representation typically includes both fields with values and
		links to other related resources.
	\item[Collection] A collection resource is a server-managed directory of resources. Clients may propose
		new resources to be added to a collection. However, it is up to the collection to choose
		to create a new resource, or not.
	\item[Store] A store is a client-managed resource repository. A store resource lets an API client put
		resources in, get them back out, and decide when to delete them. On their own, stores
		do not create new resources; therefore a store never generates new URIs. Instead, each
		stored resource has a URI that was chosen by a client when it was initially put into the
		store.
	\item[Controller] A controller resource models a procedural concept. Controller resources are like
		executable functions, with parameters and return values; inputs and outputs.
		Like a traditional web application’s use of HTML forms, a REST API relies on controller
		resources to perform application-specific actions that cannot be logically mapped to
		one of the standard methods (create, retrieve, update, and delete, also known as
		CRUD).
\end{description}

\pagebreak
\section{REST Ressourcen}

\subsection{Products}

\begin{description*}
    \item[SQL] \texttt{SELECT/INSERT}
    \item[Felder] id, *name
\end{description*}


\begin{mdframed}[style=def]
\begin{description}
	\item[URI Path] \texttt{/products/\{id\}}
	\item[Archetype] Document
	\item[Methods] GET, PUT
	\item[Response Statuscodes] \hfill
		\begin{description*}
            \item[201 Created] Product wurde erfolgreich erstellt
            \item[204 No Content] Product wurde erfolgreich geändert
            \item[400 Bad Request] Response Entity fehlt oder ist unvollständig
            \item[404 Not Found] Zu änderndes Product nicht gefunden
            \item[409 Conflict] Zu erstellendes Product existiert bereits
		\end{description*}
	\item[JSON Format] \hfill
\begin{lstlisting}
{
	id : $id$ ,
	name : "$productname$"
}
\end{lstlisting}
\end{description}
\end{mdframed}

\begin{mdframed}[style=def]
\begin{description}
	\item[URI Path] \texttt{/products}
	\item[Archetype] Collection
	\item[Methods] GET, POST
	\item[JSON Format] \hfill
\begin{lstlisting}
[
	{
		id : $id$ ,
		name : "$productname$",
		uri: "$resource-uri$"
	}
]
\end{lstlisting}
\end{description}
\end{mdframed}


\pagebreak
\subsection{Packages}

\begin{description*}
    \item[SQL] \texttt{SELECT/INSERT}
    \item[Felder] id, *name
\end{description*}


\begin{mdframed}[style=def]
\begin{description}
	\item[URI Path] \texttt{/packages/\{id\}}
	\item[Archetype] Document
	\item[Methods] GET, PUT
	\item[Response Statuscodes] \hfill
		\begin{description*}
            \item[201 Created] Package wurde erfolgreich erstellt
            \item[204 No Content] Package wurde erfolgreich geändert
            \item[400 Bad Request] Response Entity fehlt oder ist unvollständig
            \item[404 Not Found] Zu änderndes Package nicht gefunden
            \item[409 Conflict] Zu erstellendes Package existiert bereits
		\end{description*}
	\item[JSON Format] \hfill
\begin{lstlisting}
{
	id : $id$ ,
	name : "$packagename$"
}
\end{lstlisting}
\end{description}
\end{mdframed}

\begin{mdframed}[style=def]
\begin{description}
	\item[URI Path] \texttt{/packages}
	\item[Archetype] Collection
	\item[Methods] GET, POST
	\item[JSON Format] \hfill
\begin{lstlisting}
[
	{
		id : $id$ ,
		name : "$packagename$",
		uri: "$resource-uri$"
	}
]
\end{lstlisting}
\end{description}
\end{mdframed}


\pagebreak
\subsection{Versions}

\begin{description*}
    \item[SQL] \texttt{SELECT/INSERT/DELETE}
    \item[Felder] *id, *product, *package, release, security, blacklist, time
\end{description*}


\begin{mdframed}[style=def]
\begin{description}
	\item[URI Path] \texttt{/versions/\{id\}}
	\item[Archetype] Document
	\item[Methods] GET, PUT, DELETE
	\item[Response Statuscodes] \hfill
		\begin{description*}
            \item[201 Created] Version wurde erfolgreich erstellt
            \item[204 No Content] Version wurde erfolgreich geändert
            \item[400 Bad Request] Response Entity fehlt oder ist unvollständig
            \item[404 Not Found] Zu änderndes Version nicht gefunden
            \item[409 Conflict] Zu erstellendes Version existiert bereits
		\end{description*}
	\item[JSON Format] \hfill
\begin{lstlisting}
{
	id : $id$ ,
	package : "$package-uri$",
	product : "$product-uri$",
	release : "$release$",
	securtiy : $security$,
	blacklist : $blacklist$,
	time : $time$
}
\end{lstlisting}
\end{description}
\end{mdframed}

\begin{mdframed}[style=def]
\begin{description}
	\item[URI Path] \texttt{/versions}
	\item[Query/Filter]	product=\$product\$, package=\$package\$ 
	\item[Archetype] Collection
	\item[Methods] GET, POST
	\item[JSON Format] \hfill
\begin{lstlisting}
[
	{
	    id : $id$ ,
	    package : "$package-uri$",
	    product : "$product-uri$",
    	release : "$release$",
    	securtiy : $security$,
    	blacklist : $blacklist$,
    	time : $time$,
    	uri: "$resource-uri$"
    }
]
\end{lstlisting}
\end{description}
\end{mdframed}


\pagebreak
\subsection{Identities}

\begin{description*}
    \item[SQL] \texttt{SELECT/INSERT}
    \item[Felder] id, *type, *value
\end{description*}


\begin{mdframed}[style=def]
\begin{description}
	\item[URI Path] \texttt{/identities/\{id\}}
	\item[Archetype] Document
	\item[Methods] GET, PUT
	\item[Response Statuscodes] \hfill
		\begin{description*}
            \item[201 Created] Identitiy wurde erfolgreich erstellt
            \item[204 No Content] Identitiy wurde erfolgreich geändert
            \item[400 Bad Request] Response Entity fehlt oder ist unvollständig
            \item[404 Not Found] Zu änderndes Identitiy nicht gefunden
            \item[409 Conflict] Zu erstellendes Identitiy existiert bereits
		\end{description*}
	\item[JSON Format] \hfill
\begin{lstlisting}
{
	id : $id$ ,
	type : $type$,
	value : "$value$"
}
\end{lstlisting}
\end{description}
\end{mdframed}

\begin{mdframed}[style=def]
\begin{description}
	\item[URI Path] \texttt{/identities}
	\item[Query/Filter]	type=\$type\$, value=\$value\$ 
	\item[Archetype] Collection
	\item[Methods] GET, POST
	\item[JSON Format] \hfill
\begin{lstlisting}
[
	{
	    id : $id$ ,
	    type : $type$,
	    value : "$value$",
	    uri: "$resource-uri$"
    }
]
\end{lstlisting}
\end{description}
\end{mdframed}


\pagebreak
\subsection{Devices}

\begin{description*}
    \item[SQL] \texttt{SELECT/INSERT/JOIN}
    \item[Felder] *id, *value, *product, trusted, created, +desription
\end{description*}


\begin{mdframed}[style=def]
\begin{description}
	\item[URI Path] \texttt{/devices/\{id\}}
	\item[Archetype] Document
	\item[Methods] GET, PUT
	\item[Response Statuscodes] \hfill
		\begin{description*}
            \item[201 Created] Device wurde erfolgreich erstellt
            \item[204 No Content] Device wurde erfolgreich geändert
            \item[400 Bad Request] Response Entity fehlt oder ist unvollständig
            \item[404 Not Found] Zu änderndes Device nicht gefunden
            \item[409 Conflict] Zu erstellendes Device existiert bereits
		\end{description*}
	\item[JSON Format] \hfill
\begin{lstlisting}
{
	id : $id$ ,
	description : "$description$",
	value : "$value$",
	product : "$product-uri$",
	created : $created$
}
\end{lstlisting}
\end{description}
\end{mdframed}

\begin{mdframed}[style=def]
\begin{description}
	\item[URI Path] \texttt{/devices}
	\item[Query/Filter]	value=\$value\$, product=\$product\$ 
	\item[Archetype] Collection
	\item[Methods] GET, POST
	\item[JSON Format] \hfill
\begin{lstlisting}
[
	{
	    id : $id$ ,
	    description : "$description$",
	    value : "$value$",
    	product : "$product-uri$",
    	created : $created$,
    	uri: "$resource-uri$"
    }
]
\end{lstlisting}
\end{description}
\end{mdframed}


\pagebreak
\subsection{Sessions}

\begin{description*}
    \item[SQL] \texttt{SELECT/INSERT/JOIN}
    \item[Felder] *id, *device, *time, connection, product, identity, +rec
\end{description*}


\begin{mdframed}[style=def]
\begin{description}
	\item[URI Path] \texttt{/sessions/\{id\}}
	\item[Archetype] Document
	\item[Methods] GET, PUT
	\item[Response Statuscodes] \hfill
		\begin{description*}
            \item[201 Created] Session wurde erfolgreich erstellt
            \item[204 No Content] Session wurde erfolgreich geändert
            \item[400 Bad Request] Response Entity fehlt oder ist unvollständig
            \item[404 Not Found] Zu änderndes Session nicht gefunden
            \item[409 Conflict] Zu erstellendes Session existiert bereits
		\end{description*}
	\item[JSON Format] \hfill
\begin{lstlisting}
{
	id : $id$ ,
	time : $time$,
	identity : "$identity-uri$",
	device : "$device-uri$",
	product : $product-uri$,
	rec : "$rec$"
}
\end{lstlisting}
\end{description}
\end{mdframed}

\begin{mdframed}[style=def]
\begin{description}
	\item[URI Path] \texttt{/sessions}
	\item[Query/Filter]	time=\$time\$, device=\$device\$ 
	\item[Archetype] Collection
	\item[Methods] GET, POST
	\item[JSON Format] \hfill
\begin{lstlisting}
[
	{
	    id : $id$ ,
    	time : $time$,
    	identity : "$identity-uri$",
    	device : "$device-uri$",
    	product : $product-uri$,
    	rec : "$rec$",
    	uri: "$resource-uri$"
    }
]
\end{lstlisting}
\end{description}
\end{mdframed}


\pagebreak
\subsection{Workitems}

\begin{description*}
    \item[SQL] \texttt{SELECT/INSERT/DELETE/JOIN}
    \item[Felder] *id, *session, *result, enforcement, type, arg\_str, arg\_int, rec\_fail, rec\_noresult, rec\_final
\end{description*}


\begin{mdframed}[style=def]
\begin{description}
	\item[URI Path] \texttt{/workitems/\{id\}}
	\item[Archetype] Document
	\item[Methods] GET, PUT, DELETE
	\item[Response Statuscodes] \hfill
		\begin{description*}
            \item[201 Created] Workitem wurde erfolgreich erstellt
            \item[204 No Content] Workitem wurde erfolgreich geändert
            \item[400 Bad Request] Response Entity fehlt oder ist unvollständig
            \item[404 Not Found] Zu änderndes Workitem nicht gefunden
            \item[409 Conflict] Zu erstellendes Workitem existiert bereits
		\end{description*}
	\item[JSON Format] \hfill
\begin{lstlisting}
{
	id : $id$ ,
	session : "$session-uri$",
	enforcement : "$enforcement-uri$",
	type : $type$,
	arg-str : "$arg_str$",
	arg-int : $arg_int$,
	rec-fail : $rec_fail$,
	rec-noresult : $rec_noresult$", 
	rec-final : $rec_final$,
	result : "$result$"
}
\end{lstlisting}
\end{description}
\end{mdframed}

\begin{mdframed}[style=def]
\begin{description}
	\item[URI Path] \texttt{/workitems}
	\item[Query/Filter]	session=\$session\$, result=\$result\$ 
	\item[Archetype] Collection
	\item[Methods] GET, POST
	\item[JSON Format] \hfill
\begin{lstlisting}
[
	{
	    id : $id$ ,
    	session : "$session-uri$",
    	enforcement : "$enforcement-uri$",
    	type : $type$,
    	arg-str : "$arg_str$",
    	arg-int : $arg_int$,
    	rec-fail : $rec_fail$,
    	rec-noresult : $rec_noresult$", 
    	rec-final : $rec_final$,
    	result : "$result$",
    	uri: "$resource-uri$"
    }
]
\end{lstlisting}
\end{description}
\end{mdframed}


\pagebreak
\subsection{Enforcements}

\begin{description*}
    \item[SQL] \texttt{SELECT/JOIN}
    \item[Felder] *id, *group\_id, policy, rec\_fail, rec\_noresult, max\_age
\end{description*}


\begin{mdframed}[style=def]
\begin{description}
	\item[URI Path] \texttt{/enforcements/\{id\}}
	\item[Archetype] Readonly Document
	\item[Methods] GET
	\item[JSON Format] \hfill
\begin{lstlisting}
{
	id : $id$ ,
	policy : "$policy-uri$",
	group-id : "$group-uri$",
	rec-fail : $rec_fail$,
	rec-noresult : $rec_noresult$", 
	max-age : $max_age$
}
\end{lstlisting}
\end{description}
\end{mdframed}

\begin{mdframed}[style=def]
\begin{description}
	\item[URI Path] \texttt{/enforcements}
	\item[Query/Filter]	group-id=\$group\_id\$ 
	\item[Archetype] Collection
	\item[Methods] GET
	\item[JSON Format] \hfill
\begin{lstlisting}
[
    {
    	id : $id$ ,
    	policy : "$policy-uri$",
    	group-id : "$group-uri$",
    	rec-fail : $rec_fail$,
    	rec-noresult : $rec_noresult$", 
    	max-age : $max_age$,
    	uri : "$resource-uri$"
    }
]
\end{lstlisting}
\end{description}
\end{mdframed}


\pagebreak
\subsection{Policies}

\begin{description*}
    \item[SQL] \texttt{SELECT/JOIN}
    \item[Felder] *id, type, argument, rec\_fail, rec\_noresult, file, dir, +name
\end{description*}


\begin{mdframed}[style=def]
\begin{description}
	\item[URI Path] \texttt{/policies/\{id\}}
	\item[Archetype] Readonly Document
	\item[Methods] GET
	\item[JSON Format] \hfill
\begin{lstlisting}
{
	id : $id$ ,
	type : $type$,
	name : "$name$",
	argument : "$argument"$,
	rec-fail : $rec_fail$,
	rec-noresult : $rec_noresult$", 
	file : "$file-uri$",
	dir : "$dir-uri$"
}
\end{lstlisting}
\end{description}
\end{mdframed}

\begin{mdframed}[style=def]
\begin{description}
	\item[URI Path] \texttt{/policies}
	\item[Archetype] Collection
	\item[Methods] GET
	\item[JSON Format] \hfill
\begin{lstlisting}
[
    {
    	id : $id$ ,
    	type : $type$,
    	name : "$name$",
    	argument : "$argument"$,
    	rec-fail : $rec_fail$,
    	rec-noresult : $rec_noresult$", 
    	file : "$file-uri$",
    	dir : "$dir-uri$",
    	uri : "$resource-uri$"
    }
]
\end{lstlisting}
\end{description}
\end{mdframed}


\pagebreak
\subsection{Results}

\begin{description*}
    \item[SQL] \texttt{SELECT/INSERT/JOIN}
    \item[Felder] *policy, session, rec, result, +id
\end{description*}


\begin{mdframed}[style=def]
\begin{description}
	\item[URI Path] \texttt{/results/\{id\}}
	\item[Archetype] Document
	\item[Methods] GET, PUT
	\item[Response Statuscodes] \hfill
		\begin{description*}
            \item[201 Created] Result wurde erfolgreich erstellt
            \item[204 No Content] Result wurde erfolgreich geändert
            \item[400 Bad Request] Response Entity fehlt oder ist unvollständig
            \item[404 Not Found] Zu änderndes Result nicht gefunden
            \item[409 Conflict] Zu erstellendes Result existiert bereits
		\end{description*}
	\item[JSON Format] \hfill
\begin{lstlisting}
{
	id : $id$ ,
	session : "$session-uri$",
	policy : "$policy-uri$",
	rec : $rec$,
	result : "$result$"
}
\end{lstlisting}
\end{description}
\end{mdframed}

\begin{mdframed}[style=def]
\begin{description}
	\item[URI Path] \texttt{/results}
	\item[Query/Filter]	policy=\$policy\$
	\item[Archetype] Collection
	\item[Methods] GET, POST
	\item[JSON Format] \hfill
\begin{lstlisting}
[
	{
	    id : $id$ ,
    	session : "$session-uri$",
    	policy : "$policy-uri$",
    	rec : $rec$,
    	result : "$result$",
      	uri : "$resource-uri$"
    }
]
\end{lstlisting}
\end{description}
\end{mdframed}


\pagebreak
\subsection{Groups}

\begin{description*}
    \item[SQL] \texttt{SELECT}
    \item[Felder] *id, parent, +name
\end{description*}


\begin{mdframed}[style=def]
\begin{description}
	\item[URI Path] \texttt{/groups/\{id\}}
	\item[Archetype] Readonly Document
	\item[Methods] GET
	\item[JSON Format] \hfill
\begin{lstlisting}
{
	id : $id$ ,
	name : "$name$",
	parent : "$group-uri$",
}
\end{lstlisting}
\end{description}
\end{mdframed}

\begin{mdframed}[style=def]
\begin{description}
	\item[URI Path] \texttt{/groups}
	\item[Archetype] Collection
	\item[Methods] GET
	\item[JSON Format] \hfill
\begin{lstlisting}
[
    {
	    id : $id$ ,
	    name : "$name$",
	    parent : "$group-uri$",
	    uri : "$resource-uri$"
    }
]
\end{lstlisting}
\end{description}
\end{mdframed}


\pagebreak
\subsection{Groups\_members}

\begin{description*}
    \item[SQL] \texttt{SELECT/INSERT}
    \item[Felder] *device\_id, group\_id, +id
\end{description*}


\begin{mdframed}[style=def]
\begin{description}
	\item[URI Path] \texttt{/groups-members/\{id\}}
	\item[Archetype] Document
	\item[Methods] GET, PUT
	\item[Response Statuscodes] \hfill
		\begin{description*}
            \item[201 Created] Group Member wurde erfolgreich erstellt
            \item[204 No Content] Group Member wurde erfolgreich geändert
            \item[400 Bad Request] Response Entity fehlt oder ist unvollständig
            \item[404 Not Found] Zu änderndes Group Member nicht gefunden
            \item[409 Conflict] Zu erstellendes Group Member existiert bereits
		\end{description*}
	\item[JSON Format] \hfill
\begin{lstlisting}
{
	id : $id$ ,
	group : "$group-uri$",
	device : "$device-uri$"
}
\end{lstlisting}
\end{description}
\end{mdframed}

\begin{mdframed}[style=def]
\begin{description}
	\item[URI Path] \texttt{/groups-members}
	\item[Archetype] Collection
	\item[Methods] GET, POST
	\item[JSON Format] \hfill
\begin{lstlisting}
[
    {
    	id : $id$ ,
    	group : "$group-uri$",
    	device : "$device-uri$",
	    uri : "$resource-uri$"
    }
]
\end{lstlisting}
\end{description}
\end{mdframed}


\pagebreak
\subsection{Groups\_products\_defaults}

\begin{description*}
    \item[SQL] \texttt{SELECT}
    \item[Felder] *product\_id, group\_id, +id
\end{description*}


\begin{mdframed}[style=def]
\begin{description}
	\item[URI Path] \texttt{/groups-product-defaults/\{id\}}
	\item[Archetype] Readonly Document
	\item[Methods] GET
	\item[JSON Format] \hfill
\begin{lstlisting}
{
	id : $id$ ,
	group : "$group-uri$",
	product : "$product-uri$"
}
\end{lstlisting}
\end{description}
\end{mdframed}

\begin{mdframed}[style=def]
\begin{description}
	\item[URI Path] \texttt{/groups-members}
	\item[Archetype] Collection
	\item[Methods] GET, POST
	\item[JSON Format] \hfill
\begin{lstlisting}
[
    {
    	id : $id$ ,
    	group : "$group-uri$",
    	device : "$device-uri$",
	    uri : "$resource-uri$"
    }
]
\end{lstlisting}
\end{description}
\end{mdframed}


\pagebreak
\subsection{Files}

\begin{mdframed}[style=def]
\begin{description}
	\item[URI Path] \texttt{/files/\{id\}}
	\item[Archetype] Document
	\item[Methods] GET, PUT
	\item[Response Statuscodes] \hfill
		\begin{description*}
            \item[201 Created] File wurde erfolgreich erstellt
            \item[204 No Content] File wurde erfolgreich geändert
            \item[400 Bad Request] Response Entity fehlt oder ist unvollständig
            \item[404 Not Found] Zu änderndes File nicht gefunden
            \item[409 Conflict] Zu erstellendes File existiert bereits
		\end{description*}
	\item[JSON Format] \hfill
\begin{lstlisting}
{
	id : $id$ ,
	name : "$name$",
	dir : "$directory-uri$"
}
\end{lstlisting}
\end{description}
\end{mdframed}

\begin{mdframed}[style=def]
\begin{description}
	\item[URI Path] \texttt{/files}
	\item[Archetype] Collection
	\item[Methods] GET, POST
	\item[JSON Format] \hfill
\begin{lstlisting}
[
    {
    	id : $id$ ,
    	name : "$name$",
    	dir : "$directory-uri$",
    	uri : "$resource-uri$"
    }
]
\end{lstlisting}
\end{description}
\end{mdframed}


\pagebreak
\subsection{Directories}

\begin{mdframed}[style=def]
\begin{description}
	\item[URI Path] \texttt{/directories/\{id\}}
	\item[Archetype] Document
	\item[Methods] GET, PUT
	\item[Response Statuscodes] \hfill
		\begin{description*}
            \item[201 Created] Directory wurde erfolgreich erstellt
            \item[204 No Content] Directory wurde erfolgreich geändert
            \item[400 Bad Request] Response Entity fehlt oder ist unvollständig
            \item[404 Not Found] Zu änderndes Directory nicht gefunden
            \item[409 Conflict] Zu erstellendes Directory existiert bereits
		\end{description*}
	\item[JSON Format] \hfill
\begin{lstlisting}
{
	id : $id$ ,
	path : "$path$",
}
\end{lstlisting}
\end{description}
\end{mdframed}

\begin{mdframed}[style=def]
\begin{description}
	\item[URI Path] \texttt{/directories}
	\item[Archetype] Collection
	\item[Methods] GET, POST
	\item[JSON Format] \hfill
\begin{lstlisting}
[
    {
    	id : $id$ ,
    	path : "$path$",
    	uri : "$resource-uri$"
    }
]
\end{lstlisting}
\end{description}
\end{mdframed}


\pagebreak
\subsection{SWID Tags}

\begin{mdframed}[style=def]
\begin{description}
	\item[URI Path] \texttt{/swid-tags/{id}}
	\item[Archetype] Document
	\item[Methods] GET, PUT
	\item[Response Statuscodes] \hfill
		\begin{description*}
            \item[201 Created] SWID Tag wurde erfolgreich erstellt
            \item[204 No Content] SWID Tag  wurde erfolgreich geändert
            \item[400 Bad Request] Response Entity fehlt oder ist unvollständig
            \item[404 Not Found] Zu änderndes SWID Tag  nicht gefunden
            \item[409 Conflict] Zu erstellendes SWID Tag  existiert bereits
		\end{description*}
	\item[JSON Format] \hfill
\begin{lstlisting}
{
    id : $id$,
    package-name: "$package-name$",
    version : "$version$",
    unique-id : "$unique_id$",
    entity : {
        tag-creator : "$entity-uri$",
        publisher :  $entity-uri$",
        licensor : $entity-uri$"
    }
}
\end{lstlisting}
\end{description}
\end{mdframed}

\begin{mdframed}[style=def]
\begin{description}
	\item[URI Path] \texttt{/swid-tags}
	\item[Archetype] Collection
	\item[Methods] GET, POST
	\item[JSON Format] \hfill
\begin{lstlisting}
[
    {
        id : $id$,
        package-name: "$package-name$",
        version : "$version$",
        unique-id : "$unique_id$",
        entity : {
            tag-creator : "$entity-uri$",
            publisher :  $entity-uri$",
            licensor : $entity-uri$"
        },
        uri : "$resource-uri$"
    }
]
\end{lstlisting}
\end{description}
\end{mdframed}


\pagebreak
\subsection{Entities}


\pagebreak
\subsection{SWID Tags: Messung}

Die strongSwan Komponente sendet eine Liste von Software-IDs an die strongTNC 
Schnittstelle. Die Software-IDs wurden zuvor auf dem Client gemessen und widerspiegeln die momentan
installierten Software Pakete.\\
Wenn bereits SWID Tags für alle übertragenen Software-IDs bestehen, können diese direkt eingetragen werden.
\begin{mdframed}[style=def]
\begin{description}
	\item[URI Path] \texttt{/swid/measurement}
	\item[Archetype] Controller
	\item[Methods] POST
	\item[Response Statuscodes] \hfill
		\begin{description*}
			\item[200 OK] SWID Tags der übermittelten Software-IDs sind eingetragen und wurden für die übermittelte Session eingetragen
			\item[412 Precondition Failed] Es existieren nicht alle SWID Tags für die übertragenen Software-IDs. Als Payload werden die fehlenden Software-IDs übertragen.
		\end{description*}
	\item[JSON Format] \hfill
\begin{lstlisting}
[
    "{software-id}"
]
\end{lstlisting}
\end{description}
\end{mdframed}
\end{document}
