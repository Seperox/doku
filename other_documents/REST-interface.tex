\documentclass[10pt,a4paper]{scrartcl}
\pagestyle{empty}
\usepackage{a4wide}
\usepackage[english]{babel}
\usepackage{parskip} % Skip indentation of first row
\usepackage{graphicx} % Graphics support
\usepackage{longtable} % Tables across several pages
\usepackage{booktabs}
\usepackage{mdwlist} % lists with less spacing, use itemize*
\usepackage{hyperref} % Hyperlinks
\usepackage{float} % Force float position
\usepackage[automark]{scrpage2} %kopf/fusszeile
\usepackage{listings}
\usepackage[utf8x]{inputenc} % Unicode-Encoding
\usepackage[framemethod=TikZ]{mdframed}
\usepackage{mdframed}
\mdfdefinestyle{def}{%
	linecolor=gray!50!white,
	outerlinewidth=0.5pt,
	roundcorner=3pt,
	skipabove=\topskip,
	skipbelow=\topskip
	innertopmargin=\baselineskip,
	innerbottommargin=\baselineskip,
	innerrightmargin=10pt,
	innerleftmargin=10pt,
	nobreak=true
}

\lstset{
	basicstyle=\linespread{.94}\ttfamily,
	tabsize=2,
}

\linespread{1.2}

\author{Danilo Bargen, Christian Fässler, Jonas Furrer} 
\title{strongTNC REST API Draft 1\\ \small{Part of the strongTNC Bachelor's Thesis} }
\date{Spring 2014}

\pagestyle{scrheadings}
\ihead{REST API Draft 1} %linke Kopfzeile
\ohead{strongTNC BA} %rechte Kopfzeile

\let\textquotedbl="

\begin{document}

\begin{titlepage}
	\maketitle
	\vspace{120mm}
	\thispagestyle{empty} % Don't start page numbers on this page
\end{titlepage}

\newpage
	\tableofcontents
\newpage

\section{Introduction}

The REST resource design is based on the recommendations of the \textit{REST API
Design Rulebook}\cite{masse2011rest} from O'Reilly.

\subsubsection*{URI Definition}

URIs\footnote{Uniform Resource Identifier} are defined as follows, according to RFC 3986\cite{rfc3986}:

\texttt{URI = scheme \textquotedbl ://\textquotedbl{} authority \textquotedbl /\textquotedbl{}
path [ \textquotedbl ?\textquotedbl{} query ] [ \textquotedbl \#\textquotedbl{} fragment ]}

\subsubsection*{Resource-Archetypes}

We used the resource archetype terminology from Masse 2011\cite{masse2011rest}.
The explanation texts are taken directly from said book.

\begin{description}
	\item[Document] A document resource is a singular concept that is akin to an
	object instance or database record. A document’s state representation typically
	includes both fields with values and links to other related resources.
	
	\item[Collection] A collection resource is a server-managed directory of
	resources. Clients may propose new resources to be added to a
	collection. However, it is up to the collection to choose to create a new
	resource, or not.
	
	\item[Store] A store is a client-managed resource repository. A store resource
	lets an API client put resources in, get them back out, and decide when to
	delete them. On their own, stores do not create new resources; therefore a
	store never generates new URIs. Instead, each stored resource has a URI that
	was chosen by a client when it was initially put into the store.
	
	\item[Controller] A controller resource models a procedural concept. Controller
	resources are like executable functions, with parameters and return values;
	inputs and outputs. Like a traditional web application’s use of HTML forms, a
	REST API relies on controller resources to perform application-specific actions
	that cannot be logically mapped to one of the standard methods (create,
	retrieve, update, and delete, also known as CRUD).
\end{description}


\pagebreak


\section{Archetype Representation}

The JSON representation in this API documentation depends on the resource
archetype.

\begin{description}
	\item[Document] A document returns a JSON object, which contains all relevant
		fields.\\
		If a document contains references to other resources, then this reference is
		returned as an URI. Additionally, a \texttt{depth} query parameter can be
		specified to embed nested resources directly in the response.\\
		Example: \hfill
\begin{lstlisting}
{
	"field1": "<str,value-1>",
	"field2": <int,value-2>,
	"field3": <bool,value-3>,
	"subObject": "<uri,sub-object>"
	"subCollection": [
		{
			"uri": "<uri,sub-object>"
		},
		{
			"uri": "<uri,sub-object>"
		}
	]
}
\end{lstlisting}
Example with \texttt{depth=1}:
\begin{lstlisting}
{
	"field1": "<str,value-1>",
	"field2": <int,value-2>,
	"field3": <bool,value-3>,
	"subObject": {
		"field1": <int,value-1>,
		...
		"uri": "<uri,sub-object>"
	}
	"subCollection": [
		{
			"field": <int,field>,
			...
			"uri": "<uri,sub-object>"
		},
		{
			"field": <int,field>,
			...
			"uri": "<uri,sub-object>"
		}
	]
}
\end{lstlisting}

	\newpage
	\item[Collection] A collection returns a (possibly filtered) list of all
		contained documents as JSON-objects. The documents are annotated with an
		additional field called \texttt{uri}, which contains the URI to the document
		resource.\\
		Collections too allow the usage of a \texttt{depth} parameter to embed
		nested objects.\\ 
		Example: \hfill
\begin{lstlisting}
{
	{
		"field1": "<str,value-1>",
		"subObject": "<uri,sub-object>",
		"uri": "<uri,objek>"
	},
	{
		"field1": "'<str,value-1>",
		"subObject": "<uri,sub-object>",
		"uri": "<uri,resource>"
	},
}
\end{lstlisting}

Example with \texttt{depth=1}:
\begin{lstlisting}
{
	{
		"field1": "<str,value-1>",
		"subObject": {
			"field": "<str,value>",
		},
		"uri": "<uri,resource>"
	},
	{
		"field1": "'<str,value-1>",
		"subObject": {
			"field": "<str,value>",
		},
		"uri": "<uri,resource>"
	},
}
\end{lstlisting}

	\item[Controller] The output of a controller varies, depending on the
		implementation.
\end{description}


\pagebreak


\section{HTTP Status Codes}

The result of a request is communicated through HTTP status codes. In some
cases, additional information is returned in the response body.

The following status codes can be expected:

\begin{description*}
	\item[200 OK] The request has succeeded.
	\item[201 Created] The request has been fulfilled and resulted in a new
		resource being created.
	\item[204 No Content] The server has fulfilled the request but does not need
		to return an entity-body.
	\item[400 Bad Request] Generic client side error.
	\item[404 Not Found] The server has not found anything matching the
		Request-URI.
	\item[405 Method Not Allowed] The method specified in the Request-Line is not
		allowed for the resource identified by the Request-URI.
	\item[409 Conflict] The entity to be created already exists.
	\item[412 Precondition Failed] Additional steps are necessary to successfully
		process the sent request.
	\item[500 Internal Server Error] Generic server side error.
\end{description*}

If a resource uses additional error codes, then it is documented in the resource
documentation.

\pagebreak


\section{Further Remarks}

\subsection{Notation}

For the request- and response-format of a resource, pseudo-JSON notation is
used. The values are denoted as a tuple in angle brackets. The first part of the
tuple denotes the type, while the second is a description of the value.

The following types are used:

\begin{description*}
	\item[int] Integer
	\item[num] Decimal
	\item[str] String
	\item[bool] Boolean
	\item[xml] An XML document
	\item[hex] A HEX string
	\item[uri] The fully qualified URI of a resource
	\item[doc] The resource document
\end{description*}

\subsection{Default Behavior}

The following items describe the default behavior of resources:

\begin{description*}
	\item[Depth Query-Parameter] Each resource that returns an URI to another
		resource, supports a \texttt{depth} query parameter. This parameter embeds
		the nested resources into the returned document up to the depth specified by
		the query parameter. If no parameter is specified, the depth \texttt{0} is
		used.
	
	\item[Filter Query-Parameter] Filter query parameters are -- if available --
		always optional. They can be used to filter the returned data.\\
	All Collections support -- unless specified otherwise -- generic filters on
	all fields except foreign keys. The parameter looks like this:
	\texttt{fieldName=query}. Filters can also be used on document resources -- if
	the document field values don't match, then a HTTP 404 status code is
	returned.

	\item[Fields Query-Parameter] The \texttt{fields} query parameter can be used
		to limit the returned fields of a resource. This is similar to a
		\texttt{SELECT field1, field2, fieldn} SQL statement.
	
\end{description*}
 
 
\pagebreak


\section{Data Definitions}

\subsection{Policy Arguments}

The following workitem- and policy-types are currently available:
\begin{description*}
	\item[\texttt{00: RESVD}] Deny
	\item[\texttt{01: PCKGS}] Installed Packages
	\item[\texttt{02: UNSRC}] Unknown Source
	\item[\texttt{03: FWDEN}] Forwarding Enabled
	\item[\texttt{04: PWDEN}] Default Password Enabled
	\item[\texttt{05: FREFM}] File Reference Measurement
	\item[\texttt{06: FMEAS}] File Measurement 
	\item[\texttt{07: FMETA}] File Metadata
	\item[\texttt{08: DREFM}] Directory Reference Measurement
	\item[\texttt{09: DMEAS}] Directory Measurement
	\item[\texttt{10: DMETA}] Directory Metadata
	\item[\texttt{11: TCPOP}] Open TCP Listening Ports
	\item[\texttt{12: TCPBL}] Blocked TCP Listening Ports
	\item[\texttt{13: UDPOP}] Open UDP Listening Ports
	\item[\texttt{14: UDPBL}] Blocked UDP Listening Ports
	\item[\texttt{15: SWIDT}] SWID Tag Inventory
	\item[\texttt{16: TPMRA}] TPM Remote Attestation
\end{description*}

Some of the types have different parameters. They can be grouped as follows:

\begin{description*}
	\item[No arguments] 00, 01, 02, 03, 04
	\item[File path] 05, 06, 07
	\item[Directory path] 08, 09, 10
	\item[Port list] 11, 12, 13, 14
	\item[SWID request flags] 15
	\item[TPM attestation flags] 16
\end{description*}

Depending on the type, the following arguments need to be submitted:

\begin{description*}
	\item[No arguments] \hfill
\begin{lstlisting}
{}
\end{lstlisting}

	\item[File path] \hfill
\begin{lstlisting}
{
	"file": "<str,file-path>"
}
\end{lstlisting}   

	\item[Directory path] \hfill
\begin{lstlisting}
{
	"directory": "<str,directory-path>"
}
\end{lstlisting} 

	\item[Port list] \hfill
\begin{lstlisting}
{
	"portList": [
		"<str,port or port-range>"
	]
}
\end{lstlisting} 

	\item[SWID request flags] \hfill
\begin{lstlisting}
{
	"swidFlags": [
		"<str,swid-flag>"
	]
}
\end{lstlisting} 

	\item[TPM attestation flags] \hfill
\begin{lstlisting}
{
	"tpmFlags": [
		"<str,tpm-flag>"
	]
}
\end{lstlisting} 
\end{description*}


\subsection{Recommendation Types}

The following recommendation types and -actions are available:

\begin{description*}
	\item[\texttt{0: ALLOW}] It is recommended to allow access to the client
	\item[\texttt{1: BLOCK}] It is recommended to deny access to the client
	\item[\texttt{2: ISOLATE}] It is recommended to isolate the client
	\item[\texttt{3: NONE}]
\end{description*}

The values should be submitted to the API as integer values.

\subsection{Hash-Set}

A hash-set is used in the file documents. They are structured as follows:

\begin{lstlisting}
[
	{
		"algorithm": "<str,algorithm-name>",
		"hash": "<hex,hash>	",
		"product": "<uri,product>"
	}
]
\end{lstlisting}

If an algorithm doesn't exist yet, it will be created automatically. Current
default algorithms are \texttt{SHA384, SHA256, SHA1, SHA1-IMA}.


\pagebreak


\section{REST Resources}

\subsection{Session control}

\subsubsection{Controller}

\begin{mdframed}[style=def]
\begin{description*}
	\item[URI Path] \texttt{/sessionss/start/}
	\item[Archetype] Controller
	\item[Methods] POST
	\item[Request Parameters] \hfill
	\begin{description*}
		\item[\texttt{connectionId}] strongSwan Connection ID
		\item[\texttt{clientIdentity}] strongSwan Client-Identity
		\item[\texttt{hardwareId}] The ID that identifies the device, for example 
		the AIK, Android-ID, DBUS Machine-ID, etc.
		\item[\texttt{productName}] The productName is the name of the operating
			system as present in the \texttt{product} table of the database.
	\end{description*}
	\item[JSON Formatted Response] \hfill
\begin{lstlisting}
{
	"sessionId": <int,id>,
	"workitems": [
		 <doc,workitem>,
		 ...
	],
	"uri": "<uri,session>"
}
\end{lstlisting}
	\item[Description] This controller creates and starts a session. The device to
		be assigned to the session is determined by the \texttt{hardwareId} and the
		\texttt{productName}. If one of the related objects does not exist yet, it
		will be created by this controller. The ID which is returned in the response
		document is used to identify the initiated session. Furthermore, a list of
		workitems is returned in the response. These workitems should be processed
		in the corresponding session.
\end{description*}
\end{mdframed}


\begin{mdframed}[style=def]
\begin{description*}
	\item[URI Path] \texttt{/sessions/\{id\}/end/}
	\item[Archetype] Controller
	\item[Methods] POST
	\item[Request Parameters] \hfill
	\begin{description*}
		\item[\texttt{recommendation}] Final result / recommendation for this
			session.
	\end{description*}
	\item[Description] This controller finalizes the session and starts all
		related processes. Among other things, the workitems related to this session
		will be removed, their results will be stored and can be accessed via
		\texttt{/sessions/\{id\}/results}.
\end{description*}
\end{mdframed}


\pagebreak


\subsubsection{Documents and Collections}

\begin{mdframed}[style=def]
\begin{description*}
	\item[URI Path] \texttt{/sessions/\{id\}/}
	\item[Archetype] Readonly Document
	\item[Methods] GET
	\item[JSON Formatted Response] \hfill
\begin{lstlisting}
{
	"id": <int,id>,
	"uri": "<uri,resource>",
	"time": <int,time>,
	"identity": "<uri,identity>",
	"connectionId": <int,connection-id>,
	"device": "<uri,device>",
	"recommendation": <int,rec>
}
\end{lstlisting}
	\item[Description] Information related to a specific session. Sessions should
		never be changed directly, changes can only be made via the corresponding
		controllers. The reason for this is that a change in the session requires
		more processing tasks in the background.
\end{description*}
\end{mdframed}

\begin{mdframed}[style=def]
\begin{description*}
	\item[URI Path] \texttt{/sessions/}
	\item[Archetype] Readonly Collection
	\item[Filter Query] \hfill
	\begin{description*}
		\item[timeFrom] \texttt{<int,timestamp>}
		\item[timeTo] \texttt{<int,timestamp>}
	\end{description*}
	\item[Methods] GET
	\item[JSON Formatted Response] \hfill
\begin{lstlisting}
[<doc,session>, ...]
\end{lstlisting}
	\item[Description] List of all sessions. New sessions can only be created via
		the corresponding controllers.
\end{description*}
\end{mdframed}


\pagebreak


\paragraph{Workitems}\hfill \\

\begin{mdframed}[style=def]
\begin{description*}
	\item[URI Path] \texttt{/sessions/\{id\}/workitems/\{id\}/}
	\item[Archetype] Readonly Document
	\item[Methods] GET
	\item[JSON Formatted Response] \hfill
\begin{lstlisting}
{
	"id": <int,id>,
	"uri": "<uri,resource>",
	"session": "<uri,session>",
	"type": <int,type>,
	"argument": <policy-argument>
}
\end{lstlisting}
	\item[Description] A workitem related to the specified session. Workitems
		cannot be created directly, they will be created automatically during the
		session initialization depending on the configured enforcements. Workitems
		only exist as long as a session is active. After the session has ended, a
		\texttt{HTTP 404} status code will be returned.
\end{description*}
\end{mdframed}

\begin{mdframed}[style=def]
\begin{description*}
	\item[URI Path] \texttt{/sessions/\{id\}/workitems/}
	\item[Archetype] Readonly Collection
	\item[Filter Query] \hfill
	\begin{description*}
		\item[type] \texttt{<int,policy-type>}
	\end{description*}
	\item[Methods] GET
	\item[JSON Formatted Response] \hfill
\begin{lstlisting}
[<doc,workitem>, ...]
\end{lstlisting}
	\item[Description] A list of all workitems that belong to the specified
		session. Workitems cannot be created directly, they will be created
		automatically during the session initialization depending on the configured
		enforcements. Workitems only exist as long as a session is active. After the
		session has ended, an empty list will be returned.
\end{description*}
\end{mdframed}


\begin{mdframed}[style=def]
\begin{description*}
	\item[URI Path] \texttt{/sessions/\{id\}/workitems/\{id\}/result/}
	\item[Archetype] Document
	\item[Request Parameters] \hfill
	\begin{description*}
		\item[\texttt{recommendation}] Result / recommendation for this workitem.
		\item[\texttt{comment}] Comment concerning the result.
	\end{description*}
	\item[Methods] GET, POST
	\item[Response Statuscodes] \hfill
		\begin{description*}
			\item[201 Created] Result has been saved successfully.
			\item[409 Conflict] Result already exists.
		\end{description*}
	\item[JSON Formatted Response] \hfill
\begin{lstlisting}
{
	"recommendation": <int,type>,
	"comment": <str,comment>
}
\end{lstlisting}
	\item[Description] To this resource the result of the corresponding workitem
		should be submitted. The results are transferred from the workitem to the
		session result when the session is closed.
\end{description*}
\end{mdframed}


\pagebreak


\paragraph{Results} \hfill \\

\begin{mdframed}[style=def]
\begin{description*}
	\item[URI Path] \texttt{/sessions/\{id\}/results/\{id\}/}
	\item[Archetype] Readonly Collection
	\item[Methods] GET
	\item[JSON Formatted Response] \hfill
\begin{lstlisting}
{
	"id": <int,id>,
	"uri": "<uri,resource>",
	"enforcement": "<uri,enforcement>",
	"recommendation": <int,type>,	 
	"comment": "<str,comment>"
}
\end{lstlisting}
	\item[Description] A single result. In contrast to the result resource
		relative to the workitem resource, this document also contains a link to the
		corresponding enforcement.
\end{description*}
\end{mdframed}

\begin{mdframed}[style=def]
\begin{description*}
	\item[URI Path] \texttt{/sessions/\{id\}/results/}
	\item[Archetype] Readonly Collection
	\item[Methods] GET
	\item[JSON Formatted Response] \hfill
\begin{lstlisting}
[<doc,result>, ...]
\end{lstlisting}
	\item[Description] After closing a session, this collection allows the
		querying of results. The collection is readonly, so that session results
		cannot be changed afterwards.
\end{description*}
\end{mdframed}


\pagebreak


\subsection{SWID Extension to strongTNC}

\subsubsection{SWID Tags: Measurement}

\begin{mdframed}[style=def]
\begin{description*}
	\item[URI Path] \texttt{/sessions/\{id\}/swid-measurement/}
	\item[Archetype] Controller
	\item[Methods] POST
	\item[Content-Type] \texttt{application/json; charset=utf-8}
	\item[Request Parameters] \hfill
	\begin{description*}
		\item[\texttt{softwareId}] Software-IDs as JSON-list.
	\end{description*}
	\item[Response Statuscodes] \hfill
		\begin{description*}
			\item[200 OK] SWID Tags for the submitted software IDs alreay exist\\
				and have been linked with the current session.
			\item[404 Not Found] Session with the specified ID could not be found.
			\item[412 Precondition Failed] For some of the submited software IDs no
				SWID tags \\ could be found. The response body contains a list of those
				software IDs.
		\end{description*}
	\item[JSON Formatted Response] \hfill
\begin{lstlisting}
["<str,software-id>", ...]
\end{lstlisting}
	\item[Description] A list of software IDs can be submitted to this endpoint.
		If SWID tags already exist for all submitted software IDs, they are linked
		with the current session and a \texttt{HTTP 200} status code is returned.\\
		Otherwise, a list of missing software IDs is returned. The requester first
		has to add these SWID tags via the SWID tag endpoint. Afterwards he can
		repeat the original request.
\end{description*}
\end{mdframed}


\subsubsection{SWID Tags: Creation}

\begin{mdframed}[style=def]
\begin{description*}
	\item[URI Path] \texttt{/swid/add-tags/}
	\item[Archetype] Controller
	\item[Methods] POST
	\item[Content-Type] \texttt{application/json; charset=utf-8}
	\item[Request Parameters] \hfill
		\begin{description*}
			\item[\texttt{xmlData}] SWID Tag as JSON-list.
		\end{description*}
	\item[Response Statuscodes] \hfill
		\begin{description*}
			\item[200 OK] SWID tags have been processed successfully.
			\item[400 Bad Request] Details regarding the error are contained in the
				response body.
		\end{description*}
	\item[Description] The submitted tags are stored in the database. If a tag
		already exists, it is ignored.
\end{description*}
\end{mdframed}


\pagebreak


\subsection{CRUD Resourcen}
\subsubsection{Products}

\begin{mdframed}[style=def]
\begin{description*}
	\item[URI Path] \texttt{/products/\{id\}/}
	\item[Archetype] Document
	\item[Methods] GET, PUT, PATCH
	\item[JSON Formatted Response] \hfill
\begin{lstlisting}
{
	"id": <int,id>,
	"uri": "<uri,resource>",
	"name": "<str,productname>"
}
\end{lstlisting}
\end{description*}
\end{mdframed}

\begin{mdframed}[style=def]
\begin{description*}
	\item[URI Path] \texttt{/products/}
	\item[Archetype] Collection
	\item[Methods] GET, POST
	\item[JSON Formatted Response] \hfill
\begin{lstlisting}
[
	{
		"id": <int,id>,
		"uri": "<uri,resource>",
		"name": "<str,productname>"	
	}
]
\end{lstlisting}
\end{description*}
\end{mdframed}

\begin{mdframed}[style=def]
\begin{description*}
	\item[URI Path] \texttt{/products/\{id\}/default-groups/}
	\item[Archetype] Collection
	\item[Methods] GET, POST
	\item[Request Parameters] \hfill
	\begin{description*}
		\item[\texttt{groupId}] group-id
	\end{description*}
	\item[JSON Formatted Response] \hfill
\begin{lstlisting}
[<doc,group>, ...]
\end{lstlisting}
\end{description*}
\end{mdframed}


\pagebreak


\subsubsection{Packages}

\begin{mdframed}[style=def]
\begin{description*}
	\item[URI Path] \texttt{/packages/\{id\}/}
	\item[Archetype] Document
	\item[Methods] GET, PUT, PATCH
	\item[JSON Formatted Response] \hfill
\begin{lstlisting}
{
	"id": <int,id>,
	"uri": "<uri,resource>",
	"name": "<str,packagename>"
}
\end{lstlisting}
\end{description*}
\end{mdframed}

\begin{mdframed}[style=def]
\begin{description*}
	\item[URI Path] \texttt{/packages/}
	\item[Archetype] Collection
	\item[Methods] GET, POST
	\item[JSON Formatted Response] \hfill
\begin{lstlisting}
[
	{
		"id": <int,id>,
		"uri": "<uri,resource>",
		"name": "<str,packagename>",
	}
]
\end{lstlisting}
\end{description*}
\end{mdframed}

\begin{mdframed}[style=def]
\begin{description*}
	\item[URI Path] \texttt{/packages/\{id\}/versions/}
	\item[Archetype] Collection
	\item[Methods] GET, POST
	\item[JSON Formatted Response] \hfill
\begin{lstlisting}
[<doc,version>, ...]
\end{lstlisting}
\end{description*}
\end{mdframed}


\pagebreak


\subsubsection{Versions}

\begin{mdframed}[style=def]
\begin{description*}
	\item[URI Path] \texttt{/versions/\{id\}/}
	\item[Archetype] Document
	\item[Methods] GET, PUT, PATCH, DELETE
	\item[JSON Formatted Response] \hfill
\begin{lstlisting}
{
	"id": <int,id>,
	"uri": "<uri,resource>",
	"package": "<uri,package>",
	"product": "<uri,product>",
	"release": "<str,release>",
	"securtiy": <int,security>,
	"blacklist": <bool,blacklist>,
	"time": <int,time>
}
\end{lstlisting}
\end{description*}
\end{mdframed}

\begin{mdframed}[style=def]
\begin{description*}
	\item[URI Path] \texttt{/versions/}
	\item[Archetype] Collection
	\item[Filter Query] \hfill
	\begin{description*}
		\item[productName] \texttt{<str,product-name>}
		\item[packageName] \texttt{<str,package-name>}
	\end{description*}
	\item[Methods] GET, POST
	\item[JSON Formatted Response] \hfill
\begin{lstlisting}
[
	{
		"id": <int,id>,
		"uri": "<uri,resource>",
		"package": "<uri,package>",
		"product": "<uri,product>",
		"release": "<str,release>",
		"securtiy": <int,security>,
		"blacklist": <bool,blacklist>,
		"time": <int,time>
	}
]
\end{lstlisting}
\end{description*}
\end{mdframed}


\pagebreak


\subsubsection{Identities}
\begin{mdframed}[style=def]
\begin{description*}
	\item[URI Path] \texttt{/identities/\{id\}/}
	\item[Archetype] Document
	\item[Methods] GET, PUT, PATCH
	\item[JSON Formatted Response] \hfill
\begin{lstlisting}
{
	"id": <int,id>,
	"uri": "<uri,resource>",
	"type": <int,type>,
	"value": "<str,value>"
}
\end{lstlisting}
\end{description*}
\end{mdframed}

\begin{mdframed}[style=def]
\begin{description*}
	\item[URI Path] \texttt{/identities/}
	\item[Archetype] Collection
	\item[Methods] GET, POST
	\item[JSON Formatted Response] \hfill
\begin{lstlisting}
[
	{
		"id": <int,id>,
		"uri": "<uri,resource>",
		"type": <int,type>,
		"value": "<str,value>"
	}
]
\end{lstlisting}
\end{description*}
\end{mdframed}


\pagebreak


\subsubsection{Devices}

\begin{mdframed}[style=def]
\begin{description*}
	\item[URI Path] \texttt{/devices/\{id\}/}
	\item[Archetype] Document
	\item[Methods] GET, PUT, PATCH
	\item[JSON Formatted Response] \hfill
\begin{lstlisting}
{
	"id": <int,id>,
	"uri": "<uri,resource>",
	"description": "<str,description>",
	"value": "<str,value>",
	"product": "<uri,product>",
	"created": <int,created>
}
\end{lstlisting}
\end{description*}
\end{mdframed}

\begin{mdframed}[style=def]
\begin{description*}
	\item[URI Path] \texttt{/devices/}
	\item[Archetype] Collection
	\item[Methods] GET, POST
	\item[JSON Formatted Response] \hfill
\begin{lstlisting}
[
	{
		"id": <int,id>,\\
		"uri": "<uri,resource>",
		"description": "<str,description>",
		"value": "<str,value>",
		"product": "<uri,product>",
		"created": <int,created>,
	}
]
\end{lstlisting}
\end{description*}
\end{mdframed}

\begin{mdframed}[style=def]
\begin{description*}
	\item[URI Path] \texttt{/device/\{id\}/sessions/}
	\item[Archetype] Readonly Collection
	\item[Filter Query] \hfill
	\begin{description*}
		\item[timeFrom] \texttt{<int,timestamp>}
		\item[timeTo] \texttt{<int,timestamp>}
	\end{description*}	
	\item[Methods] GET
	\item[JSON Formatted Response] \hfill
\begin{lstlisting}
[<doc,session>, ...]
\end{lstlisting}
\end{description*}
\end{mdframed}

\begin{mdframed}[style=def]
\begin{description*}
	\item[URI Path] \texttt{/device/\{id\}/sessions/\{id\}/results/}
	\item[Archetype] Readonly Collection 
	\item[Methods] GET
	\item[JSON Formatted Response] \hfill
\begin{lstlisting}
[<doc,result>, ...]
\end{lstlisting}
\end{description*}
\end{mdframed}

\begin{mdframed}[style=def]
\begin{description*}
	\item[URI Path] \texttt{/device/\{id\}/groups/}
	\item[Archetype] Collection 
	\item[Methods] GET, POST
	\item[JSON Formatted Response] \hfill
\begin{lstlisting}
[<doc,group>, ...]
\end{lstlisting}
\end{description*}
\end{mdframed}

\begin{mdframed}[style=def]
\begin{description*}
	\item[URI Path] \texttt{/device/\{id\}/swid-tags/}
	\item[Archetype] Readonly Collection 
	\item[Methods] POST
	\item[JSON Formatted Response] \hfill
\begin{lstlisting}
[<doc,swid-tag>, ...]
\end{lstlisting}
\end{description*}
\end{mdframed}


\pagebreak


\subsubsection{Enforcements}
\begin{mdframed}[style=def]
\begin{description*}
	\item[URI Path] \texttt{/enforcements/\{id\}/}
	\item[Archetype] Readonly Document
	\item[Methods] GET
	\item[JSON Formatted Response] \hfill
\begin{lstlisting}
{
	"id": <int,id>,
	"uri": "<uri,resource>",
	"policy": "<uri,policy>",
	"group": "<uri,group>",
	"failRecommendation": <int,rec_fail>,
	"noresultRecommendation": <int,rec_noresult>", 
	"maxAge": <int,max_age>
}
\end{lstlisting}
\end{description*}
\end{mdframed}

\begin{mdframed}[style=def]
\begin{description*}
	\item[URI Path] \texttt{/enforcements/}
	\item[Archetype] Readonly Collection
	\item[Methods] GET
	\item[Filter Query] \hfill
	\begin{description*}
		\item[groupName] \texttt{<str,group-name>}
		\item[policyName] \texttt{<str,policy-name>}
	\end{description*}	
	\item[JSON Format] \hfill
\begin{lstlisting}
[
	{
		"id": <int,id>,
		"uri": "<uri,resource>",
		"policy": "<uri,policy>",
		"group": "<uri,group>",
		"failRecommendation": <int,rec_fail>,
		"noresultRecommendation": <int,rec_noresult>",
		"maxAge": <int,max_age>
	}
]
\end{lstlisting}
\end{description*}
\end{mdframed}

\begin{mdframed}[style=def]
\begin{description*}
	\item[URI Path] \texttt{/enforcements/\{id\}/groups/}
	\item[Archetype] Readonly Collection
	\item[Methods] GET
	\item[JSON Format] \hfill
\begin{lstlisting}
[<doc,group>, ...]
\end{lstlisting}
\end{description*}
\end{mdframed}


\pagebreak


\subsubsection{Policies}

\begin{mdframed}[style=def]
\begin{description*}
	\item[URI Path] \texttt{/policies/\{id\}/}
	\item[Archetype] Readonly Document
	\item[Methods] GET
	\item[JSON Formatted Response] \hfill
\begin{lstlisting}
{
	"id": <int,id>,
	"uri": "<uri,resource>",
	"type": <int,type>,
	"name": "<str,name>",
	"argument": <policy-argument>,
	"failRecommendation": <int,rec_fail>,
	"noresultRecommendation": <int,rec_noresult>",
}
\end{lstlisting}
\end{description*}
\end{mdframed}

\begin{mdframed}[style=def]
\begin{description*}
	\item[URI Path] \texttt{/policies/}
	\item[Archetype] Readonly Collection
	\item[Methods] GET
	\item[JSON Format] \hfill
\begin{lstlisting}
[
	{
		"id": <int,id>,
		"uri": "<uri,resource>",
		"type": <int,type>,
		"name": "<str,name>",
		"argument": <policy-argument>,
		"failRecommendation": <int,rec_fail>,
		"noresultRecommendation": <int,rec_noresult>"
	}
]
\end{lstlisting}
\end{description*}
\end{mdframed}

\begin{mdframed}[style=def]
\begin{description*}
	\item[URI Path] \texttt{/policies/\{id\}/enforcements/}
	\item[Archetype] Readonly Collection
	\item[Methods] GET
	\item[JSON Format] \hfill
\begin{lstlisting}
[<doc,enforcement>, ...]
\end{lstlisting}
\end{description*}
\end{mdframed}


\pagebreak


\subsubsection{Groups}

\begin{mdframed}[style=def]
\begin{description*}
	\item[URI Path] \texttt{/groups/\{id\}/}
	\item[Archetype] Readonly Document
	\item[Methods] GET
	\item[JSON Formatted Response] \hfill
\begin{lstlisting}
{
	"id": <int,id>,
	"uri": "<uri,resource>",
	"name": "<str,name>",
	"parent": "<uri,group>"
}
\end{lstlisting}
\end{description*}
\end{mdframed}

\begin{mdframed}[style=def]
\begin{description*}
	\item[URI Path] \texttt{/groups/}
	\item[Archetype] Readonly Collection
	\item[Methods] GET
	\item[JSON Format] \hfill
\begin{lstlisting}
[
	{
		"id": <int,id>,
		"uri": "<uri,resource>",
		"name": "<str,name>",
		"parent": "<uri,group>"
	}
]
\end{lstlisting}
\end{description*}
\end{mdframed}

\begin{mdframed}[style=def]
\begin{description*}
	\item[URI Path] \texttt{/groups/\{id\}/devices/}
	\item[Archetype] Collection
	\item[Methods] GET, POST
	\item[JSON Formatted Response] \hfill
\begin{lstlisting}
[<doc,device>, ...]
\end{lstlisting}
\end{description*}
\end{mdframed}

\begin{mdframed}[style=def]
\begin{description*}
	\item[URI Path] \texttt{/groups/\{id\}/enforcements/}
	\item[Archetype] Readonly Collection
	\item[Methods] GET
	\item[JSON Formatted Response] \hfill
\begin{lstlisting}
[<doc,enfocement>, ...]
\end{lstlisting}
\end{description*}
\end{mdframed}


\pagebreak


\subsubsection{Files}

\begin{mdframed}[style=def]
\begin{description*}
	\item[URI Path] \texttt{/files/\{id\}/}
	\item[Archetype] Document
	\item[Methods] GET, PUT, PATCH
	\item[JSON Formatted Response] \hfill
\begin{lstlisting}
{
	"id": <int,id>,
	"uri": "<uri,resource>",
	"name": "<str,name>",
	"hashes: <doc,hash-set>,
	"dir": "<uri,directory>"
}
\end{lstlisting}
\end{description*}
\end{mdframed}

\begin{mdframed}[style=def]
\begin{description*}
	\item[URI Path] \texttt{/files/}
	\item[Archetype] Collection
	\item[Methods] GET, POST
	\item[JSON Formatted Response] \hfill
\begin{lstlisting}
[
	{
		"id": <int,id>,
		"uri": "<uri,resource>",
		"name": "<str,name>"
		"hashes: <doc,hash-set>,
		"dir": "<uri,directory>"
	}
]
\end{lstlisting}
\end{description*}
\end{mdframed}


\pagebreak


\subsubsection{Directories}

\begin{mdframed}[style=def]
\begin{description*}
	\item[URI Path] \texttt{/directories/\{id\}/}
	\item[Archetype] Document
	\item[Methods] GET, PUT, PATCH
	\item[JSON Formatted Response] \hfill
\begin{lstlisting}
{
	"id": <int,id>,
	"uri": "<uri,resource>",
	"path": "<str,path>"
}
\end{lstlisting}
\end{description*}
\end{mdframed}

\begin{mdframed}[style=def]
\begin{description*}
	\item[URI Path] \texttt{/directories/}
	\item[Archetype] Collection
	\item[Methods] GET, POST
	\item[JSON Formatted Response] \hfill
\begin{lstlisting}
[
	{
		"id": <int,id>,
		"uri": "<uri,resource>",
		"path": "<str,path>"
	}
]
\end{lstlisting}
\end{description*}
\end{mdframed}

\begin{mdframed}[style=def]
\begin{description*}
	\item[URI Path] \texttt{/directories/\{id\}/files/}
	\item[Archetype] Collection
	\item[Methods] GET, POST
	\item[JSON Formatted Response] \hfill
\begin{lstlisting}
[<doc,file>, ...]
\end{lstlisting}
\end{description*}
\end{mdframed}


\pagebreak


\subsubsection{SWID Tags}

\begin{mdframed}[style=def]
\begin{description*}
	\item[URI Path] \texttt{/swid-tags/}
	\item[Archetype] Readonly Collection
	\item[Methods] GET
	\item[JSON Formatted Response] \hfill
\begin{lstlisting}
[<doc,swid-tag>, ...]
\end{lstlisting}
\end{description*}
\end{mdframed}

\begin{mdframed}[style=def]
\begin{description*}
	\item[URI Path] \texttt{/swid-tags/\{id\}/}
	\item[Archetype] Readonly Document
	\item[Methods] GET
	\item[Filter Query] \hfill
	\begin{description*}
		\item[packageName] \texttt{<str,package-name>}
		\item[version] \texttt{<str,version>}
		\item[uniqueId] \texttt{<str,unique-id>}
	\end{description*}
	\item[JSON Formatted Response] \hfill
\begin{lstlisting}
{
	"id": <int,id>,
	"uri": "<uri,resource>",
	"packageName": "<str,package-name>",
	"version": "<str,version>",
	"uniqueId": "<str,unique-id>",
	"entities": [
		{
		"entity": "<uri,entity>",
		"role":  <int,role>
		}
	],
	"tagXml": "<xml,swid-tag>"
}
\end{lstlisting}
\end{description*}
\end{mdframed}

\begin{mdframed}[style=def]
\begin{description*}
	\item[URI Path] \texttt{/swid-tags/\{id\}/files/}
	\item[Archetype] Readonly Collection
	\item[Methods] GET
	\item[JSON Formatted Response] \hfill
\begin{lstlisting}
[<doc,file>, ...]
\end{lstlisting}
\end{description*}
\end{mdframed}


\pagebreak


\subsubsection{Entities}

\begin{mdframed}[style=def]
\begin{description*}
	\item[URI Path] \texttt{/swid-entities/\{id\}/}
	\item[Archetype] Readonly Document
	\item[Methods] GET
	\item[JSON Formatted Response] \hfill
\begin{lstlisting}
{
	"id": <int,id>,
	"uri": "<uri,resource>",
	"name": "<str,name>",
	"regid": "<str,regid>"
}
\end{lstlisting}
\end{description*}
\end{mdframed}

\begin{mdframed}[style=def]
\begin{description*}
	\item[URI Path] \texttt{/swid-entities/}
	\item[Archetype] Readonly Collection
	\item[Methods] GET
	\item[JSON Formatted Response] \hfill
\begin{lstlisting}
[
	{
		"id": <int,id>,
		"uri": "<uri,resource>",
		"name": "<str,name>",
		"regid": "<str,regid>",
	}
]
\end{lstlisting}
\end{description*}
\end{mdframed}

\begin{mdframed}[style=def]
\begin{description*}
	\item[URI Path] \texttt{/swid-entities/\{id\}/swid-tags/}
	\item[Archetype] Readonly Collection
	\item[Methods] GET
	\item[JSON Formatted Response] \hfill
\begin{lstlisting}
[<doc,swid-tag>, ...]
\end{lstlisting}
\end{description*}
\end{mdframed}


\pagebreak


\begin{thebibliography}{1}
	\bibitem{masse2011rest}
	Masse, Mark. 
	\emph{REST API design rulebook.} 
	O'Reilly Media, Inc., 2011.
	
	\bibitem{rfc3986}
	Berners-Lee, Tim, Roy Fielding, and Larry Masinter. 
	\emph{RFC 3986: Uniform resource identifier (uri): Generic syntax.}
	The Internet Society (2005).
		
\end{thebibliography}
\end{document}
