\documentclass[10pt,a4paper]{scrartcl}
\pagestyle{empty}
\usepackage{a4wide}
\usepackage[ngerman]{babel} % Neudeutsche Silbentrennung (mehrsprachiges Dokument)
\usepackage{parskip} % Skip indentation of first row
\usepackage{graphicx} % Graphics support
\usepackage{longtable} % Tables across several pages
\usepackage{booktabs}
\usepackage{mdwlist} % lists with less spacing, use itemize*
\usepackage{hyperref} % Hyperlinks
\usepackage{float} % Force float position
\usepackage[automark]{scrpage2} %kopf/fusszeile
\usepackage{listings}
\usepackage[utf8x]{inputenc} % Unicode-Encoding
\usepackage[framemethod=TikZ]{mdframed}
\usepackage{mdframed}
\usepackage{minted}
\usemintedstyle{tango}
\usepackage{color}
\mdfdefinestyle{def}{%
	linecolor=gray!50!white,
	outerlinewidth=0.5pt,
	roundcorner=3pt,
	skipabove=\topskip,
	skipbelow=\topskip
	innertopmargin=\baselineskip,
	innerbottommargin=\baselineskip,
	innerrightmargin=10pt,
	innerleftmargin=10pt,
	nobreak=true
}

\definecolor{tango-bg}{HTML}{F8F8F8}
\newminted{http}{tabsize=4,bgcolor=tango-bg,frame=lines,framesep=2mm,samepage=true,fontsize=\footnotesize}
\newminted{bash}{tabsize=4,bgcolor=tango-bg,frame=lines,framesep=2mm,samepage=true,fontsize=\footnotesize}
\newminted{xml}{tabsize=4,bgcolor=tango-bg,frame=lines,framesep=2mm,samepage=true,fontsize=\footnotesize}

\linespread{1.3}

\author{Danilo Bargen, Christian Fässler, Jonas Furrer} 
\title{strongTNC REST API\\SWID Endpoints \\ \small{Im Rahmen der strongTNC BA} }
\date{Frühlingssemester 2014}

\pagestyle{scrheadings}
\ihead{SWID Measurement Manual} %linke Kopfzeile
\ohead{strongTNC BA} %rechte Kopfzeile

\newcommand*{\escape}[1]{\texttt{\textbackslash#1}}

\let\textquotedbl="

\begin{document}

\begin{titlepage}
	\maketitle
	\vspace{120mm}
	\thispagestyle{empty} % Don't start page numbers on this page
\end{titlepage}

\newpage
	\tableofcontents
\newpage

\section{Ablauf der Messung}

Eine Messung läuft wie folgt ab:

\begin{enumerate}
	\item Auf dem Client werden die Software-IDs aller installierter Pakete generiert.
	\item Die Software-IDs werden als JSON-Liste an den entsprechenden
		API-Endpoint [\ref{api:measurement}] gesendet.
	\begin{itemize}
		\item Wenn von der API ``\texttt{200 OK}'' zurückgegeben wurde, ist alles OK und die SWID
			Tags wurden mit der aktuellen Session verknüpft. ENDE.
		\item Wenn von der API ``\texttt{404 Not Found}'' zurückgegeben wurde, ist
			die Session ID ungültig. Problem beheben und weiter zu Punkt 2.
		\item Wenn von der API ``\texttt{412 Precondition Failed}'' zurückgegeben wurde, sind
			noch nicht alle SWID Tags in der Datenbank erfasst. Weiter zu Punkt 3.
	\end{itemize}
	\item Die fehlenden SWID Tags werden im Response-Body als JSON-Liste
		zurückgegeben. Diese werden nun über Targeted Requests generiert.
	\item Die neu generierten Tags werden als JSON-Listen an den entsprechenden
		API-Endpoint [\ref{api:create}] gesendet. Es können auch mehrere Tags in
		einem Request mitgesendet werden, empfehlenswert ist allerdings ein
		Batching in mehrere Gruppen, da die Requests sonst unter Umständen sehr
		lange dauern können.
	\begin{itemize}
		\item Wenn von der API ``\texttt{200 OK}'' zurückgegeben wurde, ist alles
			OK. Weiter zu Punkt 5.
		\item Wenn von der API ``\texttt{400 Bad Request}'' zurückgegeben wurde,
			stimmt etwas mit dem Request nicht. Fehler-Details werden im Response Body
			zurückgesendet.
	\end{itemize}
	\item Nachdem alle fehlenden SWID Tags eingetragen wurden, kann der
		Measurement-Request erneut gesendet werden. Weiter zu Punkt 1.
\end{enumerate}

\pagebreak
\section{Beispiel Szenario}
Im Folgenden soll der Ablauf einer Messung anhand eines konkreten Beispiels
illustriert werden. In diesem Beispiel soll eine Messung für die drei Tags mit
den folgenden Software-IDs erfolgen:
\begin{itemize*}
	\item \texttt{regid.2004-03.org.strongswan\_Ubuntu\_12.04-i686-logrotate-3.7.8-6ubuntu5}
	\item \texttt{regid.2004-03.org.strongswan\_Ubuntu\_12.04-i686-lsb-base-4.0-0ubuntu20}
	\item \texttt{regid.2004-03.org.strongswan\_Ubuntu\_12.04-i686-strongswan-4.5.2-1.5+deb7u3}
\end{itemize*}

\subsection{Request - SWID Measurement}

\subsubsection{Request}

Zuerst wird versucht eine Messung für die Session mit der ID 2 zu erfassen.

\begin{listing}
\caption{Durchführen einer SWID Messung}
\begin{httpcode}
POST /api/sessions/2/swid-measurement/ HTTP/1.1
Authorization: Basic cm9vdDpyb290
Host: tncserver:8000
Accept: application/json
Content-Type: application/json; charset=utf-8
Content-Length: 232

[
"regid.2004-03.org.strongswan_Ubuntu_12.04-i686-logrotate-3.7.8-6ubuntu5",
"regid.2004-03.org.strongswan_Ubuntu_12.04-i686-lsb-base-4.0-0ubuntu20",
"regid.2004-03.org.strongswan_Ubuntu_12.04-i686-strongswan-4.5.2-1.5+deb7u3"
]
\end{httpcode}
\end{listing}

Curl Befehl:

\begin{listing}
\caption{Durchführen einer SWID Messung mit cURL}
\begin{bashcode}
curl -i -X POST http://tncserver:8000/api/sessions/2/swid-measurement/ \
		 -u username:password \
		 -H "Accept: application/json" \
		 -H "Content-Type: application/json; charset=utf-8" \
		 -d "{ \"data\" : $DATA }"
\end{bashcode}
\end{listing}

\subsubsection{Antwort - Status 412 Precondition Failed}

Sind noch nicht alle Tags in der strongTNC Datenbank vorhanden, werden die
Software-IDs der fehlenden Tags zurückgeliefert. Der HTTP Status Code ist
``\texttt{412 Precondition Failed}''. Es werden zu diesem Zeitpunkt noch keine
Tags mit der Session verknüpft. Eine entsprechende Antwort kann wie folgt
aussehen:

\begin{listing}
\caption{Antwort einer nicht erfolgreichen SWID Messung}
\begin{httpcode}
HTTP/1.0 412 PRECONDITION FAILED
Date: Wed, 14 May 2014 15:21:45 GMT
Server: WSGIServer/0.1 Python/2.7
Vary: Accept, Cookie
Content-Type: application/json
Allow: POST, OPTIONS

["regid.2004-03.org.strongswan_Ubuntu_12.04-i686-strongswan-4.5.2-1.5+deb7u3"]
\end{httpcode}
\end{listing}

\subsubsection{Antwort - Status 200 OK}

Wären für alle Software-IDs bereits Tags in der Datenbank vorhanden, könnte die Antwort wie folgt aussehen:

\begin{listing}
\caption{Antwort einer erfolgreichen SWID Messung}
\begin{httpcode}
HTTP/1.0 200 OK
Date: Wed, 14 May 2014 15:21:45 GMT
Server: WSGIServer/0.1 Python/2.7
Vary: Accept, Cookie
Content-Type: application/json
Allow: POST, OPTIONS

[]
\end{httpcode}
\end{listing}

\subsection{Nacherfassen von SWID Tags}

\subsubsection{Request}

War die Messung nicht erfolgreich (HTTP 412), müssen für die zurückgelieferten
Software-IDs zuerst komplette SWID Tags erfasst werden.

Im SWID Generator kann ein solcher wie folgt abgefragt werden:

\begin{small}\begin{verbatim}
	swid-generator swid \
	  --software-id="regid.2004-03.org.strongswan\_Ubuntu\_12.04-i686-fortune-mod-1:1.99.1-4"
\end{verbatim}\end{small}

Ein solcher SWID Tag sieht wie folgt aus:

\begin{listing}
\caption{Beispiel eines SWID Tags}
\begin{xmlcode}
<?xml version="1.0" encoding="UTF-8"?>
<SoftwareIdentity name="strongswan" 
	uniqueId="debian_7.4-x86_64-strongswan-4.5.2-1.5+deb7u3" 
	version="4.5.2-1.5+deb7u3" versionScheme="alphanumeric" 
	xmlns="http://standards.iso.org/iso/19770/-2/2014/schema.xsd">
  <Entity name="strongSwan" regid="regid.2004-03.org.strongswan" role="tagcreator"/>
  <Entity name="HSR" regid="regid.2004-03.org.strongswan" role="publisher"/>
  <Payload>
    <File location="/usr/sbin" name="charon-cmd"/>
    <File location="/usr/sbin" name="strongswan"/>
    <File location="/usr/lib/systemd/system" name="strongswan.service"/>
    <File location="/usr/lib64/strongswan" name="libcharon.so.0"/>
    <File location="/usr/lib64/strongswan" name="libcharon.so.0.0.0"/>
    <File location="/usr/lib64/strongswan" name="libhydra.so.0"/>
    <File location="/usr/lib64/strongswan" name="libhydra.so.0.0.0"/>
    <File location="/usr/lib64/strongswan" name="libpttls.so.0"/>
    <File location="/usr/lib64/strongswan" name="libpttls.so.0.0.0"/>
    <File location="/usr/lib64/strongswan" name="libstrongswan.so.0"/>
    <File location="/usr/share/doc/strongswan-5.1.2" name="README"/>
  </Payload>
</SoftwareIdentity>
\end{xmlcode}
\end{listing}

Dieser Tag kann anschliessend in der strongTNC Datenbank mittels POST Request
auf die Ressource \texttt{/swid/add-tags/} erfasst werden. Die Daten für diesen
Request bestehen aus einer JSON Liste von XML Strings, dadurch können auch
mehrere Tags auf einmal erfasst werden. Empfehlenswert ist allerdings ein
Batching in mehrere Gruppen, da die Requests sonst unter Umständen sehr lange
dauern können. 

Falls bereits ein Tag mit der selben Software-ID existiert, wird er komplett
überschrieben. Der Request sieht folgendermassen aus:

\begin{listing}[H]
\caption{Erfassen eines SWID Tags}
\begin{httpcode}
POST /api/swid/add-tags/ HTTP/1.1
Authorization: Basic cm9vdDpyb290
Host: tncserver:8000
Accept: application/json
Content-Type: application/json; charset="utf-8"
Content-Length: 239

{ "data" : 
	[
		"<?xml version=\"1.0\" encoding=\"utf-8\"?><SoftwareIdentity name=\"fortune-mod\"...",
		"<?xml version=\"1.0\" encoding=\"UTF-8\"?><SoftwareIdentity name=\"strongswan\"..."
	]
}
\end{httpcode}
\end{listing}
\pagebreak
Curl Befehl:

\begin{listing}[H]
\caption{Erfassen eines SWID Tags mit cURL}
\begin{bashcode}
curl -i -X POST http://tncserver:8000/api/swid/add-tags/ \
		 -u username:password \
		 -H "Accept: application/json" \
		 -H "Content-Type: application/json; charset=utf-8" \
		 -d "{ \"data\" : $DATA }"
\end{bashcode}
\end{listing}


\subsubsection{Antwort - Status 400 Bad Request}

Falls der Request nicht in Ordnung ist (zB fehlerhaftes oder unvollständiges
XML), könnte die Antwort wie folgt aussehen:

\begin{listing}
\caption{Antwort beim Auftreten eines Fehlers}
\begin{httpcode}
HTTP/1.0 400 BAD REQUEST
Date: Thu, 15 May 2014 21:31:10 GMT
Server: WSGIServer/0.1 Python/2.7.6
Vary: Accept, Cookie
Content-Type: application/json
Allow: POST, OPTIONS

{
	"status": "error", 
	"message": "version: This field cannot be blank. package_name: This field cannot be blank."
}
\end{httpcode}
\end{listing}


\subsubsection{Antwort - Status 200 OK}

Wenn mit dem Request alles in Ordnung ist, könnte die Antwort wie folgt aussehen:

\begin{listing}
\caption{Erfolgreiches Hinzufügen eines SWID Tags}
\begin{httpcode}
HTTP/1.0 200 OK
Date: Thu, 15 May 2014 21:19:19 GMT
Server: WSGIServer/0.1 Python/2.7.6
Vary: Accept, Cookie
Content-Type: application/json
Allow: POST, OPTIONS

{
	"status": "success",
	"message": "Added 1 SWID tags, replaced 0 SWID tags."
}
\end{httpcode}
\end{listing}


\newpage
\section{Allgemeine Hinweise}

\begin{itemize}
	\item JSON-Objekte dürfen im Gegensatz zu Python nur in doppelte
		Anführungszeichen eingeschlossen werden. Beispiel: \texttt{\{"foo": "bar"\}}
		statt \texttt{\{'foo': 'bar'\}}.
	\item Alle übermittelten Daten sollten UTF-8 encoded sein. Der
		\texttt{Content-Type} Header sollte entsprechend übermittelt werden.
		Beispiel: \texttt{Content-Type: application/json; charset=utf-8}
	\item Nichtdruckbare Zeichen sowie doppelte Anführungszeichen innerhalb eines
		JSON Strings (zB bei den XML Tags) müssen mit einem Backslash escaped werden
		(z.B newline, tab, double quotes $\rightarrow$ \escape{n}, \escape{t}, \escape{"}).
\end{itemize}


\pagebreak
\section{API Endpoints}

\subsection{SWID Tags: Messung}
\label{api:measurement}

\begin{mdframed}[style=def]
\begin{description*}
	\item[URI Path] \texttt{/sessions/\{id\}/swid-measurement/}
	\item[Archetype] Controller
	\item[Methods] POST
	\item[Content-Type] \texttt{application/json; charset=utf-8}
	\item[Request Parameter] \hfill
	\begin{description*}
		\item[\texttt{softwareId}] Software-IDs als JSON-Liste.
	\end{description*}
	\item[Response Statuscodes] \hfill
		\begin{description*}
			\item[200 OK] SWID Tags der übermittelten Software-IDs sind eingetragen \\
				und wurden für die übermittelte Session eingetragen
			\item[404 Not Found] Session mit der spezifizierten ID wurde nicht gefunden. 
			\item[412 Precondition Failed] Es existieren nicht alle SWID Tags für die
				übertragenen Software-IDs. Als Payload werden die fehlenden Software-IDs
				übertragen.
		\end{description*}
	\item[JSON Format Response] \hfill
\begin{lstlisting}
["<str,software-id>", ...]
\end{lstlisting}
\end{description*}
\end{mdframed}


\subsection{SWID Tags: Erstellung}
\label{api:create}

\begin{mdframed}[style=def]
\begin{description*}
	\item[URI Path] \texttt{/swid/add-tags/}
	\item[Archetype] Controller
	\item[Methods] POST
	\item[Content-Type] \texttt{application/json; charset=utf-8}
	\item[Request Parameter] \hfill
		\begin{description*}
			\item[\texttt{xmlData}] SWID Tags als JSON-Liste.
		\end{description*}
	\item[Response Statuscodes] \hfill
		\begin{description*}
			\item[200 OK] SWID Tags wurden erfolgreich verarbeitet.
			\item[400 Bad Request] Fehlerhafter Request. Details zum Fehler werden im
				Response-Body zurückgesendet.
		\end{description*}
\end{description*}
\end{mdframed}

\end{document}
