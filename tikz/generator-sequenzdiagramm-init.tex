\begin{tikzpicture}
	\begin{umlseqdiag}[font=\footnotesize]

		% Objekte

		\umlobject{main}
		\umlcreatecall[dt=5, x=8, class=EnvironmentRegistry]{main}{registry}

		% Calls

		\begin{umlfragment}[type=loop, label=foreach env, inner xsep=10]
			\begin{umlcall}[dt=7, padding=2, op={register(env)}]{main}{registry}
			\end{umlcall}
		\end{umlfragment}

		\umlcreatecall[dt=7, x=4, class=MainArgumentParser]{main}{parser}
		\begin{umlcall}[dt=3, padding=4, op={parse(registry)},return=options]{main}{parser}
		\end{umlcall}

		\begin{umlcall}[dt=7, padding=3, op={get\_environment(options)},return=env]{main}{registry}
			\begin{umlfragment}[type=opt, label=if !options.env, inner xsep=15]
				\begin{umlcallself}[op=autodetect\_env()]{registry}
				\end{umlcallself}
			\end{umlfragment}
		\end{umlcall}

	\end{umlseqdiag}
\end{tikzpicture}
