\chapter{Rückblick}


\section{Schlussfolgerungen}

\subsection{Erreichte Ziele}
Alle in der Aufgabenstellung enthaltenen Punkte, Pflicht- und optional,  wurden
erfüllt, ebenso sind alle daraus resultierenden Usecases abgedeckt. Darüber
hinaus, konnten zusätzliche Ideen und Vorschläge eingebracht und umgesetzt
werden.\\
Zusätzlich konnten noch weitere Ziele erreicht werden:                 
\begin{description}                  
                                  
	\item[REST API] Während der Projektzeit wurde ein Konzept für eine umfassende
	REST Schnittstelle für strongTNC ausgearbeitet, mit dem Ziel die Schnittstelle
	in Form einer gemeinsam genutzten Datenbank abzulösen. Das Konzept konnte auch
	als Machbarkeitsnachweis im Bereich der SWID Erweiterung ungesetzt werden.
	
	\item[Code Qualität] Durch das einbringen von Django Konventionen, Testing,
	Code Guidelines und Analyse Tools konnte die Qualtät des Codes gesteigert
	werden.
	
	\item[Rückfluss von Wissen in die Community] Durch die intensive Arbeit mit
	Plugins und Erweiterungen konnten einige Fehler festgestellt werden. Diese
	konnten behoben werden und wurden als Pull Request in den Entwicklern der
	Erweiterungen weder zur Verfügung gestellt.
	
\end{description}


\subsection{Learnings}
Die intensive Arbeit an einem Projekt fordert, dass bestimmte Technologien im
Detail betrachtet werden und dass oft durch geführte Prozesse optimiert werden.
Dabei lernt man Neues und bestätigt altes, im folgenden sind einige Learnings
dieser Arbeit. Die Punkte beziehen sich auf technische oder organisatorische
Merkmale, individuelle persönliche Learnings sind den \enquote{Persönlichen
Berichten} zu entnehmen.

\begin{description}
	\item[Git: Branching und Reviews] Die Arbeit mit \enquote{Feature Branches} und
	konsequentem Review bevor ein Branch in den Master-Branch einfliessen darf,
	hatte diverse positive Effekte: Es konnten Fehler gefunden werden, bevor sie in
	die Produktion geraten, das Wissen über einzelne Komponenten konnte verteilt
	werden. Allerdings mit diesem Vorgehen viel Zeit aufgebraucht. Es ist schwierig
	zu Sagen, ob der sich der erhöhte Zeitaufwand gegenüber der eingesparten Zeit
	durch frühzeitiges finden von Bugs auszahlt. Das vorgehen hat sich jedoch
	bewährt.
	
	\item[SQLite] An verschiedenen Punkten dieses Projektes war die Datenbank
	Bestandteil von Problemstellungen. Nicht selten konnte festgestellt werden,
	dass die technischen Einschränkungen von SQLite die Verursacher dieser Probleme
	waren. SQLite eignet sich gut für Prototyping, Embedded Systeme oder kleine
	einzelbenuzer Systeme, verursacht bei grösseren Infrastrukturen mit
	mehrbenutzer Zugriff starke Probleme.
	
	\item[Testing] 
	
\end{description}

