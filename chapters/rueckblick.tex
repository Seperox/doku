\chapter{Rückblick}

\section{Schlussfolgerungen}

\subsection{Erreichte Ziele}

Alle in der Aufgabenstellung geforderten Ziele wurden erfüllt. Ebenso sind alle
aus der Aufgabenstellung resultierenden Usecases abgedeckt. Darüber hinaus
konnten zusätzliche Ideen und Vorschläge eingebracht und umgesetzt werden:

\begin{description}                  
                                  
	\item[REST API] Während der Projektarbeit wurde ein Konzept für eine
		umfassende REST Schnittstelle für strongTNC ausgearbeitet, mit dem Ziel die
		Schnittstelle in Form einer gemeinsam genutzten Datenbank abzulösen. Das
		Konzept konnte auch als Proof of Concept im Bereich der SWID Erweiterung
		umgesetzt werden.
	
	\item[Code Qualität] Durch das Einbringen von Django-Konventionen, besserem
		Testing, Coding Guidelines und Analyse-Tools konnte die Qualtät des Codes
		messbar gesteigert werden (Absatz \ref{improvements:ci}).

	\item[Zusätzliche Backends für SWID Generator] Zusätzlich zur geforderten
		Unterstützung des unter Debian und Ubuntu genutzten Paketmanagers DPKG,
		konnte auch die Unterstützung für RPM und Pacman implementiert werden.

	\item[Packaging und Deployment für SWID Generator] Um das Deployment des SWID
		Generators zu vereinfachen, haben wir diesen paketiert und im Python Package
		Index publiziert. Dies ermöglicht eine einfache Installation über ein
		einziges Kommando (Absatz \ref{swidgenerator:architektur:packaging}).

	\item[Rückfluss von Wissen in die Community] Durch die intensive Arbeit mit
		Libraries und Plugins konnten einige Fehler in diesen festgestellt werden.
		Einige davon konnten behoben werden und wurden als Pull Request in den
		Entwicklern der Erweiterungen zur Verfügung gestellt.

\end{description}


\subsection{Learnings}

Die intensive Arbeit an einem Projekt fordert, dass bestimmte Technologien im
Detail betrachtet werden und dass oft durchgeführte Prozesse optimiert werden.
Dabei lernt man Neues und bestätigt Altes. Im Folgenden sind die wesentlichsten
Learnings dieser Arbeit erwähnt. Sie beziehen sich auf technische oder
organisatorische Merkmale, individuelle persönliche Learnings sind den
\enquote{Persönlichen Berichten} (Abschnitt \ref{anhang:berichte}) zu entnehmen.

\begin{description}

	\item[Git: Branching und Reviews] Die Arbeit mit \enquote{Feature Branches} und
	konsequentem Review, bevor ein Branch in den Master-Branch einfliessen darf,
	hatte viele positive Effekte: Es konnten Fehler gefunden werden, bevor sie in
	die Produktion gelangten, zudem konnte das Wissen über einzelne Komponenten verteilt
	werden. Allerdings kostet dieses Vorgehen viel Zeit. Es ist schwierig
	zu sagen, ob sich der erhöhte Zeitaufwand gegenüber der eingesparten Zeit
	durch frühzeitiges Finden von Bugs auszahlt. Das Vorgehen hat sich jedoch
	in unserem Fall bewährt.
	
	\item[SQLite] An verschiedenen Punkten dieses Projektes war die Datenbank
	Bestandteil von Problemstellungen. Nicht selten konnte festgestellt werden,
	dass die technischen Einschränkungen von SQLite die Verursacher dieser Probleme
	waren. SQLite eignet sich gut für Prototyping, Embedded-Systeme oder kleine
	Einzelbenutzer-Systeme, verursacht bei grösseren Infrastrukturen mit
	Mehrbenutzerzugriff starke Probleme.
	
\end{description}
