\chapter{Glossar}

{
\renewcommand{\arraystretch}{1.5}
\begin{longtable}{ll}
\textbf{IMC} & Information Measurement Collector. Sammelt Daten auf dem CLient und sendet diese an den IMV zur Verifizierung.\\

\textbf{IMV} & Information Measurement Verifier. Empfängt Informationen vom IMC und verifiziert diese. Erzwingt Policies auf Serverseite. Erzwingt Policies auf Serverseite.\\

\textbf{Enforcement} & Erzwingt eine Policy auf einer Gruppe von Clients.\\

\textbf{Package} & Ein Software-Paket, das auf einem Client installiert ist. Kann auf eine schwarze Liste
gesetzt werden.\\

\textbf{Policy} & Eine Richtlinie, die ein Client einzuhalten hat, wenn er sich ins VPN einwählen will.\\

\textbf{Product} & Ein Produkt, bzw. Betriebssystem, das auf einem Client installiert ist.\\

\textbf{TNC} & Trusted Network Connect(-Server). Eine Architektur der Trusted Computing Group für
Network Access Control. Der dazugehörige Server ist im Falle von strongSwan die
Verwaltungsinstanz der IMVs.\\

\textbf{VPN }& Virtual Private Network. Eine verschlüsselte Verbindung, die dem Benutzer Zugang ins
lokale Netzwerk einer Firma oder Organisation über einen unsicheren Kanal (z.B. das
Internet) ermöglicht.\\
 
\textbf{Workitem} & Definiert einen Arbeitsauftrag. Wird von Cygnet für IMVs generiert und von diesen
ausgeführt. Die Resultate werden zurück an Cygnet geliefert und dann von Cygnet
ausgewertet.\\
 
\textbf{DBMS} & Datenbankmanagementsystem. Verwaltungssoftware für ein Datenbanksystem.\\

\textbf{REST} & Representational State Transfer. Programmierpradigma für die Implementation von zustandslosen Ressourcenorientierten Webschnittstellen.\\

\textbf{ORM} & Object-Relational-Mapping. Layer für die Speicherung von Objekten in einer Relationalen Datenbank.\\

\textbf{TNC} & Trusted Network Connect. Opensource Architekturframework für die Netzwerkzugangskontrolle.\\

\textbf{Software-ID} & Eindeutiger Identifier einer Softwarepaketes. Bestehend aus regid des Tag Creators und uniqueID des Softwarepaketes.\\

\textbf{SWID Tag} & Software Identification Tag. Beschreibung einer Softwarekomponente im XML Format. Standardisiert nach ISO 19770-2.\\

\textbf{TCG} & Trusted Computing Group. Standardisierungsorganisation, die Standards für das TNC Framework verfasst.\\
 
\textbf{TNC Client} & Gegenstück zum TNC Server. Erlaubt die Verbindungsaufnahme mit einem TNC Server.\\

\textbf{TLS} & Transport Layer Security. Hybrides Verschlüsselungsprotokoll zur Sicheren Datenübertragung via TCP.\\

\textbf{TNC Policy Manager} & Komponente aus dem TNC Framework, welche die Richtlinien (Policies) verwaltet.\\

\textbf{uniqueID} & Kennung, die eine Software innerhalbt des Namespaces des Tag Creators eindeutig identifiziert.\\

\textbf{Regid} & Unique registrastion ID. Identifizierer einer Organisation, Format: regid.YYYY-MM.<reverse domain name>. Das Datum ist das Registrierungsdatum der Domain.\\

\textbf{Model} & Abstraktion eines Datenbankobjektes.\\

\textbf{View} & Python Funktion oder Klasse, welche einen Request entgegen nimmt und eine Response zurückliefert.\\

\end{longtable}
}
