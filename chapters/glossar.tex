\chapter{Glossar}
\begin{tabular}{|l|l|}
\hline  IMC & Information Measurement Collector. Sammelt Daten auf dem CLient und sendet diese an den IMV zur Verifizierung.\\
\hline IMV & Information Measurement Verifier. Empfängt Informationen vom IMC und verifiziert diese. Erzwingt Policies auf Serverseite. Erzwingt Policies auf Serverseite.\\
\hline Enforcement & Erzwingt eine Policy auf einer Gruppe von Clients.\\
\hline Package & Ein Software-Paket, das auf einem Client installiert ist. Kann auf eine schwarze Liste
gesetzt werden.\\
Policy & Eine Richtlinie, die ein Client einzuhalten hat, wenn er sich ins VPN einwählen will.\\
\hline
Product & Ein Produkt, bzw. Betriebssystem, das auf einem Client installiert ist.\\
\hline
TNC-Server & Trusted Network Connect(-Server). Eine Architektur der Trusted Computing Group für
Network Access Control. Der dazugehörige Server ist im Falle von strongSwan die
Verwaltungsinstanz der IMVs.\\
VPN & Virtual Private Network. Eine verschlüsselte Verbindung, die dem Benutzer Zugang ins
lokale Netzwerk einer Firma oder Organisation über einen unsicheren Kanal (z.B. das
Internet) ermöglicht.\\
\hline Workitem & Definiert einen Arbeitsauftrag. Wird von Cygnet für IMVs generiert und von diesen
ausgeführt. Die Resultate werden zurück an Cygnet geliefert und dann von Cygnet
ausgewertet.\\
\hline 
DBMS & Datenbankmanagementsystem. Verwaltungssoftware für ein Datenbanksystem.\\
\hline
REST & Representational State Transfer. Programmierpradigma für die Implementation von zustandslosen Ressourcenorientierten Webschnittstellen.\\
\hline
ORM & Object-Relational-Mapping. Layer für die Speicherung von Objekten in einer Relationalen Datenbank.\\
\hline
TNC & Trusted Network Connect. Opensource Architekturframework für die Netzwerkzugangskontrolle.\\
\hline
Software-ID & Eindeutiger Identifier einer Softwarepaketes. Bestehend aus regid des Tag Creators und uniqueID des Softwarepaketes.\\
\hline
SWID Tag & Software Identification Tag. Beschreibung einer Softwarekomponente im XML Format. Standardisiert nach ISO 19770-2.\\
\hline
TCG & Trusted Computing Group. Standardisierungsorganisation, die Standards für das TNC Framework verfasst.\\
\hline 
TNC Client & Gegenstück zum TNC Server. Erlaubt die Verbindungsaufnahme mit einem TNC Server.\\
\hline
TLS & Transport Layer Security. Hybrides Verschlüsselungsprotokoll zur Sicheren Datenübertragung via TCP.\\
\hline
TNC Policy Manager & Komponente aus dem TNC Framework, welche die Richtlinien (Policies) verwaltet.\\
\hline
uniqueID & Kennung, die eine Software innerhalbt des Namespaces des Tag Creators eindeutig identifiziert.\\
\hline
Regid & Unique registrastion ID. Identifizierer einer Organisation, Format: regid.YYYY-MM.<reverse domain name>. Das Datum ist das Registrierungsdatum der Domain.\\
\hline
Model & Abstraktion eines Datenbankobjektes.\\
\hline
View & Python Funktion oder Klasse, welche einen Request entgegen nimmt und eine Response zurückliefert.\\
\hline


\end{tabular} 