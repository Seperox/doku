\chapter{Glossar}

{
\renewcommand{\arraystretch}{1.5}
\begin{longtable}{l >{\raggedright}p{.75\textwidth}}
\textbf{DBMS} & Datenbankmanagementsystem. Verwaltungssoftware für ein Datenbanksystem.\tabularnewline
\textbf{DTO} & Data Transfer Object, eine Klasse, die kein Eigenverhalten aufweist, sondern lediglich Daten in einem Objekt bündelt.\tabularnewline
\textbf{Enforcement} & Erzwingt eine Policy auf einer Gruppe von Clients.\tabularnewline
\textbf{IMC} & Information Measurement Collector. Sammelt Daten auf dem CLient und sendet diese an den IMV zur Verifizierung.\tabularnewline
\textbf{IMV} & Information Measurement Verifier. Empfängt Informationen vom IMC und verifiziert diese. Erzwingt Policies auf Serverseite. Erzwingt Policies auf Serverseite.\tabularnewline
\textbf{Model} & Abstraktion eines Datenbankobjektes.\tabularnewline
\textbf{ORM} & Object-Relational-Mapping. Layer für die Speicherung von Objekten in einer Relationalen Datenbank.\tabularnewline
\textbf{Package} & Ein Software-Paket, das auf einem Client installiert ist. Kann auf eine schwarze Liste gesetzt werden.\tabularnewline
\textbf{Policy} & Eine Richtlinie, die ein Client einzuhalten hat, wenn er sich ins VPN einwählen will.\tabularnewline
\textbf{Product} & Ein Produkt, bzw. Betriebssystem, das auf einem Client installiert ist.\tabularnewline
\textbf{Regid} & Unique registrastion ID. Identifizierer einer Organisation, Format: regid.YYYY-MM.<reverse domain name>. Das Datum ist das Registrierungsdatum der Domain.\tabularnewline
\textbf{REST} & Representational State Transfer. Programmierpradigma für die Implementation von zustandslosen Ressourcenorientierten Webschnittstellen.\tabularnewline
\textbf{Software-ID} & Eindeutiger Identifier einer Softwarepaketes. Bestehend aus regid des Tag Creators und uniqueID des Softwarepaketes.\tabularnewline
\textbf{SWID Tag} & Software Identification Tag. Beschreibung einer Softwarekomponente im XML Format. Standardisiert nach ISO 19770-2.\tabularnewline
\textbf{TCG} & Trusted Computing Group. Standardisierungsorganisation, die Standards für das TNC Framework verfasst.\tabularnewline
\textbf{TLS} & Transport Layer Security. Hybrides Verschlüsselungsprotokoll zur Sicheren Datenübertragung via TCP.\tabularnewline
\textbf{TNC Client} & Gegenstück zum TNC Server. Erlaubt die Verbindungsaufnahme mit einem TNC Server.\tabularnewline
\textbf{TNC Policy Manager} & Komponente aus dem TNC Framework, welche die Richtlinien (Policies) verwaltet.\tabularnewline
\textbf{TNC} & Trusted Network Connect. Opensource Architekturframework für die Netzwerkzugangskontrolle.\tabularnewline
\textbf{TNC} & Trusted Network Connect(-Server). Eine Architektur der Trusted Computing Group für Network Access Control. Der dazugehörige Server ist im Falle von strongSwan die Verwaltungsinstanz der IMVs.\tabularnewline
\textbf{uniqueID} & Kennung, die eine Software innerhalbt des Namespaces des Tag Creators eindeutig identifiziert.\tabularnewline
\textbf{View} & Python Funktion oder Klasse, welche einen Request entgegen nimmt und eine Response zurückliefert.\tabularnewline
\textbf{VPN} & Virtual Private Network. Eine verschlüsselte Verbindung, die dem Benutzer Zugang ins lokale Netzwerk einer Firma oder Organisation über einen unsicheren Kanal (z.B. das Internet) ermöglicht.\tabularnewline
\textbf{Workitem} & Definiert einen Arbeitsauftrag. Wird von Cygnet für IMVs generiert und von diesen ausgeführt. Die Resultate werden zurück an Cygnet geliefert und dann von Cygnet ausgewertet.
\end{longtable}
}
