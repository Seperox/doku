\chapter{Ausblick}

\section{Offene Punkte} Während der gesamten Projektarbeit wurden Bugs,
weiterführenden Arbeiten und Ideen im Github
Issuetracker\footnote{\url{https://github.com/tnc-ba/strongTNC/issues}} erfasst,
ungeachtet davon, ob sie im Rahmen dieser Arbeit bearbeitet werden konnten.
Diese Issues können noch bearbeitet werden, eventuell in einer Nachfolgearbeit
oder durch die Community.\\
Es gibt noch offene Issues, welche die Performance
hinsichtlich Laufdauer einzelner Abfragen betreffen. Konkret scheint der
Algorithmus für die Erstellung der SWID Inventory View nicht optimal zu sein.
Die langen Laufzeiten sind durch den hohen Memory Footprint der Abfragen
bedingt. Die dadurch entstehenden Memoryallozierungen und allfällige
Pagingvorgänge verzögern die Ausführung erheblich. Das Problem könnte durch
verzögertes laden einzelner Felder (vor allem SWID XML String) verbessert
werden. Aus zeitlichen Gründen konnte die Optimierung nicht vorgenommen
werden.\\ Eine wesentliche Verbesserung betrifft die Log View. Durch einbauen
der Filterfunktion wäre es möglich, schneller festzustellen wann ein Paket auf
einem Gerät installiert und wieder entfernt wurde. Dies könnte man auch bereits
in der SWID Tag Detail unterstützen. Anhand einer zusätzlichen Spalte in der
\enquote{Reported by Devices} Tabelle, könnte angezeigt werden, wann der Tag auf
dem Gerät das letzte Mal als installiert gemeldet wurde.

\section{Empfehlungen} \begin{description} \item[REST API vollständig umsetzen]
Die vollständige Trennung von strongTNC und strongSwan sollte auf jeden Fall
umgesetzt werden. \item[Wechsel von SQLite zu MySQL] Durch die Erweiterungen in
dieser Arbeit wurde die Applikation gerüstet mit grossen Datenmengen umgehen zu
können. Ein Aspekt der dies zur Zeit noch verhindern könnte, ist die Verwendung
von SQLite. Daher empfehlen wir die nötigen Schritte vorzunehmen, um MySQL als
Datenbankbackend nutzen zu können. Es wurden Workarounds implementiert, die
lediglich der SQLite Kompatibilität dienen. Diese sollten zu diesem Zeitpunkt
entfernt werden. \item[Class Based Views] Die Verwendung von Class Based Views
sollte konsequenterweise für alle bestehenden Views eingesetzt werden. Bisher
folgen nur die Views der SWID Erweiterungen diesem Konzept. \item[Django Form
Framework] Die Formularvalidierung wird derzeit noch aufwändig \enquote{von
Hand} gemacht. Das Django bietet mit der Form Framework Erweiterung ein
konfortables Hilfsmittel, mit der diese aufwändige Arbeit elegant erledigt
werden kann. Die Komplexität der Views könnte erheblich reduziert werden.
\item[] \end{description}