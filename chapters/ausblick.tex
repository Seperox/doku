\chapter{Ausblick}

\section{Offene Punkte} 
Während der gesamten Projektarbeit wurden Bugs, weiterführende Arbeiten und
Ideen im Github
Issuetracker\footnote{\url{https://github.com/tnc-ba/strongTNC/issues}} erfasst,
ungeachtet davon, ob sie im Rahmen dieser Arbeit bearbeitet werden konnten.
Diese Issues können noch bearbeitet werden, eventuell in einer Nachfolgearbeit
oder durch die Community.\\
Ein Verbesserungsvorschlag betrifft die SWID Log View. Durch Einbauen
einer Filterfunktion wäre es möglich, schneller festzustellen, wann ein Paket auf
einem Gerät installiert beziehungsweise entfernt wurde. Dies könnte man auch
bereits in der SWID Tag Detail unterstützen. Anhand einer zusätzlichen Spalte in
der \enquote{Reported by Devices} Tabelle könnte angezeigt werden, wann der Tag
auf dem Gerät das letzte Mal als installiert gemeldet wurde.


\section{Empfehlungen} 

\begin{description} 

	\item[REST API vollständig umsetzen] Die vollständige Trennung von strongTNC
	und strongSwan sollte auf jeden Fall umgesetzt werden.
	
	\item[Wechsel von SQLite zu MySQL] Durch die Erweiterungen in dieser Arbeit
	wurde die Applikation darauf vorbereitet, mit grossen Datenmengen umgehen zu
	können. Ein Aspekt der dies zur Zeit teilweise noch verhindert, ist die
	Verwendung von SQLite. Daher empfehlen wir die nötigen Schritte vorzunehmen, um
	MySQL als Datenbankbackend nutzen zu können. Es wurden teilweise Workarounds
	implementiert, die lediglich der SQLite Kompatibilität dienen. Diese sollten zu
	diesem Zeitpunkt entfernt werden.
	
	\item[Class Based Views] Die Verwendung von Class Based Views sollte
	konsequenterweise für alle bestehenden Views eingesetzt werden. Bisher folgen
	nur die Views der SWID Erweiterungen diesem Konzept.
	
	\item[Django Form Framework] Die Formularvalidierung wird derzeit noch manuell
	per Javascript gemacht. Django bietet mit der \enquote{Form Framework}
	Erweiterung ein komfortables Hilfsmittel, mit der diese aufwändige Arbeit
	elegant erledigt werden kann. Die Komplexität der Views dadurch könnte
	erheblich reduziert werden.
	
	\item[Sicherheit API] Falls sich der HTTP Signatures Draft zu einem Standard durchsetzt empfehlen wir ihn für die REST API zu implementieren. Damit könnte die
	Authentisierung und Datenintegrität der Kommunikation sichergestellt werden.
	Zudem sollte empfehlen wir stets TLS einzusetzen, da das HTTP Protokoll verschiedene
	Verletzlichkeiten\cite{httpsecconsiderations2014} aufweist, die durch TLS
	behoben werden.
	
	\item[Sicherheit Frontend]
	Wir empfehlen auch für das Frontend immer TLS einzusetzen. Es wurde eine detaillierte Deployment Dokumentation für das Setup von Apache mit TLS erstellt.
	
\end{description}