\chapter{Ausblick}

\section{Offene Issues} 

Während der Projektarbeit wurden Bugs, weiterführende Arbeiten und Ideen im
Github Issuetracker\footnote{\url{https://github.com/tnc-ba/strongTNC/issues}}
erfasst, ungeachtet davon, ob sie im Rahmen dieser Arbeit bearbeitet werden
konnten oder nicht.  Diese Issues können noch bearbeitet werden, eventuell in
einer Nachfolgearbeit oder durch die Community.


\nobrsection{Empfehlungen} 

\begin{description} 

	\item[REST API vollständig umsetzen] Die vollständige Trennung von strongTNC
	und strongSwan sollte auf jeden Fall umgesetzt werden.
	
	\item[Wechsel von SQLite zu MySQL] Durch die Anpassungen an strongTNC
	wurde die Applikation darauf vorbereitet, mit grossen Datenmengen umgehen zu
	können. Ein Aspekt der dies zur Zeit teilweise noch verhindert, ist die
	Verwendung von SQLite. Daher empfehlen wir die nötigen Schritte vorzunehmen, um
	MySQL als Datenbankbackend nutzen zu können. Es wurden teilweise Workarounds
	implementiert, die lediglich der SQLite Kompatibilität dienen. Diese könnten
	entfernt werden, sobald die Unterstützung von SQLite eingestellt wird.

	\item[Upgrade auf Django 1.7] Da Django
	1.7\footnote{\url{https://docs.djangoproject.com/en/dev/releases/1.7/}} erst
	im Sommer 2014 veröffentlicht wird, konnten wir ein Upgrade auf diese
	Version noch nicht durchführen. Django bringt mit der Version 1.7 viele neue
	Features mit, unter anderem ein eigenes Datenbank-Migrations-Framework. Es
	wäre sinnvoll, Django auf Version 1.7 zu aktualisieren, sobald die Trennung
	der Datenbank umgesetzt wurde. Die Datenbank kann dann gleich mit dem
	Migrations-Framework initialisiert und verwaltet werden.
	
	\item[Class Based Views] Die Verwendung von Class Based Views sollte
	konsequenterweise für alle bestehenden Views eingesetzt werden. Bisher folgen
	nur die Views der SWID Erweiterungen diesem Konzept.
	
	\item[Django Forms Framework] Die Formularvalidierung wird derzeit noch
		manuell im Frontend per Javascript und im Backend in den jeweiligen Views
		gemacht.  Django bietet jedoch mit dem \enquote{Forms Framework} ein
		komfortables Hilfsmittel, mit welchem diese aufwändige Arbeit elegant
		erledigt werden kann. Die Komplexität der Views könnte dadurch erheblich
		reduziert werden.
	
	\item[API Sicherheit] Falls sich der HTTP Signatures Draft zu einem
		offiziellen Standard entwickeln sollte, empfehlen wir, diesen in der REST API
		einzusetzen. Damit könnte die Authentisierung und Datenintegrität der
		Kommunikation besser sichergestellt werden, als dies mit Basic Authentication
		möglich ist.  Zudem sollte auch mit HTTP Signatures stets TLS eingesetzt
		werden, da das HTTP Protokoll verschiedene Verletzlichkeiten aufweist, die
		durch TLS behoben werden\cite{httpsecconsiderations2014}.
	
	\item[Frontend Sicherheit] Wir empfehlen auch für das Frontend den
		universellen Einsatz von TLS. Als Hilfestellung haben wir eine detaillierte
		Deployment-Dokumentation für die Konfiguration von Apache mit
		TLS-Verschlüsselung erstellt (Abschnitt \ref{anhang:deployment-manual}).
	
\end{description}
