\section{Requirements - SWID Generator}
\subsection{Nicht-Funktional}
\begin{itemize}
\item Als Implementationsprache wird Python verwendet
\item Es sollen möglichst wenig Abhängigkeiten zu 3. Komponenten wie Libraries/Framworks entstehen. Wo möglich sollen Komponenten aus dem Standardumfang von Python verwendet werden.
\item Die Software soll einfach zu installieren sein, bspw. durch integration in pip
\item Als Quelle der Paketinformationen sollen yum und dpkg verwendet werden
\end{itemize}
Nachfolgend sind sind die identifizierten Use Cases für die Client Komponenten des SWID-Generator aufgeführt.
\subsection*{UC01: Standard SWID Generierung}
Die IMC Komponente kann via stdout des SWID-Generators XML Dokumente erhalten, welche Informationen über die zur Zeit installierte Software des Zielsystems beinhalten.
Das Format dieser XML Dokumente folgt dem ISO Draft 19770-2-5
Pro Installiertes Packet wird ein separates ein XMl Dokument generiert.
Diese Dokumente bestehen im wesentlichen aus einem SoftwareEntity und einem Entity Tag mit entsprechenden Attributen TODO Beispiel

\paragraph{Standard Szenario}
Die Attribute werden mit vordefinierten Standardwerten befüllt.
TODO Beispiel
\paragraph{Alternatives Szenario}
Die Attribute sollen mittels optionalem Parametern spezifiziert werden können.

\subsection*{UC02: SWID Tag mit File Payload}
Mittels Optionalem Parameter können die XML Dokumente mit einem Payload Tag versehen werden, welcher für jedes Packet die darin enthaltenen Dateien auflistet.
TODO Beispiel

\subsection*{UC03: Pretty Printing}
Der benutzer möchte mittels optionalem Parameter zu analysezwecken die XML Dokumente in Menschenlesebarer Form ansehen. Diese Ausgabe der Tags ist hierarchisch eingerückt um so die Menschenlesebarkeit sicherzustellen.

\subsection*{UC04: Automatische Erkennung der Umgebung/Packetmanagaer }
Beim Aufruf des Swid-Generators soll dieser versuchen die Der swid Generator soll den verwendeten Packagemanager selber erkennen.
\paragraph{Alternatives Szenario}
Mittels optionalem Parameter kann der zu verwendeten Packetmanager angegeben und die Autoerkennung übersteuert werden.


