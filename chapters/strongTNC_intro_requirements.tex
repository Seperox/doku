\chapter{strongTNC}


\section{Requirements}
\subsection{Nichtfunktionale Anforderungen}
Da es sich bei diesem Projekt um eine aufbauende Arbeit handelt, werden keine
neuen nichtfunktionalen Anforderungen definiert. Es sollen die Anforderungen der
Vorgängerarbeit\cite{cygnet:2013} eingehalten werden.


\subsection{Use Cases}
\label{strongTCN:usecases}

Nachfolgend werden die identifizierten Use Cases beschrieben. Als
\enquote{System under Decision} wird die strongTNC Django Webapp als Ganzes
betrachtet.

\subsubsection{UC01: CMDB, Softwareinventar eines Gerätes}
\label{strongTNC:UC01}
\begin{usecase}
\hline
\textbf{Akteur} & strongTNC Benutzer \\
\hline
\textbf{Story} &
Der Akteur will feststellen, welche Software in welcher Version zu einem
bestimmten Zeitpunkt auf einem Gerät installiert war, beispielsweise weil in
einer Version eine kritische Sicherheitslücke enthalten ist.\\
\hline
\textbf{Standard Szenario} &
Der Akteur betrachtet das SWID Tag Inventar des gewünschten Gerätes und kann
dort zu jedem Mess-Zeitpunkt die installierte Software abrufen. \\
\hline
\textbf{Alternatives Szenario} &
Der Akteur wählt den gewünschten SWID Tag
in der SWID Liste und die Geräte, welche die Software installiert hatten, werden
aufgelistet.\\
\hline
\end{usecase}

\subsubsection{UC02: Detaillierte SWID-Tag File Information}
\begin{usecase}
\hline
\textbf{Akteur} & strongTNC Benutzer \\
\hline
\textbf{Story} &
Der Akteur will feststellen, welche Dateien zu einem bestimmten Softwarepaket gehören. \\
\hline
\textbf{Standard Szenario} &
Der Akteur findet den SWID Tag, der zum gesuchten Softwarepaket gehört in der
SWID Tag Ansicht. Die Dateien, die mit dem SWID Tag gemessen wurden, sind
aufgelistet. \\
\hline
\textbf{Alternatives Szenario} & 
Der Akteur findet das Softwarepaket in der Paketübersicht. Dort kann auf einen
verknüpfen SWID Tag zugegriffen werden. So gelangt der Akteur auf die SWID Tag
Ansicht und findet dort die Liste der gemessenen Dateien. \\
\hline
\end{usecase}

\subsubsection{UC03: Verfügbare Entities auflisten}
\label{strongTNC:UC03}
\begin{usecase}
\hline
\textbf{Akteur} & strongTNC Benutzer \\
\hline
\textbf{Story} &
Der Akteur will alle im System erfassten Entites auflisten und herausfinden, 
welche SWID Tags zu der jeweiligen Entity gehören. \\
\hline
\textbf{Standard Szenario} &
Der Akteur kann sich in der Entity Ansicht alle erfassten Entities auflisten
lassen. Wenn eine Entity angewählt wird, werden alle SWID Tags, die der Entity
zugeordnet sind, aufgelistet. \\
\hline
\phantom{\textbf{Alternatives Szenario}} & 
\end{usecase}

\subsubsection{UC04: Verfügbare SWID Tags auflisten}
\label{strongTNC:UC04}
\begin{usecase}
\hline
\textbf{Akteur} & strongTNC Benutzer \\
\hline
\textbf{Story} &
Der Akteur will alle im System erfassten SWID Tags auflisten. Dabei ist er an
folgenden Informationen eines Tags interessiert:
\begin{itemize}
\item Software-ID
\item Unique-ID
\item Tag Creator Entity
\item RAW XML String
\item Welche Files sind in diesem SWID Tag enthalten
\item Gibt es ein bestehendes Paket dessen Name dem des SWID Tags entspricht
\item Welche Devices haben diesen SWID Tag installiert gemeldet (Abschnitt \ref{strongTNC:UC01})
Case UC01)
\end{itemize}\\
\hline
\textbf{Standard Szenario} &
Der Akteur kann sich in der SWID Tag Ansicht alle erfassten SWID Tags auflisten
lassen. Wenn ein SWID Tag angewählt wird, sind alle gewünschten Details
aufgelistet. \\
\hline
\phantom{\textbf{Alternatives Szenario}} &
\end{usecase}

\subsubsection{UC05: Erfassen eines neuen SWID Tags}
\begin{usecase}
\hline
\textbf{Akteur} & strongSwan TNC Server oder Administrator\\
\hline
\textbf{Story} &
Der Akteur will einen neuen SWID Tag im System erfassen. Der Akteur liefert ein
XML Dokument gemäss SWID Spezifikation\cite{iso19770-2} an das System. Das
System nimmt das XML Dokument entgegen und speichert die notwendigen
Informationen ab.\\
\hline
\textbf{Standard Szenario} &
Der Akteur liefert ein XML Dokument an das System und erhält eine Erfolgsmeldung
die bestätigt, dass der Tag gespeichert wurde.\\
\hline
\textbf{Alternatives Szenario 1} & 
Eine Entity welche im gelieferten XML enthalten ist existiert bereits im System
(identifiziert durch die regid) besitzt aber einen anderen Namen. Der Name der
bestehenden Entity wird entsprechend aktualisiert.\\
\hline
\textbf{Alternatives Szenario 2} & 
Der Tag welcher erfasst werden sollte existiert bereits im System, identifiziert durch die Software-ID.
\begin{description}
\item[2a] Die Erfassung eines neuen SWID Tags erfolgt programmatisch über die
TNC Server Komponente. Ein Tag mit derselben Unique-ID existiert bereits und
wird nicht ersetzt. \item[2b] Die Erfassung eines neuen SWID Tags erfolgt
manuell durch den Administrator. Ein Tag mit derselben Unique-ID existiert
bereits und wird mit den neuen Daten ersetzt.
\end{description}
\\
\hline
\textbf{Alternatives Szenario 3} & 
Es wurde ein invalides XML Dokument ans System geliefert. Das System informiert
den Benutzer über den Fehler und erstellt keinen neuen Tag.\\
\hline
\textbf{Häufigkeit des Auftretens} &
Bei Systemeinführung wird diese Operation häufig auftreten. Idealerweise sieht
das Deployment vor, dass für jeden Produktetyp ein Referenzgerät vorhanden ist,
mit welchem initial alle Tags generiert und erfasst werden und danach in
regelmässigen Abständen aktualisiert werden. Dadurch ist die Anzahl der nötigen
Erfassungs-Operationen durch User Endgeräte klein zu halten.\\
\hline
\textbf{Datenvariation} &
Es soll möglich sein, eine Liste mit mehreren XML Dokumenten auf einmal ans System
auszuliefern (Batch-Create). Das Erfassen soll in diesem Fall so erfolgen, 
dass entweder alle übermittelten SWID Tags oder keine gespeichert werden.
Dadurch kann die Anzahl nötiger Systemaufrufe erheblich reduziert werden.\\
\hline
\end{usecase}

\subsubsection{UC06: Durchführen einer SWID Messung für ein Gerät}
\label{strongTNC:UC06}
\begin{usecase}
\hline
\textbf{Akteur} & strongSwan TNC Server \\
\hline
\textbf{Story} &
Der Akteur will alle auf einem Device vorhandenen SWID Tags mit einer Session
verknüpfen\\
\hline
\textbf{Standard Szenario} &
Der Akteur sendet eine Liste der Software-IDs an das System, die er mit der
aktuellen Session verknüpfen möchte. Der Server assoziiert die entsprechenden
Tags mit der Session und bestätigt den erfolgreichen Vorgang.\\
\hline
\textbf{Alternatives Szenario} & 
Es existieren nicht für alle gelieferten Software-IDs Tags im System. Das
System assoziiert keine Tags mit der aktuellen Session und gibt eine Liste
derjenigen Software-IDs zurück, für welche keine Tags vorhanden sind. Der Akteur
erfasst daraufhin alle fehlenden Tags im System. Sobald alle Tags im System
erfasst sind tritt das Standard Szenario ein.
\end{usecase}

\subsubsection{UC07: Auflisten aller Tags zu einem File}
\begin{usecase}
\hline
\textbf{Akteur} & strongTNC Benutzer \\
\hline
\textbf{Story} &
Der Akteur will wissen, in welchen Tags ein bestimmtes File enthalten ist.\\
\hline
\textbf{Standard Szenario} &
Der Akteur kann sich in der File Ansicht alle erfassten Files auflisten lassen.
Wenn ein File angewählt wird, sind alle zugeordneten SWID Tags aufgelistet. \\
\hline
\phantom{\textbf{Alternatives Szenario}} & 
\end{usecase}
