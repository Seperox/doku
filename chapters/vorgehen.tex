\chapter{Vorgehen}

Für die Durchführung dieses Projekts haben wir uns für eine
iterative Vorgehensweise entschieden. Für ein Team von drei Personen und einer
Projektdauer von rund 17 Wochen haben wir uns für ein Subset des Scrum Vorgehensmodelles
entschieden, dabei waren die drei Säulen von Scrum von zentraler Bedeutung:

\begin{description}
	\item[Transparenz] Durch die Verwendung von Github zur Sourcecodeverwaltung und
	auch für das Issuetracking, ist der aktuelle Fortschritt und allfällige
	Hindernisse im Verlauf des Projektes stets sichtbar, öffentlich und
	nachvollziehbar festgehalten.

	\item[Überprüfung] Das Resultat dieser Arbeit wird zeitnah nach ihrem Abschluss
	an einer internationalen IT Security Konferenz präsentiert. Um zu diesem
	Zeitpunkt einen stabilen Stand zu erreichen, haben wir uns entschieden, regelmässig den aktuellen Stand unserer Arbeiten mittels Github Pull Requests
	in das offizielle strongTNC Repository zu integrieren. Dadurch wird
	sichergestellt, dass die Funktionalität von den Betreuern stets überprüft und
	getestet werden kann und die Feedback Roundtrips minmal gehalten werden.

	\item[Anpassung] Da das Produkt bereits im Einsatz ist, werden bei der
	Bedienung Probleme und neue Anforderungen entdeckt. Dadurch, dass die
	Anforderungen an das Produkt nicht ein für alle Mal festgelegt, sondern laufend
	neu bewertet und angepasst werden, entsteht ein Produkt, welches den Ansprüchen
	der Benutzer besser entspricht.

\end{description}

\section{Arbeitsweise und Hilfsmittel}
Wir haben uns für folgende Artefakte\cite{alliance2008scrum} entschieden:

\begin{description}
	\item[Definition of Done] Wir haben eine Checkliste von Aktivitäten	definiert,
	welche festhält, wie einzelne Aufgaben einheitlich gelöst werden. Damit wird
	einerseits das Schätzen des Aufwandes bei der Iterationsplanung erleichtert und
	andererseits festgelegt, wann ein Task als erledigt betrachtet werden kann.
	Auch Qualitätsmassnahmen, wie beispielsweise das Testen und Durchführen von
	Reviews sind darin enthalten (\autoref{DefinitionOfDone}).

	\item[Scrumboard] Sämtliche Tasks werden zur verbesserten Visualisierung auf
	einem klassischen Scrumboard mit Post-It Zetteln gepflegt, zusätzlich werden
	alle Task auch im Github Issue Tracker erfasst, um die Nachvollziehbarkeit
	zu gewährleisten.
	
	\item[Backlog] Sämtliche geplanten Features und offenen Bugs werden in einem
	Backlog gehalten, so ist der Umfang der offenen Arbeit klar ersichtlich.
	
	\item[Iteration Backlog] Ebenfalls wird für jede Iteration ein Backlog
	geführt.
	
	\item[Planning Meeting] Zu Beginn jeder Iteration wird ein Planning Meeting
	durchgeführt, indem die Tasks für die kommende Iteration mit Hilfe von
	\enquote{Planning Poker} eingeplant und geschätzt werden.
		
	\item[Retrospektive] Vor jedem Planning Meeting wird eine kurze Retrospektive
	zur vergangenen Iteration durchgeführt. Die Arbeitszeiten und die Schätzungen,
	werden durchgesehen und es wird beurteilt, wie genau die Planung der letzten
	Iteration war und was für die kommende Iteration verbessert werden muss.
	
\end{description}

Bei der Einarbeitung in die vorhandene strongTNC Umgebung haben wir
festgestellt, dass die Architektur und die Codequalität einige Mängel aufweist.
Deshalb haben wir uns entschieden, diesem Aspekt genauer zu betrachten
und haben entsprechende Massnahmen definiert. Diese sollen die Qualität dieser
Arbeit, als auch jene von möglichen Nachfolgearbeiten oder Weiterentwicklungen
durch die Community sicherstellen.

\begin{description}
	\item[Continous Integration] Zur fortlaufenden Integration des entwickelten
	Codes wird Travis CI\footnote{\url{http://www.travis-ci.org}} eingesetzt.
	Dadurch wird sichergestellt, dass sich stets lauffähiger Code im Master Branch
	des Repositories befindet oder fehlerhafter Code schnell entdeckt wird. Als
	zusätzliches Hilfsmittel wurde auf der Webseite des Github Repositories ein
	Statusindikator integriert, welcher optisch über den aktuellen Buildstand
	informiert.

	\item[Statische Code-Analyse] Durch die Integration von
	Landscape.io\footnote{\url{http://landscape.io}}, einem online Service, welcher
	statische Code-Analysen durchführt und die Codequalität bewertet, kann
	sichergestellt werden, dass Code-Smells, welche nicht durch Unit Tests oder
	während der Entwicklung erkannt werden, trotzdem bemerkt werden. Um immer über
	den Stand des Codes informiert zu sein, wurde auch für Landscape.io auf Github
	ein optischer Statusindikator integriert.

	\item[Coding-Styleguide] Es wurde ein Coding-Styleguide
	(\autoref{anhang:coding-styleguide}) definiert, welcher zusammengefasst folgende
	Punkte definiert:
	\begin{itemize*}
		\item Anwendung und Format von Docstrings zur Inline-Dokumentation von Code
		\item Code-Style, PEP8 
		\item Future Imports für Vorwärtskompatibilität zu Python 3 
		\item PEP8 Style Checking Konfiguration 
		\item Pytest Testing Framework
	Konfiguration \end{itemize*}

	\item[Git Guidelines] Zum einheitlichen Umgang mit Git wurden Guidelines
	verfasst (\autoref{ref:git-guidelines}), welche folgende Punkte
	beschreiben
	\begin{itemize*}
		\item Aussehen von Commit Messages (Sprache, Format, Referenzen)
		\item Wann und wie \enquote{History Rewriting} vorgenommen werden soll
		\item Wie Pull Requests erstellt werden sollen
		\item Wie Merging in den Master abläuft
	\end{itemize*}

\end{description}
