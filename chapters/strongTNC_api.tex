\chapter{API Konzept}
\section{Trennung, API Konzept}
In diesem Abschnitt werden wichtige Punkte des API Konzeptes genauer betrachtet.
Das vollständige Konzept ist im Anhang (<TODO: Ref>) zu finden. Detaillierte
Beschreibungen zu der in den Beispielen verwendeten Syntax sowie die
Definition der Archetypen sind ebenfalls im Konzept zu finden.

\section{Vorgehen}
Das Ziel der REST API war eine vollständige Ablösung der gemeinsamen Datenbank
als Schnittstelle. Um einen Überblick über den benötigten Umfang einer TNC
Policy Manager Schnittstelle zu bekommen, wurden die Datenbankzugriffe der IMV
Code der strongSwan Implementation betrachtet. Dazu wurden sämtliche SQL
abfragen im IMV Quellcode gesammelt und analysiert. Das Ergebnis war eine Liste
der Tabellen, auf die schreibend oder lesend zugegriffen wurde, ebenso konnte
ermittelt werden, welche Tabelle durch einen \texttt{JOIN} verknüpft wurden.
Basierend auf diesen Daten konnte ein Set von REST Endpunkten zusammengestellt
werden, mit dem es möglich ist, sämtliche Datenbankzugriffe abzudecken.

\section{Documents und Collections}
Für alle Elemente der strongTNC Domäne wurden REST Ressourcen von den Archetype
\enquote{Document} und \enquote{Collection} konzipiert. Oft handelt es sich
dabei um vollständige CRUD Ressourcen, wie beispielsweise bei der
\texttt{Version} Ressource sichtbar ist:

\begin{mdframed}[style=def]
\begin{description*}
	\item[URI Path] \texttt{/versions/\{id\}/}
	\item[Archetype] Document
	\item[Methods] GET, PUT, PATCH, DELETE
	\item[JSON Format Response] \hfill
\begin{jsoncode}
{
	"id": 5,
	"uri": "https://strongtnc/api/versions/5/",
	"package": "https://strongtnc/api/packages/42/",
	"product": "https://strongtnc/api/products/23/",
	"release": "5.0.2-2.2+squeeze1",
	"securtiy": true,
	"blacklist": false,
	"time": 1402061820
}
\end{jsoncode}
\end{description*}
\end{mdframed}

\begin{mdframed}[style=def]
\begin{description*}
	\item[URI Path] \texttt{/versions/}
	\item[Archetype] Collection
	\item[Filter Query] \hfill
	\begin{description*}
		\item[productName] \texttt{<str,product-name>}
		\item[packageName] \texttt{<str,package-name>}
	\end{description*}
	\item[Methods] GET, POST
	\item[Response] List of Version documents
\end{description*}
\end{mdframed}

Die angebotenen Operationen, beziehungsweise HTTP Methoden, wurden immer so weit
wie möglich eingeschränkt, so gibt es diverse Ressourcen die kein
\texttt{DELETE} anbieten oder nur lesenden Zugriff erlauben. Dadurch soll
erreicht werden, dass die Schnittstelle nicht zu offen wird oder durch das
Anbieten von nicht benötigten Operationen unnötig komplex erscheint.

Um den \texttt{JOIN} Verknüpfungen in einer ressourcenorientierten Weise
Rechnung zu tragen, wurden verschiedene virtuelle Ressourcen geschaffen.
Beispielsweise lassen sich die \enquote{Sessions} eines Gerätes direkt auf der
\texttt{devices} Ressource abfragen:

\begin{mdframed}[style=def]
\begin{description*}
	\item[URI Path] \texttt{/device/\{id\}/sessions/}
	\item[Archetype] Readonly Collection
	\item[Filter Query] \hfill
	\begin{description*}
		\item[timeFrom] \texttt{<int,timestamp>}
		\item[timeTo] \texttt{<int,timestamp>}
	\end{description*}	
	\item[Methods] GET
	\item[Response] List of Session documetns
\end{description*}
\end{mdframed}

\section{Controller}
Komplexere Abläufe, welche bei der gegenwärtigen Schnittstelle direkt mit
mehreren Datenbankzugriffen ablaufen, werden im REST Konzept durch
\enquote{Controller} abgebildet. Zu diesen Abläufen gehören das Starten und
Stoppen einer \enquote{Session}. Ebenfalls wurde die bisher noch nicht
existierende Funktion der SWID Tag Messung, beschrieben in Abschnitt
\ref{swiderweiterung}, als \enquote{Controller} umgesetzt.

\subsection{Session Steuerung und Ablauf}
Das grundsätzliche Konzept der Session Steuerung wurde gegenüber der jetzigen
Schnittstelle nicht geändert, es werden weiterhin sogenannte \enquote{Workitems}
verwendet, um anzuzeigen, welche Messungen durchgeführt werden müssen. Der Ablauf
einer Session nach dem gegenwärtigen Verfahren kann der 
\autoref{masurement-diagramm} entnommen werden. Ausführliche
Informationen dazu sind in der Vorgängerarbeit
\enquote{Cygnet}\cite{cygnet:2013} zu finden.

Folgendes ist der Aufbau der Endpunkte für den Sessionstart und das Beenden der
Session:

\begin{mdframed}[style=def]
\begin{description*}
	\item[URI Path] \texttt{/sessions/start/}
	\item[Archetype] Controller
	\item[Methods] POST
	\item[Request Parameter] \hfill
	\begin{description*}
		\item[\texttt{connectionId}] strongSwan Connection Id
		\item[\texttt{clientIdentity}] strongSwan Client-Identity
		\item[\texttt{hardwareId}] Die Id, welche das Gerät identifiziert, so zum
		Beispiel, AIK, Android-Id, DBUS machine-id, o.ä. Dies entspricht dem
		\texttt{value} Feld in der \texttt{device} Tabelle in der Datenbank
		\item[\texttt{productName}] Der Productname ist der Name des OS wie er in der
		\texttt{product} Tabelle der Datenbank steht
	\end{description*}
	\item[JSON Format Response] \hfill
\begin{jsoncode}
{
	"sessionId": 420,
	"workitems": [
		 {
		 	"id": 5,
		 	"uri": "https://strongtnc/api/sessions/420/workitems/5/",
		 	"session": "https://strongtnc/api/sessions/420",
		 	"type": 15,
		 	"argument": {
		 		"swidFlags": [
		 			"R"
		 		]
		 	}
		 }
	],
	"uri": "https://strongtnc/api/sessions/420"
}
\end{jsoncode}
\end{description*}
\end{mdframed}
Dieser Controller erstellt und startet eine Session, das Device welches der
Session zugeordnet werden soll wird anhand der \texttt{hardwareId} und dem
\texttt{productName} bestimmt. Falls eines der Objekte noch nicht existiert, wird
dieses durch den Controller erstellt.\\
Die Id, die im Response Dokument zurück geliefert wird, dient zur zukünftigen
Identifikation der soeben gestarteten Session. Ausserdem erhält man eine Liste
von Workitems, die für diese Session abgearbeitet werden müssen.

Um aktive und vergangene Sessions abzufragen existiert eine readonly
Collection, die Session Documents sind ebenfalls als readonly definiert.
Sessions sollen nicht direkt geändert werden, sondern nur über die
entsprechenden Controller. Der Grund dafür ist, dass im Hintergrund noch
zusätzliche Operationen vorgenommen werden müssen.\\
Dasselbe gilt für die Workitems, für diese sind ebenfalls eine readonly
Collection mir readonly Documents definiert. Worktitems werden durch den
Session-Start Controller anhand der Enforcements eines Devices automatisch
erstellt, darum dürfen diese nicht manuell verändert werden.

Die Messresultate müssen für jedes Workitem erfasst werden, dafür steht eine
virtuelle Ressource zur Verfügung:

\begin{mdframed}[style=def]
\begin{description*}
	\item[URI Path] \texttt{/session/\{id\}/workitems/\{id\}/result/}
	\item[Archetype] Document
	\item[Request Parameter] \hfill
	\begin{description*}
		\item[\texttt{recommendation}] Resultat/Empfehlung für dieses Workitem.
		\item[\texttt{comment}] Kommentar zum Resultat.
	\end{description*}
	\item[Methods] GET, POST
	\item[Response Statuscodes] \hfill
		\begin{description*}
			\item[201 Created] Resultat wurde erfolgreich gespeichert.
			\item[409 Conflict] Resultat existiert bereits.
		\end{description*}
	\item[JSON Format Response] \hfill
\begin{jsoncode}
{
	"recommendation": 0,
	"comment": "received inventory of 2165 SWID tag IDs and 0 SWID tags"
}
\end{jsoncode}
\end{description*}
\end{mdframed}

Da eine Session nach dem Abschluss nicht mehr verändert werden soll, sind die
Workitems flüchtig. Wenn die dazugehörige Session beendet wird, werden die
Workitems entfernt und die eingetragenen Resultate auf die Session übertragen.
Auf diese Weise können die Messresultate einer Session dauerhaft nachvollzogen
werden. Über den Verweis auf das Enforcement kann die Messung auch dem
ursprünglichen Zweck zugeordnet werden:

\begin{mdframed}[style=def]
\begin{description*}
	\item[URI Path] \texttt{/session/\{id\}/results/\{id\}/}
	\item[Archetype] Readonly Collection
	\item[Filter Query] \hfill
	\item[Methods] GET
	\item[JSON Format Response] \hfill
\begin{jsoncode}
{
	"id": 5,
	"uri": "https://strongtnc/api/session/420/results/5/",
	"enforcement": "https://strongtnc/api/enforcements/13/",
	"recommendation": 0,	 
	"comment": "received inventory of 2165 SWID tag IDs and 0 SWID tags"
}
\end{jsoncode}
\end{description*}
\end{mdframed}

\begin{mdframed}[style=def]
\begin{description*}
	\item[URI Path] \texttt{/session/\{id\}/results/}
	\item[Archetype] Readonly Collection
	\item[Methods] GET
	\item[Response] List of Result documents
\end{description*}
\end{mdframed}

Über folgenden Endpunkt kann eine Session abgeschlossen werden. Im Normalfall
wird dies gemacht, wenn alle Workitems abgearbeitet sind, dies ist allerdings
keine Voraussetzung, eine Session kann jederzeit abgeschlossen werden.

\begin{mdframed}[style=def]
\begin{description*}
	\item[URI Path] \texttt{/sessions/\{id\}/end/}
	\item[Archetype] Controller
	\item[Methods] POST
	\item[Request Parameter] \hfill
	\begin{description*}
		\item[\texttt{recommendation}] Endgültiges Resultat/Empfehlung für diese
		Session.
	\end{description*}
\end{description*}
\end{mdframed}

Wenn eine Session abgeschlossen wird, werden alle Workitems dieser Session
abgeräumt und die jeweiligen Resultate auf die Session übertragen.

\section{Implementation}
Die im Rahmen dieser Arbeit konzipierte HTTP Schnittstelle wurde stellenweise
als Machbarkeitsnachweis umgesetzt. Da die Entwicklung einer HTTP Schnittstelle
für strongTNC nicht Bestandteil der Aufgabenstellung war, wurde aus Zeitgründen
nicht der gesamte Umfang implementiert.\\
Für die Implementation wurde das \enquote{Django REST
Framework}\footnote{\url{http://www.django-rest-framework.org/}} (DRF)
verwendet.\\
<TODO: Danilo, Ein, zwei Gründe für die Wahl von DRF>.
- Class Based Views
- Browsable API
- Mozilla verwendet es auch
- Contribution
- ...

\subsection{Serialisierung}
\label{api:serialisierung}
Die Django Models müssen für die Übermittlung in ein passendes Format
serialisiert werden. Das von Client gewünschte Format wird durch den
\texttt{Content-Type} Header spezifiziert. DRF implementiert
Standard-Serialisierer für das JSON und XML Format, welche direkt verwendet
werden können.\\ Wie im REST Konzept beschrieben, wäre es hilfreich, den
Benutzern der API die Möglichkeit anzubieten, über Filter nur ausgewählte Felder
der Datenstrukturen zu erhalten. Das heisst, es müssen nicht immer alle Felder
eines Objektes zu serialisiert werden, sondern nur ein Subset davon,
vergleichbar mit einer SQL Projektion, \texttt{SELECT feld1, feld2 ...}. DRF
bietet diese Möglichkeit standardmässig nicht. Um dieses Verhalten zu erreichen
wurde eine Komponente entwickelt, welche die Auswahl der Felder durch
Query-Parameter in der URL erlaubt. Der Aufruf von
\texttt{https://strongtnc/api/swid-tags/2/?fields=packageName,version} liefert
beispielsweise nur die Felder \texttt{packageName} und \texttt{version} des
abgefragten SWID-Tags. Die Umsetzung wurde mittels \enquote{Mixin-based
Inheritance}\cite{bracha1990mixin} realisiert.\\
Serverseitig werden die Model-Serialisierer durch die Klasse
\texttt{DynamicFieldsMixin} erweitert. Diese extrahiert vor dem Serialisieren
die Parameter aus dem Request und übergibt sie dem Serialisierer (Listing
\ref{api:tagserializer}). Die Mixin-Klasse ist vollständig entkoppelt von der
Implementierung des Serialisierens und so kann für alle Serialisierer verwendet
werden.

\begin{listing}
\caption{Erweiterung durch \texttt{DynamicFieldsMixin} zur Abfrage bestimmter Felder}
\label{api:tagserializer}
\begin{pythoncode}
class TagSerializer(DynamicFieldsMixin, serializers.HyperlinkedModelSerializer):
    entities = EntityRoleSerializer(source='entityrole_set', many=True)
\end{pythoncode}
\end{listing}

\subsection{Fehlerbehandlung}
Um für API Benutzer die Fehlerbehandlung zu erleichtern, werden bei auftretenden
Fehlern detaillierte Informationen im Body der Response ausgeliefert. Ein
möglicher Fehler ist beispielsweise das Übermitteln eines ungültig formatierten
JSON Objektes. (TODO kann man das so generell sagen, wird das überall gemacht?
oder nur bei der SWID Messung? Nur im Debugmode?)

\begin{listing}
\caption{Fehlerinformation beim Übermitteln eines ungültigen JSON Objektes}
\begin{httpcode}
HTTP/1.0 400 BAD REQUEST
Content-Type: application/json
Vary: Accept
Allow: POST, OPTIONS

{
    "detail": "JSON parse error - Expecting , delimiter: line 1 column 27 (char 27)"
}
\end{httpcode}
\end{listing}

\subsection{Verlinkung}
Meist besitzen Objekte nicht nur einfache Attribute sondern auch Verweise auf
andere Objekte, zum Beispiel besitzt ein SWID Tag mindestens eine Entity. Bei
einer Entity handelt es sich um ein Objekt mit weiteren Attributen.\\
Solche entfernten Objekte werden nicht serialisiert. Anstelle des serialisierten
Objektes wird eine URI ausgeliefert, über welche das Objekt aufgerufen werden
kann. Das API Konzept sieht jedoch vor, einen Parameter einzuführen, welcher es
erlaubt, die Serialisierungstiefe zu bestimmen. Aus zeitlichen Gründen konnte
diese Erweiterung nicht implementiert werden.

\begin{listing}
\caption{Serialisierter SWID-Tag}
\begin{httpcode}
HTTP/1.0 200 OK
Content-Type: application/json
Vary: Accept
Allow: GET, HEAD, OPTIONS

{
    "id": 2, 
    "uri": "https://strongtnc:8000/api/swid-tags/2/", 
    "packageName": "account-plugin-facebook", 
    "version": "0.11+14.04.20140409.1-0ubuntu1", 
    "uniqueId": "Ubuntu_14.04-x86_64-account-plugin-facebook-0.11+14.04.20140409.1-0ubuntu1", 
    "entities": [
        {
            "entity": "https://strongtnc/api/swid-entities/11/", 
            "role": 2
        }
    ], 
    "swidXml": "<?xml version='1.0' encoding='UTF-8'?>\n<SoftwareIdentity ..."
}
\end{httpcode}
\end{listing}

\subsection{Browsable API}
TODO Danilo
Das Django REST Framework bietet die möglichkeit die API mittels Browser zu durchsuchen. 
