\chapter{Abstract}

Das amerikanische National Cybersecurity Center of Excellence
schlägt in einem Entwurfsdokument vor, durch eine kontinuierliche
Überwachung der installierten Software auf Client-Systemen
(Desktop PCs, Laptops, Tablets, Smartphones, etc) die Gefahr von
Cyberattacken zu minimieren. Das Software Asset Management soll
über Software Identification (SWID) Tags erfolgen, die durch die
ISO/IEC 19770-2:2009 Norm international standardisiert sind.

Im Rahmen dieser Arbeit wurde eine Client Komponente für Linux Systeme
entwickelt, welche SWID Tags gemäss ISO 19770-2:2009 aus den installierten
Softwarepaketen generiert. Implementiert wurde die Unterstützung für drei weit
verbreitete Paket Manager. Die Architektur des Generators ist modular aufgebaut, so dass die
Unterstützung für weitere Umgebungen einfach hinzugefügt werden kann.

Das so generierten SWID Tags werden an einen zentralen Server
übermittelt. Die Trusted Computing Group hat ein offenes Framework für
das aktive Monitoring von Endgeräten mit dem Namen Trusted Network
Connect (TNC) entwickelt. In einer Vorgängerarbeit wurde bereits eine erste Version eines
TNC Policy Managers implementiert, welcher nun für die Verwaltung der SWID Tags
erweitert wurde.

Über frei definierbare Policies können Clients mit bestimmten Softwarepaketen
oder -versionen automatisch aus dem Netzwerk ausgeschlossen oder zu Software
Updates gezwungen werden. Dadurch wird ein umfassendes Software Asset
Management ermöglicht, welches sich in bestehende Geschäftsprozesse integrieren
lässt.

Der strongTNC Policy Manager was bisher über eine gemeinsame Datenbank eng mit
der strongSwan IPsec Implementierung gekoppelt. Diese Kopplung wirkt sich
negativ auf die Wartbarkeit und die Interoperabilität mit Drittsystemen aus.
Aus diesem Grund wurde im Rahmen dieser Arbeit ein Konzept erarbeitet, welches
eine Serviceorientierte Architektur vorsieht und so den Grad der Kopplung minimiert.
Bestandteil des Konzeptes ist die Definition einer REST Schnittstelle sowie ein Proof of Concept, welcher
diese Schnittstelle gemäss Vorschlag für die Verwaltung der SWID Tags bereits
erfolgreich implementiert. Da das TNC Framework bisher keine einheitlichen Schnittstellen für die Kommunikation
zwischen dem Policy Manager und Umsystemen vorsieht, wird das
erarbeitete Konzept der Trusted Computing Group als Vorschlag
unterbreitet.