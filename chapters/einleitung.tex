\chapter{Einleitung}
\section{Einleitung}

\subsection{Vorbemerkungen}
Bei diesem Dokument handelt es sich um die Bachelorarbeit von Danilo Bargen,
Christian Fässler und Jonas Furrer, erstellt an der Hochschule für Technik Rapperswil 
im Studiengang Informatik. Betreut und begleitet durch Prof. Dr
Andreas Steffen und Tobias Brunner, ITA HSR (Institut für Internettechnologien
und Applikationen).\\
Die Inhalte dieser Arbeit wurden von den Studenten zu gleichen Teilen
erarbeitet.

\subsection{Zweck}
Der Technische Bericht beschreibt den technischen Aufbau der erarbeiten Lösung,
zudem erläutert er Entscheidungen betreffend Design und Vorgehen.

\subsection{Einführung}
Die Infrastruktur und die Organisation von Netzwerken wird über die Zeit immer
komplexer. Neue Technologien und Trends müssen in bestehende Konstrukte
integriert werden. Beinahe täglich gibt es neue Herausforderungen, die
gemeistert werden müssen.\\
Insbesondere im Bereich der Netzwerksicherheit ist es wichtig, die aktuellen
Entwicklungen nicht aus den Augen zu verlieren. Seit längerem ist \enquote{Bring
your own device} (BYOD) ein Thema, welches in der Netzwerksicherheit Probleme
bereitet. Virtuelle Netzwerke mit integriertem \enquote{Trusted Network Connect}
(TNC) bieten eine gute Grundlage um zu kontrollieren, welche Geräte
Zugang zu einem Netzwerk erhalten und welche nicht. Eine wünschenswerte
Ergänzung ist die kontinuierliche Überwachung der installierten Software eines
Gerätes, dies schlägt die National Cybersecurity Center of Excellence (NCCoE) in
einem Entwurfsdokument vor. Laut NCCoE sind die, durch die ISO international
standardisierten, Software Identification Tags für diesen Zweck geeignet.\\
Nun soll die Unterstützung für Software Identification Tags in vorhandene TNC
Implementationen integriert werden, um dadurch die Sicherheit von Netzwerken zu
erhöhen und sich den Herausforderungen der Zukunft besser stellen zu können.