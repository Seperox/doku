\chapter{Einleitung}

\nobrsection{Vorbemerkungen}

Bei diesem Dokument handelt es sich um die Bachelorarbeit von Danilo Bargen,
Christian Fässler und Jonas Furrer, erstellt an der Hochschule für Technik
Rapperswil (HSR) im Studiengang Informatik. Betreut und begleitet wurde die
Arbeit durch Prof. Dr. Andreas Steffen und Tobias Brunner, Institut für
Internettechnologien und Applikationen (ITA). Die Inhalte dieser Arbeit wurden
von den Studenten zu gleichen Teilen erarbeitet.

\nobrsection{Zweck}

Der Technische Bericht beschreibt den Aufbau der erarbeiteten Lösung und
erläutert Entscheidungen betreffend Design und Vorgehen.

\nobrsection{Einführung}

Die Infrastruktur und die Organisation von Netzwerken wird immer komplexer. Neue
Technologien und Trends müssen in bestehende Umgebungen integriert werden,
beinahe täglich tauchen es neue Herausforderungen auf, die gemeistert werden
müssen. \\ Gerade im Bereich der Netzwerksicherheit ist es wichtig, die
aktuellen Entwicklungen nicht aus den Augen zu verlieren. Seit längerem ist
\enquote{Bring your own device} (BYOD) ein Thema, welches mit Netzwerksicherheit
schwer zu vereinbaren ist. Virtuelle Netzwerke mit integriertem \enquote{Trusted
Network Connect} (TNC) bieten eine gute Grundlage, um zu kontrollieren, welche
Geräte Zugang zu einem Netzwerk erhalten und welche nicht. Eine wünschenswerte
Ergänzung ist die kontinuierliche Überwachung der installierten Software eines
Gerätes. Dies schlägt das National Cybersecurity Center of Excellence (NCCoE) in
einem Entwurfsdokument vor. Laut dem NCCoE sind die durch die ISO international
standardisierten Software Identification (SWID) Tags für diesen Zweck geeignet.
\\ Nun soll die Unterstützung für SWID Tags in vorhandene TNC Implementationen
integriert werden, um dadurch die Sicherheit von Netzwerken zu erhöhen und sich
den Herausforderungen der Zukunft besser stellen zu können.
