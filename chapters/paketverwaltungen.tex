\section{Paketverwaltungen}


\subsubsection{DPKG}

DPKG\footnote{\url{https://alioth.debian.org/projects/dpkg}} (Abkürzung für
\textit{Debian Package}) ist die Basis der Paketverwaltung in Debian und in
verwandten Distributionen wie Ubuntu. DPKG verwaltet unter anderem alle
installierten Software-Pakete inklusive Meta-Informationen. Diese Paketliste
kann mit \texttt{dpkg-query} abgefragt werden.

\paragraph{Installierte Pakete abfragen} \hspace{0pt} \\

\noindent Mit \texttt{dpkg ---show} wird eine Liste aller installierter Pakete
ausgegeben, ein Paket pro Zeile. Um die Ausgabe zu personalisieren kann das
\texttt{---showformat} Flag verwendet werden. Im Falle des swidGenerators
ist folgendes Format ideal:

\begin{bashcode}
dpkg-query --show --showformat='${Package}\t${Version}\t${Status}\n'
\end{bashcode}

\noindent Hierbei ist noch anzumerken, dass entfernte DPKG Pakete einen sogenannten RC Status haben können. Dieser liegt vor, wenn ein Paket deinstalliert, aber die Konfigurationsdateien auf dem System belassen wurden. Die \texttt{---show} Option liefert auch diese Pakete . Dieser Der Zustand im Status Feld als \texttt{deinstall ok config-files} ersichtlich, entsprechende Pakete können so nachträglich im swidGenerator noch gefiltert werden.
\paragraph{Paket-Dateien abfragen} \hspace{0pt} \\

\noindent Um die zu einem Paket zugehörigen Dateien abzufragen, kann das
\texttt{---listfiles} Flag verwendet werden:

\begin{bashcode}
dpkg-query --listfiles <package-name>
\end{bashcode}

\paragraph{Datei-Hashes abfragen} \hspace{0pt} \\

\noindent Es besteht die Möglichkeit, aus DPKG MD5-Hashes der Dateien abzufragen. Weitere
Hash-Algorithmen (\zb SHA) sind nicht verfügbar.


\subsubsection{RPM}

RPM\footnote{\url{https://rpm.org}} (Abkürzung für
\textit{Red Hat Package Manager}) ist die Standardpaketverwaltung für Red Hat Systeme sowie zahlreichen verwandten Distributionen wie SUSE, Mandriva und Fedora.

\paragraph{Installierte Pakete abfragen} \hspace{0pt} \\

\noindent Ähnlich wie bei dpkg können alle auf dem System vorhandenen Pakete abgefragt werden. Die Ausgabe kann auch via Format String personalisiert werden. Für den swidGenerator ist folgendes Format ideal:

\begin{bashcode}
rpm -qa --queryformat %{name}\t%{version}-%{release}
\end{bashcode}

\paragraph{Paket-Dateien abfragen} \hspace{0pt} \\

\noindent Um die zu einem Paket zugehörigen Dateien abzufragen, können die Parameter \texttt{--ql} verwendet werden:

\begin{bashcode}
rpm -ql <package-name>
\end{bashcode}

\noindent TODO
