\section{Paketverwaltungen}


\subsubsection{DPKG}

DPKG\footnote{\url{https://alioth.debian.org/projects/dpkg}} (Abkürzung für
\textit{Debian Package}) ist die Basis der Paketverwaltung in Debian und in
verwandten Distributionen wie Ubuntu. DPKG verwaltet unter anderem alle
installierten Software-Pakete inklusive Meta-Informationen. Diese Paketliste
kann mit \texttt{dpkg-query} abgefragt werden.

\paragraph{Installierte Pakete abfragen} \hspace{0pt} \\

\noindent Mit \texttt{dpkg ---show} wird eine Liste aller installierter Pakete
ausgegeben, ein Paket pro Zeile. Um die Ausgabe zu personalisieren kann das
\texttt{---showformat} Flag verwendet werden. Im Falle des swidGenerators
ist folgendes Format ideal:

\begin{bashcode}
dpkg-query --show --showformat='${Package}\t${Version}\t${Status}\n'
\end{bashcode}

\paragraph{Paket-Dateien abfragen} \hspace{0pt} \\

\noindent Um die zu einem Paket zugehörigen Dateien abzufragen, kann das
\texttt{---listfiles} Flag verwendet werden:

\begin{bashcode}
dpkg-query --listfiles <package-name>
\end{bashcode}

\paragraph{Datei-Hashes abfragen} \hspace{0pt} \\

\noindent Es besteht die Möglichkeit, aus DPKG MD5-Hashes der Dateien abzufragen. Weitere
Hash-Algorithmen (\zb SHA) sind nicht verfügbar.


\subsubsection{RPM}

TODO Einleitung

\paragraph{Installierte Pakete abfragen} \hspace{0pt} \\

\noindent TODO

\paragraph{Paket-Dateien abfragen} \hspace{0pt} \\

\noindent TODO
