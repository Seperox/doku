\chapter{Management Summary}

\section{Ausgangslage}
Das amerikanische National Cybersecurity Center of Excellence schlägt in einem
Entwurfsdokument  vor, durch eine kontinuierliche Überwachung der installierten
Software auf Client-Systemen die Gefahr von Cyberattacken zu minimieren. Das
Software  Asset Management soll über Software Identification (SWID) Tags
erfolgen, welche durch die ISO/IEC 19770-2\cite{iso19770-2} Norm international
standardisiert sind. Die Trusted Computing Group hat ein offenes Framework für
das aktive Monitoring von Endgeräten  mit dem Namen Trusted Network Connect
(TNC) entwickelt. In einer Vorgängerarbeit wurde bereits eine erste Version
eines TNC Policy Managers implementiert, welcher nun für das Software Asset
Managements mittels sogenannter Software Identification (SWID) Tags, welche
durch die ISO/IEC 19770-2 Norm international standardisiert sind,
erweitert werden soll. Zusätzlich soll ein Client für Debian und Ubuntu basierte
Systeme erstellt werden, welcher SWID Tags aus den Informationen des
Paketmanagers generiert.

\section{Vorgehen / Technologien}
Der SWID Generator wurde vollständig in Python entwickelt. Es wurde darauf
geachtet, dass keine Abhängigkeiten zu externen Bibliotheken bestehen. Dadurch
kann sichergestellt werden, dass der Client problemlos zusammen mit dem
strongSwan VPN Client verteilt werden kann. Die Analyse des bestehenden
strongTNC Policy Managers hat gezeigt, dass eine enge Kopplung mit der
strongSwan IPsec Implementierung besteht. Zudem ist aufgefallen, dass der
bereits bestehende Code teilweise Qualitätsmängel aufweist. Diese Beiden Punkte
wurden neben der Integration der SWID Tag Verwaltung zum zentralen Bestandteil
dieser Arbeit.

\section{Ergebnis}
Die Architektur des SWID Generator wurde modular ausgelegt, so dass die
Unterstützung für weitere Paketmanager einfach hinzugefügt werden kann. Zum
jetzigen Zeitpunkt werden die drei am weitverbreitetsten Paketmanager (DPKG,
RPM, Pacman) unterstützt. Um das Deployment zu erleichtern, wurde  der Generator
in den Python Package Index aufgenommen. Für die Entkopplung des strongTNC
Policy Managers gegenüber den Umsystemen wurde ein API Konzept ausgearbeitet,
welches eine serviceorientierte Architektur vorsieht und bereits erfolgreich für
die neu integrierten Komponenten umgesetzt wurde. Das Konzept soll nun als
Vorschlag zur Umsetzung einer TNC Schnittstelle der TCG unterbreitet werden.
Mittels Continuous Integration, Refactoring, und zusätzlichen Integrations- und
Unittests konnte die Codequalität messbar verbessert werden.
