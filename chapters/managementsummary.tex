\chapter{Management Summary}

\nnobrsection{Ausgangslage}

Das amerikanische National Cybersecurity Center of Excellence schlägt in einem
Entwurfsdokument vor, durch eine kontinuierliche Überwachung der installierten
Software auf Client-Systemen (Desktop PCs, Laptops, Tablets, Smartphones) die
Gefahr von Cyberattacken zu minimieren. Das Software Asset Management soll über
Software Identification (SWID) Tags erfolgen, welche durch die ISO/IEC
19770-2\cite{iso19770-2} Norm international standardisiert sind.

Die Trusted Computing Group (TCG) hat ein offenes Framework für das aktive
Monitoring von Endgeräten mit dem Namen Trusted Network Connect (TNC)
entwickelt. In einer Vorgängerarbeit wurde bereits eine erste Version eines TNC
Policy Managers implementiert, welcher nun für das Software Asset Managements
mittels SWID Tags erweitert werden soll.

Zusätzlich soll ein Client für Debian und Ubuntu basierte Systeme erstellt
werden, welcher SWID Tags aus den Informationen des Paketmanagers generiert.

\nnobrsection{Vorgehen / Technologien}

Der SWID Generator wurde vollständig in Python entwickelt. Es wurde darauf
geachtet, dass keine Abhängigkeiten zu externen Bibliotheken bestehen. Dadurch
kann sichergestellt werden, dass der Client problemlos zusammen mit dem
strongSwan VPN Client verteilt werden kann. Die Analyse des bestehenden
strongTNC Policy Managers hat gezeigt, dass eine enge Kopplung mit dem
strongSwan TNC Server besteht. Zudem ist aufgefallen, dass der bereits
bestehende Code Qualitätsmängel aufweist. Diese beiden Punkte wurden neben der
Integration der SWID Tag Verwaltung zum zentralen Bestandteil dieser Arbeit.\\

\nnobrsection{Ergebnis}

Die Architektur des SWID Generators wurde modular ausgelegt, so dass die
Unterstützung für weitere Paketmanager einfach hinzugefügt werden kann. Zum
jetzigen Zeitpunkt werden die drei weitverbreitetsten Paketmanager (DPKG,
RPM, Pacman) unterstützt. Um das Deployment zu erleichtern, wurde der Generator
in den Python Package Index aufgenommen.

Für die Entkopplung des strongTNC Policy Managers gegenüber den Umsystemen wurde
ein API Konzept ausgearbeitet, welches eine serviceorientierte Architektur
vorsieht und bereits erfolgreich für die neu integrierten Komponenten umgesetzt
wurde. Das Konzept soll nun als Vorschlag zur Umsetzung einer TNC Schnittstelle
der Trusted Computing Group unterbreitet werden.

Mittels Continuous Integration, Refactoring und zusätzlichen Integrations- und
Unittests konnte die Codequalität von strongTNC messbar verbessert werden.
