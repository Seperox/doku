\chapter{Analyse}

\section{Ist Situation}

\section{Soll Situation}

\section{Abgrenzung}

\section{Aufgabe und Ziel}

\section{ISO Standard 19770-2}
\subsection{Einleitung}

In diesem Kapitel wird auf den ISO Standard 19770-2 eingegangen. 

\subsection{Doppelpunkte in der Unique ID}
\begin{quote}
unique\_id und that may be either a GUID, or any reference unique for the tag\_creator\_regid. The unique\_id shall 
follow the restrictions for URI character use as specified in IETF RFC 3986, section 2, characters.
\end{quote}
\textit{ISO/IEC CD 19770-2 (N6078 - ISO CD 19770-2-5.pdf) Page 13}

Die Unique ID unterliegt den Restrikitonen einer URI. Bei URIs gilt, ein Doppelpunkt ist kein unerlaubtes Zeichen jedoch ein reserviertes: 

\begin{verbatim}
reserved = gen-delims / sub-delims gen-delims = ":" / "/" / "?" / "#" / "[" / "]" / "@" 
sub-delims = "!" / "\$" / "&" / "'" / "(" / ")" / "*" / "+" / "," / ";" / "="
unreserved = ALPHA / DIGIT / "-" / "." / "_ " / "~"
\end{verbatim}

\textit{See: http://www.ietf.org/rfc/rfc3986.txt}

Die Unique ID kann als Attribut des Links Tags, in Form einer URI mit dem \texttt{swid:\\} Schema. Daraus folgt, dass die Unique ID keine Doppelpunkte enthalten sollte.
