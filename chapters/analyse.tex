\chapter{Analyse}

\section{Ist-Situation}

In der Vorgängerarbeit \enquote{Cygnet} von Stefan Rohner und Marco
Tanner\cite{cygnet:2013} wurde der Policy Manager \enquote{strongTNC}
implementiert. Das Produkt und die Codebasis dienen als Grundlage
für diese Arbeit.

Im Bereich des SWID Generators gibt es zur Zeit noch keine bestehenden Produkte.

\subsection{Elemente der strongTNC App}
Im Folgenden werden die zentralen Elemente der strongTNC Umgebung beschrieben.
Die Namen entsprechen jenen der bestehenden Django-Models. Diese Liste soll einen
kurzen Überblick geben, eine ausführliche Beschreibung ist in der Dokumentation
\enquote{Cygnet}\cite{cygnet:2013} zu finden.

\begin{description}
	\item[Policy] Eine Policy beschreibt eine zu überprüfende Richtlinie. Durch
	einen Policytyp wird festgelegt, was überprüft werden soll (\zb erlaubte offene
	Ports, installierte Software). Weiterhin kann festgelegt werden, welche Aktion
	bei positivem oder negativen Ausgang der Messung (Überprüfung der Richtlinie)
	ausgeführt werden soll.
	
	\item[Enforcement] Policies werden durch Enforcements erzwungen. Ein
	Enforcement wird einer Gruppe von Geräten zugeteilt. Spezifiert wird zudem, in
	welchen zeitlichen Abständen die verknüpfe Policy überprüft wird.

	\item[Group] Geräte können in Gruppen zusammengefasst werden. Gruppen dienen
	letztendlich dazu, zu bestimmen, welche Enforcements zur Anwendung kommen.
	Gruppen lassen sich hierarchisch aufbauen.

	\item[Session] Wenn sich ein Gerät mit einem strongSwan TNC Netzwerk verbindet,
	wird durch strongSwan eine Session für das verbundene Gerät erstellt. Sämtliche
	Messungen werden mit dieser Session verknüpft.

	\item[Workitem] Für jede Policy, die durch ein Enforcement während einer
	Session geprüft werden muss, wird ein Workitem erstellt. Diese Workitems werden
	durch die IMVs (Integrity Measurement Verifiers) von strongSwan abgearbeitet.
	Die Resultate werden nach den Messungen in den Workitems abgelegt.

	\item[Device] Ein Device repräsentiert ein Gerät, welches sich in ein
	strongSwan TNC Netzwerk verbinden kann. Ein Gerät wird durch die Kombination
	einer eindeutigen Hardware ID sowie dem Betriebssystem (siehe Product)
	identifiziert. Ein Device ist Mitglied einer oder mehrerer Gruppen; diese
	Mitgliedschaft bestimmt, welche Policies auf das Gerät angewendet werden.

	\item[Product] Ein Product entspricht dem Betriebssystem eines Devices. Einem
	Product können Gruppen zugeteilt werden. Diese Gruppen stellen die
	Standardgruppen eines Gerätes mit einem bestimmten Betriebssystem dar.

	\item[Package] Ein Package beschreibt ein Softwarepaket. Packages werden mit
	Versionen verknüpft.

	\item[Version] Eine Version ist immer mit einem Package und einem Product
	verknüpft. Das heisst, für jedes Betriebssystem existiert eine eigene Version,
	auch wenn die Versionsbezeichnung dieselbe ist. Versionen können global als
	\enquote{blacklisted} und \enquote{security patch} markiert werden. Versionen
	können Messziele einer Policy sein.

	\item[Directory] Ein Directory kann Bestandteil einer Policy und somit
	Grundlage einer Messung sein.

	\item[File] Files dienen als Grundlage für File-Messungen.

	\item[FileHash] Bei File-Messungen können Hashes einer gewissen Datei erfasst
	werden, diese werden zusammen mit dem Product des gemessenen Gerätes einem
	File zugewiesen.
	
\end{description}

\needspace{4\baselineskip}
In \autoref{django-ist-diagram} ist eine Übersicht der verwendeten Django-Models
zu sehen:
\begin{figure}[H]
	\centering
	\includegraphics[width=\textwidth]{images/architecture/django_apps_ist}
	\caption{Ursprünglich vorhandene strongTNC Models}
	\label{django-ist-diagram}
\end{figure}

\subsection{Ablauf einer TNC Messung mit strongSwan und strongTNC}

% TODO ganzen abschnitt etwas umformulieren? viele "wird" und "durch".
Wenn sich ein Gerät mit einem strongSwan TNC Netzwerk verbindet, wird durch
einen Integrity Measurement Verifier (IMV) eine strongTNC Session gestartet.
Von diesem wird das Gerät anhand seiner Hardware ID und seines Betriebssystems
identifiziert. Falls sich das Gerät zum ersten Mal verbindet, wird es neu
erfasst. Anhand der Gruppen, denen das Gerät angehört, werden die relevanten
Enforcements gesammelt und daraus Workitems generiert. Diese Vorgänge werden
derzeit alle durch den IMV mittels direktem Zugriff auf die Datenbank
durchgeführt.\\
Verschiedene IMVs führen anhand der Workitems entsprechende Messungen durch. Die
Ergebnisse der Messungen werden durch die IMVs in die Workitems der strongTNC
Datenbank geschrieben. Anhand der abgearbeiteten Workitems fällt der TNC Server
eine abschliessende Entscheidung darüber, ob das Gerät in das Netzwerk
zugelassen wird oder nicht. Ein IMV schliesst die Session in der strongTNC
Datenbank ab und die Resultate der Workitems werden archiviert.\\

Den Ablauf einer Messung kann man dem Sequenzdiagramm in
\autoref{masurement-diagramm} entnehmen.

\begin{figure}[H]	
	\centering
	\begin{tikzpicture}
	\begin{umlseqdiag}[font=\sffamily]

		% Objekte
		\umlobject{TNC Server}
		\umlobject{IMV Manager}
		\umlmulti{IMVs}
		\umlobject{strongTNC}
		
		% Calls
		\begin{umlcall}[dt=5, op=startSession]
			{TNC Server}{IMV Manager}
		\end{umlcall}
		
		\begin{umlcall}[dt=5, op=startSession]
			{IMV Manager}{IMVs}
		\end{umlcall}	
		
		\begin{umlcall}[dt=5, op=startSession]
			{IMVs}{strongTNC}
		\end{umlcall}
		
		\begin{umlcall}[dt=5, op=evaluateEnforcements]
			{IMVs}{strongTNC}
		\end{umlcall}
		
		\begin{umlcall}[dt=5, op=generateWorkitems]
			{IMVs}{strongTNC}
		\end{umlcall}
		
		\begin{umlcall}[dt=5, op=setWorkitemList]
			{IMVs}{IMV Manager}
		\end{umlcall}	
		
		\begin{umlfragment}[type=loop, label={foreach Workitem}, inner xsep=15, inner ysep=2]
			\begin{umlcall}[dt=9, op=getOpenWorkitem]
				{IMVs}{IMV Manager}
			\end{umlcall}	
			
			\begin{umlcallself}[dt=5, op=doMeasurement]
				{IMVs}
			\end{umlcallself}	
			
			\begin{umlcall}[dt=5, op=storeResult]
				{IMVs}{strongTNC}
			\end{umlcall}
			
		\end{umlfragment}

		\begin{umlcall}[dt=25, op=noWorkitemsLeft]
			{IMV Manager}{TNC Server}
		\end{umlcall}			
		
		\begin{umlcall}[dt=5, op=stopSession]
			{TNC Server}{IMV Manager}
		\end{umlcall}
		
		\begin{umlcall}[dt=5, op=stopSession]
			{IMV Manager}{IMVs}
		\end{umlcall}	
		
		\begin{umlcall}[dt=5, op=stopSession]
			{IMVs}{strongTNC}
		\end{umlcall}
		
		\begin{umlcall}[dt=5, op=cleanUpWorkitems]
			{IMVs}{strongTNC}
		\end{umlcall}

	\end{umlseqdiag}
\end{tikzpicture}

	\caption{Sequenzdiagramm, strongSwan TNC Messung}
	\label{masurement-diagramm}
\end{figure}


\subsection{Stand der strongTNC Webapplikation} 
\label{analyse:stand}

strongTNC ist eine Implementierung eines TNC Policy Managers. Umgesetzt wurde
diese als Webapp mit dem Python basierten \textit{Django Web Framework} in
Version 1.5\footnote{\url{https://www.djangoproject.com/}}. Die Applikation
bietet Unterstützung für das Verwalten von Geräten, Policies und Enforcements
und erlaubt die Ausführung von Enforcements mittels hierarchischer Gruppen
zu steuern. Stammdaten wie Dateien oder Softwarepakete können eingesehen und
teilweise bearbeitet werden. Auch Messresultate werden passend aufbereitet.

Für SWID Tags besteht derzeit keine Unterstützung. Es wurden zwar bereits Views
für SWID Tags und Regids erstellt, diese haben jedoch ausser der Auflistung der
Datenbankinhalte keine weitere Funktionalität und dienen bisher als Vorschau
für zukünftige Features.

Im Frontend der Webapplikation wurden Bootstrap
2.3\footnote{\url{http://getbootstrap.com/2.3.2/}} sowie jQuery
1.9\footnote{\url{http://jquery.com/1.9}} eingesetzt. Grundsätzlich werden die
Views als \enquote{Master-Detail Ansichten} dargestellt. Als Schnittstelle zur
strongSwan TNC Server Infrastruktur dient eine gemeinsam genutzte
SQLite\footnote{\url{http://www.sqlite.org/}} Datenbank.

Der Code ist in einer einzelnen Django-App organisiert, es gibt keine
thematische Aufteilung in mehrere Apps.

Das Schema der Datenbank ist unabhängig von den Models, diese bilden das
aktuelle Schema ab. Um die Datenbank zu initialisieren wird ein separat
bereitgestelltes SQL Schema verwendet.

Laut dem Technischen Bericht der Vorgängerarbeit
\enquote{Cygnet}\cite{cygnet:2013} sollen die Richtlinien des \enquote{Style
Guide for Python Code}\cite{PEP8:2001}, genannt PEP8, eingehalten werden.
Ausserdem existieren elf Unit-Tests, welche den bestehenden Python Code zu 21\%
abdecken.


\subsubsection{Einschätzung der Ist-Situation}
\label{analyse:einschaezung}

\begin{description}

	\item[SWID Integration] Es existiert noch keine Möglichkeit zur Erfassung und
	Verwaltung von SWID Tags. Die bestehenden Views können gegebenenfalls als
	Vorlage für eine solche Integration verwendet werden.
	
	\item[Gemeinsame Datenbank] Durch die gemeinsame Datenbank wird die
	Interoperabilität mit anderen Systemen eingeschränkt und die Wartbarkeit sowie
	die Ausbaufähigkeit beteiligter Komponenten erschwert. In diesem Fall wird
	ausserdem SQLite als Datenbank verwendet. SQLite ist nur bei lesendem Zugriff
	mehrbenutzerfähig, bei schreibendem Zugriff wird die Datenbank exklusiv
	gesperrt. Da eine gemeinsam genutzte Datenbank Mehrbenutzerbetrieb impliziert,
	besteht Handlungsbedarf.

	\item[Benutzerschnittstelle] Die Benutzeroberfläche wurde noch nicht für den
	Umgang mit grossen Datenmengen optimiert.Beispielsweise kann das Inventar der
	Dateinamen in einem Linux System durchaus 40000 Objekte umfassen. Wird die
	Policy View aufgerufen, werden alle 40000 Einträge geladen.\\ Diesen
	Datenmengen wird zur Zeit noch nicht Rechnung getragen. Tendenziell ist der
	Einsatz einer Endpoint Compliance Lösung in mittelgrossen und grossen
	Unternehmungen zu erwarten, dementsprechend ist auch die Menge der zu
	verwaltenden Daten als hoch einzuschätzen.
	
	\item[Django-Apps] Django Webapplikationen lassen sich in Module, sogenannte
	\enquote{Apps}, aufteilen. Diese Möglichkeit wird allerdings bisher nicht
	genutzt. Stattdessen befindet sich der gesamte Code in einer einzelnen App
	namens \texttt{tncapp}. Eine solche Unterteilung würde die Wartbarkeit erhöhen.
	
	\item[Datenbank-Schema] Bei der Umsetzung von Webapplikationen mit Django ist es
	gängig, dass das Datenbankschema aus den Models generiert wird. Dadurch muss man
	sich als Entwickler nicht um Schema-Detailfragen kümmern, sondern kann sich auf
	die abstrakten Models konzentrieren.\\
	Ein weiterer Vorteil ist, dass das genutzte DBMS beliebig austauschbar ist. Da
	die Datenbank bei strongTNC jedoch als Schnittstelle dient und nicht
	ausschliesslich von Django verwendet wird, kann dieses Feature derzeit nicht
	genutzt werden. Stattdessen bilden die Model-Definitionen den jeweiligen Stand
	der Datenbank (\autoref{fig:database-model}) ab. Weil die Datenbank nicht aus
	den Models generiert wird, ist es möglich, dass die Models und das effektive
	Schema bei Änderungen nicht mehr synchron sind und manuell angepasst werden
	müssen. Eine \enquote{Autorität} für das Schema sowie die Nutzung eines
	Schema-Migrations-Frameworks würde die Situation stark verbessern.
	
	\item[Codequalität] Obwohl sich die Vorgängerarbeit gemäss eigener Aussagen auf
	die Einhaltung der PEP8-Coderichtlinien\cite{PEP8:2001} beruft
	\cite[S.~79]{cygnet:2013}, wurden diese Richtlinien nicht konsequent
	eingehalten. \\
	Alle Views wurden als Funktionen definiert, obwohl sie häufig über gemeinsame
	Funktionalität verfügen. Auch die zyklomatische Komplexität der Views ist oft
	hoch. Ein klassenbasierter Ansatz, eventuell auch in Kombination mit den
	generischen Views von Django, könnte die Wartbarkeit verbessern.\\
	Die Validierung von übermittelten Formulardaten wurde manuell mittels Javascript
	gemacht, anstatt das vorhandene Formular-Framework von Django zu verwenden.
	Auch dies verursacht mehrfach vorhandenen Code und verletzt unnötigerweise das
	DRY-Prinzip\cite[S.~26--27]{hunt1999pragmatic}. \\
	Die Durchführung statischer Code-Analyse zeigt Mängel auf, wie
	beispielsweise unerreichbarer Code, ungenutzte Variablen, mehrfach definierte
	Funktionen oder verdeckte Python-Standardfunktionen. \\
	Die Templates sind inkonsistent eingerückt, man findet sich im Markup schwer
	zurecht. Zudem sind viele Elemente -- wie \zb das Suchfeld -- mehrfach
	vorhanden und sollten in ein eigenes Include-Template extrahiert werden.

\end{description}

\section{Soll-Situation}
\label{analyse:soll-situation}

\subsection{Aufgabe}
Nebst den geforderten Pflichtteilen, der Implementation eines SWID Generators
für Linux Systeme und der Integration der SWID Erweiterung in strongTNC,
schlagen wir nachfolgende Anpassungen vor, die wir bei vorhandener Kapazität
noch umsetzen möchten.

\subsection{Entkopplung der Datenbank}
\label{analyse:architekturen}
Wie bereits erwähnt entstehen durch die gemeinsame Nutzung einer SQLite
Datenbank einige Probleme. Nachfolgend möchten wir kurz auf die daraus
resultierenden Nachteile eingehen.

\begin{description}
	\item[Mehrbenutzerfähigkeit] Die Daten einer SQLite Datenbank können nur von
		einem Prozess gleichzeitig geändert werden. Da es sich bei der aktuellen
		Situation aber bereits um eine Multiprozessumgebung handelt ist dieser
		Einsatz nicht zweckmässig, und kann zu unerwünschten Locks führen.

	\item[Verteilung der Komponenten] Die Policy Decision Komponente (TNC Server
		aus dem strong\-Swan Projekt) kann derzeit nicht getrennt von der strongTNC
		Webserver Komponente betrieben werden. Dies kann in einem Enterprise
		Deployment jedoch durchaus erwünscht sein, beispielsweise wenn die strongTNC
		App auf einem Microsoft Webserver betrieben werden soll, während der Policiy
		Decision Point auf einem Linux System läuft. Zusätzlich kann es aus
		sicherheitstechnischen Gründen angezeigt sein, den VPN Gateway, den Decision
		Point und den strongTNC Webserver in getrennten Netzwerksegmenten zu
		betreiben.

	\item[Anbindung an Drittsysteme] Die Anbindung an Umsysteme gestaltet sich als
		kaum realisierbar. Das TNC Framework sieht aber bereits weitere Komponenten,
		wie beispielsweise IF-MAP Devices vor, die integriert werden könnten. Auch
		eine Integration in bestehende Geschäftsprozesse (zum Beispiel CMDB) ist nur
		schwer realisierbar.
	
	\item[Flexiblere Arbeit mit Django] Wenn die strongTNC Anwendung ihre
		Datenbank nicht mit anderen Systemen teilen muss, können alle
		Datenbank-Abstraktionsfeatures von Django voll ausgenutzt werden. Im Moment
		muss man sich aktiv um das Datenbankschema kümmern. Eine Entkopplung hätte
		zur Folge, dass auch
		\enquote{Migrations}\footnote{\url{https://docs.djangoproject.com/en/1.7/topics/migrations/}}
		verwendet werden können, welche es erlauben Änderungen an den Models
		automatisiert in die Datenbank einfliessen zu lassen.
	
\end{description} 

Aus diesen Gründen möchten wir ein Konzept entwerfen, um die gemeinsame Nutzung
einer SQLite Datenbank aufzuheben.

\subsection{Codequaliät}
Um die Entwicklung und mögliche Weiterentwicklung dieses Projektes einfacher zu
gestalten, soll die Codequalität verbessert werden. Folgendes sind die
Hauptansatzpunkte:
\begin{description}
	\item[Coderichtlinien] Die PEP8 Coderichtlinien sollen im Verlaufe dieses
	Projektes und auch in Zukunft strikt eingehalten werden.
	
	\item[Django Best Practice] Die Möglichkeiten des Django Webframeworks sollen
	besser genutzt werden, in dem aktuelle Konzepte wie Class Based Views,
	separierte Apps oder das Form Framework zum Einsatz kommen.
	
	\item[Testing] Um die Qualität und Stabilität längerfristig zu gewährleisten,
	soll die Testabdeckung erhöht werden.
	
	\item[Continuous Integration] Um stets über den Stand der Software Informiert
	zu sein, soll der Code mit Hilfe von Continuous Integration regelmässig und bei
	jeder Änderung gebuilded und getestet werden.
	
\end{description}


\section{Abgrenzung}
Folgende Punkte werden von uns als gegeben betrachtet oder sind nicht
Bestandteil unserer Arbeit:

\begin{description}
	\item[Datenbankschema] Das Schema wird nicht grundlegend geändert, sondern
	übernommen wie es ist. Da verschiedene Komponenten auf die Datenbank zugreifen,
	kann eine Änderung am Schema im schlechtesten Fall eine Anpassung in allen
	Komponenten mit Datenbankzugriff zur Folge haben.

	\item[Benutzerschnittstelle] Das Frontend der Webapp wird nicht grundlegend
	verändert, es werden Korrekturen, Ergänzungen und Anpassungen vorgenommen,
	jedoch ohne dabei das bestehende Layout grundsätzlich zu ändern.

	\item[strongSwan] Anpassungen und Ergänzungen seitens strongSwan sind nicht
	Bestandteil dieser Arbeit. Änderungen in strongTNC, welche Einfluss auf
	strongSwan haben, werden mit den verantwortlichen strongSwan Entwicklern
	besprochen.
\end{description}


\section{ISO Standard 19770-2}
\label{analyse:swidstandard} 
Der ISO Standard 19770-2\cite{iso19770-2} ist der zweite Teil einer Gruppe von ISO Standards
zu den Prozessen und Technologien des Software Asset Management. Dieser Teil des
Standards beschreibt den Aufbau und die Verwendung von Software Identification
Tags und befindet sich zur Zeit noch im Entwurfsstadium (\enquote{Draft}).

\subsection{Bestandteile eines SWID Tags}
Nachfolgend eine Zusammenstellung der Bestandteile eines SWID Tags, wie sie in
diesem Projekt verwendet werden. Der Standard beschreibt noch weitere Elemente
und Anwendungsmöglichkeiten, welche in dieser Arbeit allerdings nicht verwendet
werden. Die jeweiligen Listen der Attribute sind nicht vollständig, sondern eine
Auswahl, wie sie für diese Arbeit relevant sind. (ISO
19770-2\cite{iso19770-2}, Seite 19, 8.6)

\begin{description}
	\item[SoftwareIdentity] Repräsentation des Wurzelelementes eines SWID Tags.
	Dieses Element hat Attribute, welche die Anwendung des Tags und dessen Identität,
	innerhalb seiner Entität, beschreiben.
	\begin{description}
		\item[delta] Das \texttt{delta} Attribut beschreibt, ob es sich um ein Patch
		oder eine Modifikation der Software handelt. Dieses Attribut wird von uns
		nicht verwendet, da bei Linux Systemen üblicherweise keine Patches verteilt
		werden, sondern gepatchte Softwarepakete. Der Standardwert ist \texttt{false}.

		\item[name] Dieses Pflichtfeld enthält den Namen der Software so wie man
		üblicherweise darauf verweisen würde.
		
		\item[uniqueId] Die \texttt{uniqueId} ist eine Kennung, die eine Software
		innerhalb des Namespaces des Tag Creators eindeutig identifizieren kann.
		Vorschläge für den Aufbau einer \texttt{uniqueId} sind \texttt{publisher +
		product + version} oder eine GUID. Die \texttt{uniqueId} ist optional.
		
		\item[version] In diesem Attribut wird die Version der Software festgelegt.
		Das Attribut ist optional und hat einen Standardwert von \texttt{0.0}.
	\end{description}
	
	\item[Entity] Beschreibt die Organisation, die für diesen SWID Tag
	verantwortlich ist. In einem Tag können theoretisch beliebig viele Entity
	Elemente enthalten sein. Eine Entity mit der Rolle \texttt{tagcreator} (siehe
	Attribut \texttt{role}) ist obligatorisch. Jeder Tag muss mindestens eine
	Entity mit der Rolle \texttt{tagcreator} enthalten. Das \texttt{Entity} Element
	ist ein Kind des \texttt{SoftwareIdentity} Elements.
	\begin{description}
		\item[name] Name der Organisation, die eine bestimmte Rolle in diesem Tag für
		sich beansprucht. Dieses Attribut ist ein Pflichtfeld.
		
		\item[regid] Die \texttt{regid} ist die \enquote{Unique registration ID} einer
		Organisation. Die \texttt{regid} ist grundsätzlich wie folgt aufgebaut:
		\texttt{regid.YYYY-MM.<reverse domain name>}, die vierstellige Jahreszahl und
		der zweistellige Monat ist das Registrierungsdatum der Domain. Für den Aufbau
		einer \texttt{regid} gibt der Standard noch weitere Richtlinien vor, es ist
		auch definiert wie eine \texttt{regid} für Organisationen ohne registrierte
		Domain auzusehen hat. Dieses Attribut ist optional und wird mit dem
		Standardwert \texttt{invalid.unavailable} befüllt.
		
		\item[role] Die Rolle beschreibt, in welcher Beziehung die Organisation zu
		diesem SWID Tag steht. Das \texttt{role} Attribut ist ein Aufzählungstyp mit
		den möglichen Werten \texttt{publisher, tagcreator, licensor} und ist
		obligatorisch.
	\end{description}
	
	\item[Payload] Das \texttt{Payload} Element enthält die zu erwartenden
	Bestandteile dieser Software, wenn sie installiert ist, diese Information ist
	optional. Das \texttt{Payload} Element ist ein Kind des
	\texttt{SoftwareIdentity} Elements und hat keine Attribute.
	
	\item[File] \texttt{File} Elemente sind Kinder des \texttt{Payload} Elements.
	Sie enthalten Informationen zu den Dateien, die einer Software angehören.
	\begin{description}
		\item[name] Name der Datei, ohne Verzeichnis.
		\item[location] Verzeichnis in dem die Datei zu finden ist, dieses Attribut
		ist optional.
	\end{description}
	
\end{description}

\subsection{Wichtige Punkte}
\begin{itemize}
	\item Ein Tag darf nur durch die Organisation modifiziert werden, die ihn
	initial erstellt hat. Diese Organisation hat mindestens die Rolle des
	\enquote{Tag Creator}. Dieser Tag heisst \enquote{Primary Tag}. (ISO
	19770-2\cite{iso19770-2}, Seite 6, 5.3)
	
	\item Wenn ein Tag ergänzt werden soll, muss dies mit Hilfe eines
	\enquote{Supplemental Tags} geschehen, da der \enquote{Primary Tag} nicht
	modifiziert werden darf. ((ISO 19770-2\cite{iso19770-2}, Seite 7, 5.4.2)
	
	\item Ein SWID Tag wird grundsätzlich als XML Datei abgelegt, diese Datei muss
	sich im Installationsverzeichnis der repräsentierten Software befinden. Der Tag
	kann zusätzlich über andere Kanäle wie URIs oder zentrale Verwaltungen
	zugänglich gemacht werden. (ISO 19770-2\cite{iso19770-2}, Seite 10, 6.1.3 und
	Seite 15, 7.1)
	
	\item Als eindeutige Kennung für einen SWID Tag dient die Kombination von
	\texttt{tag\_createor\_regid} und \texttt{uniqueId}. Diese Kennung wird
	\enquote{software\_id} genannt. Es liegt in der Verantwortung des \enquote{Tag
	Creators} dafür zu sorgen, dass seine Tags eindeutig sind. (ISO
	19770-2\cite{iso19770-2}, Seite 10, 6.1.5 und Seite 16, 8.1)
	
	\item Es ist nicht obligatorisch die Echtheit von SWID Tags zu garantieren;
	falls eine Echtheitsvalidierung gewünscht oder nötig ist, kann dies durch den
	Einsatz von der \enquote{XML signature syntax}\cite{bartel2002xml}
	implementiert werden. Echtheitsprüfung ist nicht Bestandteil dieser Arbeit.
	(ISO 19770-2\cite{iso19770-2}, Seite 14, 6.1.19)
	
	\item Die minimale Information, die ein valider SWID Tag enthalten muss, ist
	\texttt{SoftwareIdentity.name} und \texttt{Entity.role}. Bei der Rolle muss es
	sich um die Rolle des \enquote{Tag Creators} handeln. Das empfohlene
	Minimum besteht aus folgenden Informationen:
	\begin{itemize*}
		\item SoftwareIdentity
			\begin{itemize*}
			\item name
			\item tagVersion
			\item uniqueId
			\item version
			\item versionScheme
			\end{itemize*}
		\item Entity
			\begin{itemize*}
			\item name
			\item regid
			\item role
			\end{itemize*}
	\end{itemize*}	
	(ISO 19770-2\cite{iso19770-2}, Seite 16, 8.2)
	
\end{itemize}


\subsection{Probleme}
Bei der Analyse und der Implementation des ISO Standard 19770-2 sind
einige Punkte aufgetaucht in denen sich der Standard widerspricht, etwas unklar
ist oder Schwierigkeiten bei der Implementation verursacht.

\subsubsection{Keine garantierte Identität}
Die eindeutige Kennung eines SWID Tags setzt sich aus der \texttt{regid} des
\enquote{Tag Creators} und der \texttt{uniqueId} zusammen, da jedoch diese
beiden Attribute als optional definiert werden, garantiert der Standard keine
eindeutige Identität für jeden SWID Tag.
\begin{quote}
\textit{(...) so it is up to each tag creator to ensure each of their tags is
unique.}
\end{quote}
\textit{ISO 19770-2\cite{iso19770-2}, Seite 16}\\

Da der \enquote{Tag Creator} für die Eindeutigkeit seiner Tags verantwortlich
ist, muss man davon ausgehen, dass ein erhaltener Tag eindeutig identifizierbar
ist. Das Problem bei Tags mit wenig Informationen liegt allerdings beim Tag
Konsumenten:
\begin{quote}
\textit{A conforming consumer shall not reject any conforming SWID tag
document.}
\end{quote}
\textit{ISO 19770-2\cite{iso19770-2}, Seite 4}\\

Vom Tag Ersteller wird lediglich erwartet, dass die produzierten Tags
standardkonform sind:
\begin{quote}
\textit{A conforming producer shall be able to produce SWID tag documents
conforming with this part of ISO/IEC 19770.}
\end{quote}
\textit{ISO 19770-2\cite{iso19770-2}, Seite 4}\\

Diese Anforderung ist bereits erfüllt, wenn der Name der Software und der
Organisation, welche den Tag erstellt hat, vorhanden sind.

In dieser Arbeit ist dieses Problem dadurch entschärft, dass der SWID Generator
die einzige Tag creator Instanz ist und damit eindeutige Tags kreiert. Sobald
jedoch mehrere Tag creators vorhanden sind, und die Eindeutigkeit nicht mehr
zentral geregelt ist, können Probleme entstehen. Des Weiteren muss davon
ausgegangen werden, dass es im Interesse des Tag Erstellers liegt, dass seine
SWID Tags nicht Ursache eines Konfliktes werden.

\subsubsection{SWID Tag XML Dateien}
\begin{quote}
\textit{SWID tag data is stored in an XML file and shall be located on a devices
file system in the same file directory as the application they represent.}
\end{quote}
\textit{ISO 19770-2\cite{iso19770-2}, Seite 10}\\

Diese Anforderung ist bei Linux Systemen nicht direkt umsetzbar, da
Softwarepakete unter Linux oft nicht nur in einem Verzeichnis installiert
werden, sondern Dateien in verschiedene Verzeichnisse kopieren.

In dieser Arbeit wird diese Anforderung nicht beachtet. Der SWID Generator
kreiert Tags dynamisch aus den Informationen des Linux Paketmanagers
und liefert diese direkt auf die Standardausgabe der Konsole. Die SWID Tags
werden nicht in einer Datei gespeichert. 

\subsubsection{Doppelpunkte in der UniqueId} 
Der SWID Generator setzt die \texttt{uniqueId} aus dem Betriebsystemnamen, der
Prozessorarchitektur, dem Paketnamen sowie der Versionsnummer zusammen:
\texttt{<OS>\_<Arch>\_<Package>\_<Version>}. Versionen unter der Linux
Distribution \enquote{Ubuntu} enthalten oft Sonderzeichen wie Doppelpunkte oder
Plus-Zeichen. Anhand der folgenden Aussage aus dem Standard kann man darauf
schliessen, dass diese Zeichen in einer \texttt{uniqueId} nicht erlaubt sind.

\begin{quote}
\textit{(...) may be either a GUID, or any reference unique
for the tag\_creator\_regid. The unique\_id shall follow the restrictions for
URI character use as specified in IETF RFC 3986, section 2, characters.}
\end{quote} 
\textit{ISO 19770-2\cite{iso19770-2}, Seite 13}\\

Die \texttt{uniqueId} unterliegt den Restriktionen einer URI. Für diese gilt: ein
Doppelpunkt ist kein unerlaubtes Zeichen jedoch ein reserviertes:

\begin{verbatim} 
reserved = gen-delims / sub-delims 
gen-delims = ":" / "/" / "?"/ "#" / "[" / "]" / "@" 
sub-delims = "!" / "\$" / "&" / "'" / "(" / ")" / "*" /"+" / "," / ";" / "=" 
unreserved = ALPHA / DIGIT / "-" / "." / "_ " / "~"
\end{verbatim}
\textit{IETF RFC 3986\cite{berners2005rfc}}\\

Die \texttt{uniqueId} kann als \texttt{href} Attribut des \texttt{Link} Tags, in Form
einer URI mit einem \texttt{swid:} Prefix verwendet werden. Daraus folgt, dass
die \texttt{uniqueId} zum \enquote{Authority} Teil einer URI gehört und somit
keine Doppelpunkte enthalten darf.

In dieser Arbeit werden daher Zeichen, welche in URIs als reserviert gelten, in
der \texttt{uniqueId} durch eine Tilde (\texttt{~}) ersetzt.
