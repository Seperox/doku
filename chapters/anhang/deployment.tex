\section{strongTNC Deployment Manual}
\label{anhang:deployment-manual}

Due to security- and performance-considerations, strongTNC should never be
deployed into production using the \texttt{./manage.py runserver} command.
Instead, a proper WSGI Webserver should be used, and all debug settings should
be turned off.

\textbf{Warning: Never deploy strongTNC in production without using SSL/TLS to
secure your connections, especially if you use the API.}

The following how-to assumes you're using Ubuntu or a similar Linux
distribution, but the basic concepts can be applied to any Linux distribution.

(Note: If you just want to try strongTNC, you can also use the automatially
configured testing VMs. See \texttt{vm/README.rst} for more information.)

\subsection*{Install the base system}

Set up your base Ubuntu system. Make sure all the packages are up to date.

\subsection*{Install required packages}

Install the dependencies for strongTNC:

\begin{bashcode}
sudo apt-get install wget build-essential apache2 libapache2-mod-wsgi python2.7 \
    python2.7-dev python-pip python-virtualenv libxml2-dev libxslt1-dev
\end{bashcode}

If you want to use MySQL instead of SQLite, you need a few additional
dependencies:

\begin{bashcode}
sudo apt-get install mysql-server mysql-client libmysqlclient-dev
\end{bashcode}

\subsection*{Create directories}

\begin{bashcode}
sudo mkdir /var/www/strongTNC /etc/strongTNC
sudo chown $(whoami):www-data /var/www/strongTNC /etc/strongTNC
sudo chmod 775 /var/www/strongTNC /etc/strongTNC
\end{bashcode}

\subsection*{Download current strongTNC release}

Download the current snapshot of the master branch from Github:

\begin{bashcode}
cd /tmp
wget https://github.com/strongswan/strongTNC/archive/master.tar.gz
tar xfvz master.tar.gz
mv strongTNC-master/* /var/www/strongTNC/
\end{bashcode}

If you want you could also use git to clone the repository.

\subsection*{Install Python dependencies}

The recommended way to install the Python dependencies is to put them in a
Virtualenv\footnote{\url{http://virtualenv.readthedocs.org/}}.

\begin{bashcode}
cd /var/www/strongTNC
virtualenv --no-site-packages VIRTUAL
VIRTUAL/bin/pip install -U -r requirements.txt
\end{bashcode}

If you use MySQL, you need an additional Python package:

\begin{bashcode}
VIRTUAL/bin/pip install -U MySQL-python
\end{bashcode}

\subsection*{strongTNC configuration}

Copy the sample configuration to /etc:

\begin{bashcode}
cp /var/www/strongTNC/config/settings.sample.ini /etc/strongTNC/settings.ini
cd /etc/strongTNC/
sudo chown $(whoami):www-data settings.ini
sudo chmod 640 settings.ini
\end{bashcode}

Now edit the configuration file with your favorite text editor. First of all,
update the database configuration. If you want to use SQLite, set them to the
following values:

\begin{bashcode}
DJANGO_DB_URL = sqlite:////var/www/strongTNC/django.db
STRONGTNC_DB_URL = sqlite:////var/www/strongTNC/ipsec.config.db
\end{bashcode}

For MySQL, use the following values:

\begin{bashcode}
DJANGO_DB_URL = mysql://strongtnc:<mysql-password>@127.0.0.1/strongtnc_django
STRONGTNC_DB_URL = mysql://strongtnc:<mysql-password>@127.0.0.1/strongtnc_data
\end{bashcode}

(Choose a secure password and remember it for later, when we create the MySQL
user!)

Now you need to set a value for \texttt{SECRET\_KEY}, which is used by Django to
encrypt all kinds of stuff. A way to generate such a key is the following code
snippet:

\begin{bashcode}
dd if=/dev/urandom bs=128 count=1 2>/dev/null | base64 -w 175
\end{bashcode}

Then set \texttt{ALLOWED\_HOSTS} to a list of hostnames that will be allowed to
serve strongTNC, e.g.

\begin{bashcode}
ALLOWED_HOSTS = 127.0.0.1,strongtnc.example.org
\end{bashcode}

If you enable SSL/TLS for your setup (you really should!), enable secure CSRF
cookies:

\begin{bashcode}
CSRF_COOKIE_SECURE = 1
\end{bashcode}

You should also take a look at the \texttt{[admins]} section and add your name
and e-mail address there.

\subsection*{Set up database}

\subsubsection*{SQLite}

If you're using SQLite, all you need to do is changing the database permissions:

\begin{bashcode}
cd /var/www/strongTNC/
sudo chgrp www-data django.db ipsec.config.db
sudo chmod 660 django.db ipsec.config.db
\end{bashcode}

\subsubsection*{MySQL}

First, log in to the MySQL console with the root user:

\begin{bashcode}
mysql -u root -p
\end{bashcode}

Create the required databases:

\begin{sqlcode}
mysql> CREATE DATABASE strongtnc_django CHARACTER SET utf8 COLLATE utf8_unicode_ci;
mysql> CREATE DATABASE strongtnc_data CHARACTER SET utf8 COLLATE utf8_unicode_ci;
\end{sqlcode}

Create a new user (make sure to replace \texttt{<password>} with the previously
chosen MySQL password):

\begin{sqlcode}
mysql> GRANT ALL PRIVILEGES ON strongtnc_django.* TO strongtnc@localhost
-> IDENTIFIED BY '<password>';
mysql> GRANT ALL PRIVILEGES ON strongtnc_django.* TO strongtnc@localhost
-> IDENTIFIED BY '<password>';
\end{sqlcode}

Create the necessary schema in your database:

\begin{bashcode}
cd /var/www/strongTNC/
VIRTUAL/bin/python manage.py syncdb --database=meta --noinput
VIRTUAL/bin/python manage.py syncdb --database=default --noinput
\end{bashcode}

\subsection*{Collect static files}

Run the following command to collect all static files in a single directory:

\begin{bashcode}
cd /var/www/strongTNC/
VIRTUAL/bin/python manage.py collectstatic --noinput
\end{bashcode}

\subsection*{Apache configuration}

Write the following configuration to \texttt{/etc/apache2/sites-available/strongTNC}

\begin{apachecode}
WSGIPythonPath /var/www/strongTNC:/var/www/strongTNC/VIRTUAL/lib/python2.7/site-packages

NameVirtualHost *:80
<VirtualHost *:80>
    RewriteEngine On
    RewriteCond %{HTTPS} off
    RewriteRule (.*) https://%{HTTP_HOST}%{REQUEST_URI}
</VirtualHost>

<VirtualHost _default_:443>
    # The ServerName directive sets the request scheme, hostname and port that
    # the server uses to identify itself. This is used when creating
    # redirection URLs. In the context of virtual hosts, the ServerName
    # specifies what hostname must appear in the request's Host: header to
    # match this virtual host. For the default virtual host (this file) this
    # value is not decisive as it is used as a last resort host regardless.
    # However, you must set it for any further virtual host explicitly.
    #ServerName strongtnc.example.com

    SSLEngine on
    SSLCertificateFile /etc/apache2/ssl/strongtnc.crt
    SSLCertificateKeyFile /etc/apache2/ssl/strongtnc.key
    SSLProtocol all -SSLv2 -SSLv3
    SSLHonorCipherOrder on
    SSLCompression off
    SSLCipherSuite "EECDH+ECDSA+AESGCM EECDH+aRSA+AESGCM EECDH+ECDSA+SHA384 \
        EECDH+ECDSA+SHA256 EECDH+aRSA+SHA384 EECDH+aRSA+SHA256 EECDH+aRSA+RC4 \
        EECDH EDH+aRSA RC4 !aNULL !eNULL !LOW !3DES !MD5 !EXP !PSK !SRP !DSS"
    Header add Strict-Transport-Security "max-age=15768000"

    ServerAdmin webmaster@localhost
    DocumentRoot /var/www/strongTNC

    <Directory /var/www/strongTNC>
        <Files wsgi.py>
            Order deny,allow
            Allow from all
        </Files>
        Options -Indexes
    </Directory>

    WSGIScriptAlias / /var/www/strongTNC/config/wsgi.py
    Alias /static/ /var/www/strongTNC/static/

    WSGIPassAuthorization On

    ErrorLog ${APACHE_LOG_DIR}/error.log
    CustomLog ${APACHE_LOG_DIR}/access.log combined
</VirtualHost>

# vim: syntax=apache ts=4 sw=4 sts=4 sr noet
\end{apachecode}

Then disable the default configuration and enable strongTNC:

\begin{bashcode}
sudo a2dissite 000-default
sudo a2ensite strongTNC
\end{bashcode}

Enable necessary plugins and create ssl directory:

\begin{bashcode}
sudo a2enmod ssl rewrite headers
sudo mkdir /etc/apache2/ssl
\end{bashcode}

Copy your TLS certificate and the private key to \texttt{/etc/apache2/ssl}. If
you want to create self-signed certificates, execute the following command:

\begin{bashcode}
sudo openssl req -x509 -nodes -sha256 -days 365 -newkey rsa:3072 -utf8 \
    -keyout /etc/apache2/ssl/strongtnc.key -out /etc/apache2/ssl/strongtnc.crt
\end{bashcode}

Make sure the permissions are restrictive:

\begin{bashcode}
sudo chown root:root /etc/apache2/ssl/*
sudo chmod 400 /etc/apache2/ssl/*
\end{bashcode}

Now restart Apache and strongTNC should be up and running!

\begin{bashcode}
sudo service apache2 restart
\end{bashcode}

\subsection*{Create default users}

In order to be able to login into strongTNC, you need to set a password for a
readonly user and an admin user.

\begin{bashcode}
cd /var/www/strongTNC/
VIRTUAL/bin/python manage.py setpassword
\end{bashcode}

Visit \texttt{http://yourserver/} to log in.
