\subsection{Danilo Bargen}
Phasellus ullamcorper ipsum rutrum nunc. Aliquam erat volutpat. Proin magna.
Praesent congue erat at massa.

Morbi mattis ullamcorper velit. Pellentesque dapibus hendrerit tortor.
Suspendisse feugiat. Nullam sagittis.

Mauris sollicitudin fermentum libero. Duis lobortis massa imperdiet quam.
Phasellus tempus. Aenean imperdiet.

Cras id dui. Duis arcu tortor, suscipit eget, imperdiet nec, imperdiet iaculis,
ipsum. Praesent ut ligula non mi varius sagittis. Praesent vestibulum dapibus
nibh.


\subsection{Christian Fässler} 
An erster Stelle möchte ich bei meinem Teamkollegen Jonas Furrer für die super
Zusammenarbeit und stetige Motivation, und bei Herrn Stoop für die sehr
angenehme und wohlwollende Betreuung bedanken. Der Spass und die Motivation in
Arbeit steht und fällt und der Regel mit dem Team. Da wir in dieser
Zusammensetzung bereits einige Arbeiten umgesetzt haben, waren die
Voraussetzungen optimal! Entsprechend erfolgreich war auch die Zusammenarbeit
und die gegenseitige Unterstützung. Die forschende und kreative Arbeitsweise,
die für diese Arbeit erforderlich war, war für mich Neuland. Sehr ungewohnt war
die Tatsache, dass es wenig bestehende "Ergebnisse" in dem bearbeitenden Gebiet
gibt, auf die zurückgegriffen werden kann.  Dies erforderte einerseits das
Selbstbewusstsein etwas neues auszuprobieren und die erreichten Ergebnisse als
Fortschritte zu betrachten, obwohl kein Mass zum Vergleich vorhanden ist.
Gleichzeitig  brauchte es dennoch eine gesunde Selbstkritik, um das erreichte
objektiv zu betrachten. Das Bewusstsein über diese Vorgehensweise war ein
wichtiger Teil unserer Arbeit. Entsprechend neu waren auch die Werkzeuge, die
wir eingesetzt haben: Graphentheorie und Mathematica. Mit Mathematica habe ich
ein wertvolles Tool kennengelernt, welches ich sonst wohl nie in diesem Umfang
kennengelernt hätte. Dass der Schwerpunkt der Arbeit in einem rein
mathematischen Gebiet liegt, und nicht der in klassischen Informatik/Software
Entwicklung war sehr spannend. Denn die mathematischen Themen, die während des
Studium erarbeitet werden, können selten in Software Projekten angewendet
werden. Daher war es schön, Teile dieses Wissen einmal anwenden zu können.
Obwohl die Zeit von 14 Wochen sehr schnell vorbei ging, war es spannend etwas
tiefer in ein neues Gebiet einzutauchen. Ich hoffe, mit den Ergebnissen unserer
Arbeit hilfreiche Grundlagen für weitergehende Arbeiten liefern zu können.


\subsection{Jonas Furrer} 



\enquote{Endpoint Compliance Monitoring based on Software Identification Tags}, dieses Thema klingt spannend, doch zu Beginn dieser Bachelorarbeit konnte ich mir noch nichts genaues darunter vorstellen. Nach einem Gespräch mit Prof. Dr. Steffen wurde klar in welche Richtung die Arbeit gehen würde.\\
Wir hatten die Gelegenheit eine Applikation 

Wir hatten die Gelegenheit uns einen
Bereich einzuarbeiten wo noch vieles offen ist. Damit begann die Einarbeitung in
die diversen Themen rund um Graphentheorie und deren Phänomene, Musiktheorie,
Mathematik und die Arbeit mit Wolfram Mathematica. Nach einer kurzen
Einarbeitungszeit konnten bereits erste Ergebnisse erzielt werden, doch mit den
Ergebnissen kam auch die Erkenntiss, dass weder das Themengebiet noch die
Grenzen der Arbeit klar abzustecken sind. Mit dieser Tatsache hatte ich zu
beginn Mühe, ich bin es mir Gewohnt Lösungsorientiert zu arbeiten. Diese Aufgabe
forderte aber einen Problemorientierten Ansatz. Als ich erkannte, dass wir nicht
auf \enquote{eine richtige} Lösung hin arbeiten, sondern darauf, das Problem zu
erkennen und wege zu einer Lösungsmöglichkeit zu finden, betrachtete ich die
Arbeit aus einer anderen Perspektive und konnte offener an die Problemstellung
heran gehen. Durch diese offene Herangehensweise kam mit jeder Idee ein neues
Problem, und mit jedem Problem zwei neue Ideen. Wir mussten uns auf einige Ideen
einschränke und diese genauer betrachten. Es fiel mir nicht einfach
Eintscheidungen zu treffen, da man nie genau sagen konnte ob eine Idee oder
deren Ergebnisse richtig oder falsch sind. Die Arbeit hat durch das grosse
Potential des Themengebietes noch viele Bereich die Vertieft werden konnten. Wir
hatten nicht die Zeit alle Ideen und Ansätze umzusetzen, einige Möglichkeiten
und Anätze konnte ich erst gegen Ende der Arbeit erkennen, da anfangs das
Grundwissen fehlte. Ich konnte von dieser Arbeit in vielerlei Hinsicht
profitieren: Ich hatte die Glegenheit mich in ein interessantes Themengebiet
einzuarbeiten und neue Werkzeuge auszuprobieren, ich habe eine für mich
ungewohnte Herangehensweise für Projekte erlebt, von den Diskusionen mit
Christian Fässler und Prof. Stoop konnte ich stets profitieren.
