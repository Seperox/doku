\subsection{Danilo Bargen}

Das Thema der vorliegenden Bachelorarbeit hat mir sehr zugesagt. Einerseits
interessiere ich mich für Security-Themen und habe deshalb bereits während dem
Studium alle Module zu Informationssicherheit an der HSR besucht. Andererseits
habe ich bereits mehrere Jahre Entwicklungserfahrung mit Python und dem
Django-Framework, das kam mir natürlich sehr entgegen.

Sehr gefreut habe ich mich auch darüber, dass es möglich war, die Arbeit zu
dritt mit meinen Mitstudenten Christian und Jonas durchzuführen. Dadurch, dass
wir in der Vergangenheit bereits mehrere Arbeiten erfolgreich zusammen
abgeschlossen haben, waren wir bereits ein eingespieltes Team und die
Zusammenarbeit klappte reibungslos. An dieser Stelle möchte ich ihnen ein
grosses Dankeschön aussprechen für die gemeinsame Zeit. \\
Auch unseren beiden Betreuern -- Prof. Dr. Andreas Steffen und Tobias Brunner --
möchte ich meinen Dank aussprechen für die gute Begleitung unserer Arbeit.

Leider verlief diese Bachelorarbeit nicht ganz ohne Hindernisse. In der vierten
Semesterwoche zog ich mir an meiner praktischen Gleitschirm-Brevetprüfung einen
komplizierten Ellbogenbruch zu, worauf ich zwei Wochen im Kantonsspital St.
Gallen verbringen musste. Zudem konnte ich auch während der zwei darauffolgenden
Wochen aufgrund der Operationswunde eine Tastatur nur einhändig bedienen.\\
Glücklicherweise konnten wir uns im Entwicklungs-Team gut mit der Situation
arrangieren. Während des Spitalaufenthaltes konnte ich über einen Tablet
Computer Code Reviews durchführen und später klappte auch das einhändige
Programmieren nach kurzer Eingewöhnungszeit relativ gut.

Obwohl wir zuletzt noch mehrere Nachtschichten einlegen mussten um unseren
Ansprüchen an die Dokumentation zu genügen, blicke ich positiv auf die
vergangene Zeit zurück und hoffe, dass die geleistete Arbeit nicht vergebens
war, sondern dass die resultierenden Softwareprodukte auch wirklich erfolgreich
in der Praxis eingesetzt werden können.



\subsection{Christian Fässler} 
An erster Stelle möchte ich bei meinen Teamkollegen Jonas Furrer und Danilo
Bargen für die super Zusammenarbeit und ihr Engagement bedanken. Da wir in
dieser Zusammensetzung bereits einige Arbeiten umgesetzt haben, waren die
Voraussetzungen für diese Arbeit optimal! Entsprechend erfolgreich war auch die
Zusammenarbeit und die gegenseitige Unterstützung.\\ Herzlichen Dank auch an die
Betreuer Prof. Dr. Andreas Steffen und Tobias Brunner für die gute Betreuung der
Arbeit. Durch die wertvollen und zeitnahen Feedbacks hatte ich stets den
Eindruck, dass unsere Arbeit sehr geschätzt wurde.

Zu Beginn der Arbeit war Danilo Bargen wegen eines Unfalls leider während 2
Wochen im Spital. Durch das eingespielte Team war dies kein Problem und wir
konnten uns schnell arrangieren.

Im Rahmen des Aufbaustudiums habe ich die Sicherheitsmodule an der HSR besucht.
Deshalb hat mir das Thema dieser Arbeit sehr zugesagt. Python-Kenntnisse konnte
ich bereits beim Start in die Arbeit mitbringen. Das Django Framework war für
mich jedoch komplettes Neuland. Spannend an dieser Arbeit war die Mischung aus
dem Entwickeln einer kompletten Standalone-Anwendung (SWID-Generator) und dem
Erweitern einer bestehenden Applikation. Ich bin sehr zufrieden mit dem
erreichten Resultat. Umso schöner finde ich es, dass es sich um ein Produkt
handelt, welches voraussichtlich auch in der Praxis eingesetzt wird.

Ich hoffe, mit den Ergebnissen unserer Arbeit hilfreiche Grundlagen für
weitergehende Arbeiten liefern zu können.


\subsection{Jonas Furrer} 



\enquote{Endpoint Compliance Monitoring based on Software Identification Tags}, dieses Thema klingt spannend, doch zu Beginn dieser Bachelorarbeit konnte ich mir noch nichts genaues darunter vorstellen. Nach einem Gespräch mit Prof. Dr. Steffen wurde klar in welche Richtung die Arbeit gehen würde.\\
Wir hatten die Gelegenheit eine Applikation 

Wir hatten die Gelegenheit uns einen
Bereich einzuarbeiten wo noch vieles offen ist. Damit begann die Einarbeitung in
die diversen Themen rund um Graphentheorie und deren Phänomene, Musiktheorie,
Mathematik und die Arbeit mit Wolfram Mathematica. Nach einer kurzen
Einarbeitungszeit konnten bereits erste Ergebnisse erzielt werden, doch mit den
Ergebnissen kam auch die Erkenntiss, dass weder das Themengebiet noch die
Grenzen der Arbeit klar abzustecken sind. Mit dieser Tatsache hatte ich zu
beginn Mühe, ich bin es mir Gewohnt Lösungsorientiert zu arbeiten. Diese Aufgabe
forderte aber einen Problemorientierten Ansatz. Als ich erkannte, dass wir nicht
auf \enquote{eine richtige} Lösung hin arbeiten, sondern darauf, das Problem zu
erkennen und wege zu einer Lösungsmöglichkeit zu finden, betrachtete ich die
Arbeit aus einer anderen Perspektive und konnte offener an die Problemstellung
heran gehen. Durch diese offene Herangehensweise kam mit jeder Idee ein neues
Problem, und mit jedem Problem zwei neue Ideen. Wir mussten uns auf einige Ideen
einschränke und diese genauer betrachten. Es fiel mir nicht einfach
Eintscheidungen zu treffen, da man nie genau sagen konnte ob eine Idee oder
deren Ergebnisse richtig oder falsch sind. Die Arbeit hat durch das grosse
Potential des Themengebietes noch viele Bereich die Vertieft werden konnten. Wir
hatten nicht die Zeit alle Ideen und Ansätze umzusetzen, einige Möglichkeiten
und Anätze konnte ich erst gegen Ende der Arbeit erkennen, da anfangs das
Grundwissen fehlte. Ich konnte von dieser Arbeit in vielerlei Hinsicht
profitieren: Ich hatte die Glegenheit mich in ein interessantes Themengebiet
einzuarbeiten und neue Werkzeuge auszuprobieren, ich habe eine für mich
ungewohnte Herangehensweise für Projekte erlebt, von den Diskusionen mit
Christian Fässler und Prof. Stoop konnte ich stets profitieren.
