\subsection{Danilo Bargen}

Das Thema der vorliegenden Bachelorarbeit hat mir sehr zugesagt. Einerseits
interessiere ich mich für Security-Themen und habe deshalb bereits während dem
Studium alle Module zu Informationssicherheit an der HSR besucht. Andererseits
habe ich bereits mehrere Jahre Entwicklungserfahrung mit Python und dem
Django-Framework, das kam mir natürlich sehr entgegen.

Sehr gefreut habe ich mich auch darüber, dass es möglich war, die Arbeit zu
dritt mit meinen Mitstudenten Christian und Jonas durchzuführen. Dadurch, dass
wir in der Vergangenheit bereits mehrere Arbeiten erfolgreich zusammen
abgeschlossen haben, waren wir bereits ein eingespieltes Team und die
Zusammenarbeit klappte reibungslos. An dieser Stelle möchte ich ihnen ein
grosses Dankeschön aussprechen für die gemeinsame Zeit. \\
Auch unseren beiden Betreuern -- Prof. Dr. Andreas Steffen und Tobias Brunner --
möchte ich meinen Dank aussprechen für die gute Begleitung unserer Arbeit.

Leider verlief diese Bachelorarbeit nicht ganz ohne Hindernisse. In der vierten
Semesterwoche zog ich mir an meiner praktischen Gleitschirm-Brevetprüfung einen
komplizierten Ellbogenbruch zu, worauf ich zwei Wochen im Kantonsspital St.
Gallen verbringen musste. Zudem konnte ich auch während der zwei darauffolgenden
Wochen aufgrund der Operationswunde eine Tastatur nur einhändig bedienen.\\
Glücklicherweise konnten wir uns im Entwicklungs-Team gut mit der Situation
arrangieren. Während des Spitalaufenthaltes konnte ich über einen Tablet
Computer Code Reviews durchführen und später klappte auch das einhändige
Programmieren nach kurzer Eingewöhnungszeit relativ gut.

Obwohl wir zuletzt noch mehrere Nachtschichten einlegen mussten um unseren
Ansprüchen an die Dokumentation zu genügen, blicke ich positiv auf die
vergangene Zeit zurück und hoffe, dass die geleistete Arbeit nicht vergebens
war, sondern dass die resultierenden Softwareprodukte auch wirklich erfolgreich
in der Praxis eingesetzt werden können.



\subsection{Christian Fässler} 
An erster Stelle möchte ich bei meinen Teamkollegen Jonas Furrer und Danilo
Bargen für die super Zusammenarbeit und ihr Engagement bedanken. Da wir in
dieser Zusammensetzung bereits einige Arbeiten umgesetzt haben, waren die
Voraussetzungen für diese Arbeit optimal! Entsprechend erfolgreich war auch die
Zusammenarbeit und die gegenseitige Unterstützung.\\ Herzlichen Dank auch an die
Betreuer Prof. Dr. Andreas Steffen und Tobias Brunner für die gute Betreuung der
Arbeit. Durch die wertvollen und zeitnahen Feedbacks hatte ich stets den
Eindruck, dass unsere Arbeit sehr geschätzt wurde.

Zu Beginn der Arbeit war Danilo Bargen wegen eines Unfalls leider während 2
Wochen im Spital. Durch das eingespielte Team war dies kein Problem und wir
konnten uns schnell arrangieren.

Im Rahmen des Aufbaustudiums habe ich die Sicherheitsmodule an der HSR besucht.
Deshalb hat mir das Thema dieser Arbeit sehr zugesagt. Python-Kenntnisse konnte
ich bereits beim Start in die Arbeit mitbringen. Das Django Framework war für
mich jedoch komplettes Neuland. Spannend an dieser Arbeit war die Mischung aus
dem Entwickeln einer kompletten Standalone-Anwendung (SWID-Generator) und dem
Erweitern einer bestehenden Applikation. Ich bin sehr zufrieden mit dem
erreichten Resultat. Umso schöner finde ich es, dass es sich um ein Produkt
handelt, welches voraussichtlich auch in der Praxis eingesetzt wird.

Ich hoffe, mit den Ergebnissen unserer Arbeit hilfreiche Grundlagen für
weitergehende Arbeiten liefern zu können.


\subsection{Jonas Furrer} 
Die Möglichkeit an einem Produkt zu arbeiten das produktiv eingesetzt wird --
und dazu im IT Security Bereich -- hat mich sehr angesprochen. Dazu kommt, dass
die verwenden Technologien so zusammengestellt waren, dass ich einerseits viel
lernen aber auch vorhandenes Wissen einbringen konnte.

Die Implementation eines ISO Standards war für mich eine interessante Erfahrung.
Ich hätte bis vor dieser Arbeit nicht geglaubt, dass ich mich jemals so intensiv
mit einem Standard-Dokument beschäftigen würde. Ich war zuvor der Meinung, dass
Standards unbezwingbare Dokumente seien, doch diese Arbeit konnte mich vom
Gegenteil überzeugen.

Die Zusammenarbeit mit meinen Kollegen Christian und Danilo war motivierend und
hori\-zont\-erweiternd. Dies war jedoch nicht die erste Arbeit die wir in dieser
Konstellation durchgeführt haben, darum funktionierte die Zusammenarbeit
reibungslos und wir wussten bereits wo unsere Stärken und Schwächen liegen. Für
diese Zeit möchte ich mich bei meinen Kollegen herzlich bedanken.

Die Zusammenarbeit mit Prof. Dr. Andreas Steffen und Tobias Brunner war
ebenfalls sehr angenehm. Als besonders gut empfand ich die Art der
Zusammenarbeit, Herr Steffen gab immer zeitnah und direkt Rückmeldung zu unserer
Arbeit, er erstellte auch direkt Github Issues und förderte so die dynamische
Art unserer Projektarbeit. Die Rückmeldungen von Tobias zu unserem Code war
immer sehr präzise und lehrreich. \\ Für diese Zusammenarbeit möchte ich Herrn
Steffen und Tobias herzlich danken.
